%%%%%%%%%%%%%%%%%%%%%%%%%%%%%%%%%%%%%%%%%%%%%%%%%%%%%%%%%%%%%%%%%%%%%%%%%%%%%
%                                                                           %
%          LaTeX File for Doctor (Master) Thesis of ECNU                    %
%            华东师范大学博士(硕士)论文模板 ____lizb                        %
%                                                                           %
%%%%%%%%%%%%%%%%%%%%%%%%%%%%%%%%%%%%%%%%%%%%%%%%%%%%%%%%%%%%%%%%%%%%%%%%%%%%%
%!TEX program = xelatex
\documentclass[12pt,openany,a4paper,fancyhdr,oneside]{ctexbook}
%\documentclass[12pt,openright,a4paper,fancyhdr,twoside]{ctexbook}
%draft 选项可以使插入的图形只显示外框,以加快预览速度。
%\documentclass[11pt,a4paper,openany,draft]{book}

\usepackage{multirow}
\usepackage{listings}
\usepackage{xcolor}

\usepackage[CJKbookmarks,linkcolor=black,citecolor=black]{hyperref}
\usepackage{shortvrb,indentfirst,ulem,makeidx}
\usepackage{fancyhdr}
\usepackage{graphicx}
\usepackage{indentfirst,latexsym,amsthm,colortbl,subfigure,clrscode}
\usepackage{algorithm}
\usepackage{algorithmic}
\usepackage{bm}                     % 处理数学公式中的黑斜体的宏包
\usepackage{amsmath}                % AMSLaTeX宏包 用来排出更加漂亮的公式
\usepackage{amssymb}                % AMSLaTeX宏包 用来排出更加漂亮的公式
\usepackage{mathrsfs}
\usepackage[subnum]{cases}
\usepackage[numbers,sort&compress]{natbib}
%\usepackage[super,square,numbers,sort&compress]{natbib}
\usepackage{hypernat}
\usepackage{geometry}

\usepackage{times}
\usepackage{fontspec}
\usepackage{libertine}

\usepackage{caption}
\usepackage{titletoc}
\usepackage{titlesec}


\usepackage{inputenc}
\usepackage{listings}
\usepackage{color}
\usepackage{fontspec}

\usepackage{booktabs}%表格
\usepackage{colortbl}%表格虚线
\usepackage{arydshln}%表格虚线

\usepackage{multirow}%表格虚线
\usepackage{multicol}%表格虚线
%\usepackage{url}%参考文献网址


\makeindex
\pagestyle{fancy}

\fancyhead[RO,LE]{\bfseries 华东师范大学研究生硕士学位论文}
\fancyhead[LO]{\small \leftmark} \fancyhead[RE]{\small \leftmark}

\renewcommand{\headrulewidth}{0.4pt}
\fancyfoot[CO,CE]{\thepage}

\renewcommand{\algorithmicrequire}{\textbf{Input:}}
\renewcommand{\algorithmicensure}{\textbf{Output:}}
\renewcommand{\algorithmiccomment}[1]{// #1}


%                    根据自己正文需要做的一些定义                 %
%==================================================================
\def\diag{{\rm diag}}
\def\rank{{\rm rank}}
\def\RR{{\cal R}}
\def\NN{{\cal N}}
\def\R{{\mathbb R}}
\def\C{{\mathbb C}}
\let\dis=\displaystyle

\def\p{\partial}
\def\f{\frac}
\def\mr{\mathrm}

\def\mb{\mathbf}
\def\mc{\mathcal}
\def\b{\begin}
\def\e{\end}

\newtheorem{thm1}{Theorem}[part]
\newtheorem{thm2}{Theorem}[section]
\newtheorem{thm3}{Theorem}[subsection]
\newtheorem{them}[thm2]{定理}
\newtheorem{theorem}[thm2]{定理}
\newtheorem{defn}[thm2]{定义}
\newtheorem{define}[thm2]{定义}
\newtheorem{ex}[thm2]{例}
\newtheorem{exs}[thm2]{例}
\newtheorem{example}[thm2]{例}
\newtheorem{prop}[thm2]{命题}
\newtheorem{lemma}[thm2]{引理}
\newtheorem{cor}[thm2]{推论}
\newtheorem{remark}[thm2]{注释}
\newtheorem{notation}[thm2]{记号}
\newtheorem{abbre}[thm2]{缩写}
% \newtheorem{algorithm}[thm2]{算法}
\newtheorem{problem}[thm2]{问题}

\newcommand{\tabincell}[2]{\begin{tabular}{@{}#1@{}}#2\end{tabular}}

\newcommand{\yihao}{\fontsize{26pt}{36pt}\selectfont}           % 一号, 1.4 倍行距
\newcommand{\erhao}{\fontsize{22pt}{28pt}\selectfont}          % 二号, 1.25倍行距
\newcommand{\xiaoer}{\fontsize{18pt}{18pt}\selectfont}          % 小二, 单倍行距
\newcommand{\sanhao}{\fontsize{16pt}{24pt}\selectfont}        % 三号, 1.5倍行距
\newcommand{\xiaosan}{\fontsize{15pt}{22pt}\selectfont}        % 小三, 1.5倍行距
\newcommand{\sihao}{\fontsize{14pt}{21pt}\selectfont}            % 四号, 1.5 倍行距
\newcommand{\banxiaosi}{\fontsize{13pt}{19.5pt}\selectfont}    % 半小四, 1.5倍行距
\newcommand{\xiaosi}{\fontsize{12pt}{18pt}\selectfont}            % 小四, 1.5倍行距
\newcommand{\dawuhao}{\fontsize{11pt}{11pt}\selectfont}       % 大五号, 单倍行距
\newcommand{\wuhao}{\fontsize{10.5pt}{15.75pt}\selectfont}    % 五号, 单倍行距



%============================ 可以自定义文字块 ================================%

\newcommand{\aaa}{这是我给你们的一个示例}
\newcommand{\bbb}{\aaa \aaa \aaa}
\newcommand{\ccc}{\bbb \bbb \bbb \bbb \bbb

\bbb \bbb \bbb \bbb \bbb }
\newcommand{\abc}{abcdefg1234567890}
\newcommand{\upabc}{ABCDEFGHIJK}
%%% ----------------------------------------------------------------------
\newenvironment{myfont}{\fontfamily{lmtt}\selectfont}{\par}
\DeclareTextFontCommand{\textmyfont}{\myfont}

\newfontfamily\courier{Courier New}

%============================= 版芯控制 ================================%
\setlength{\oddsidemargin}{0.57cm}
\setlength{\evensidemargin}{\oddsidemargin}
\voffset-6mm \textwidth=150mm \textheight=230mm \headwidth=150mm
%\rightmargin=35mm
%                                                                       %


%============================= 页面设置 ================================%
%-------------------- 定义页眉和页脚 使用fancyhdr 宏包 -----------------%
% 定义页眉与正文间双隔线
\newcommand{\makeheadrule}{%
\makebox[0pt][l]{\rule[.7\baselineskip]{\headwidth}{0.4pt}}%
\rule[0.85\baselineskip]{\headwidth}{0.4pt} \vskip-.8\baselineskip}
\makeatletter
\renewcommand{\headrule}{%
{\if@fancyplain\let\headrulewidth\plainheadrulewidth\fi
\makeheadrule}} \makeatother

\newcommand{\upcite}[1]{\textsuperscript{\textsuperscript{\cite{#1}}}}

\newcommand{\adots}{\mathinner{\mkern 2mu%
\raisebox{0.1em}{.}\mkern 2mu\raisebox{0.4em}{.}%
\mkern2mu\raisebox{0.7em}{.}\mkern 1mu}}

%\setmainfont{Times New Roman}
\dottedcontents{chapter}[1.5cm]{\xiaosi\heiti}{3.8em}{9.5pt}
\dottedcontents{section}[1.5cm]{\xiaosi\heiti}{2.8em}{9.5pt}




%=============================== 代码展示设置 ================================%

\definecolor{codegreen}{rgb}{0,0.6,0}
\definecolor{codegray}{rgb}{0.5,0.5,0.5}
\definecolor{codepurple}{rgb}{0.58,0,0.82}
\definecolor{backcolour}{rgb}{0.95,0.95,0.92}

%Code listing style named "mystyle"
\lstdefinestyle{mystyle}{
  backgroundcolor=\color{backcolour},
  commentstyle=\color{codegreen},
  keywordstyle=\color{magenta},
  numberstyle=\tiny\color{codegray},
  stringstyle=\color{codepurple},
  basicstyle=\small\courier,
  breakatwhitespace=false,
  breaklines=true,
  captionpos=b,
  keepspaces=true,
  numbers=left,
  numbersep=5pt,
  showspaces=false,
  showstringspaces=false,
  showtabs=false,
  tabsize=2
}

\lstset{style=mystyle}

%=============================== 正文部分 ================================%


\begin{document}

 \pagestyle{empty}
\setlength{\baselineskip}{25pt}  %%正文设为25磅行间距
\vspace{-2.0cm}
\noindent{{\zihao{4} {\large 2022} 届硕士专业学位研究生学位论文}}\\
\vspace{-0.8cm}
\begin{flushleft}
\hspace{-0.5cm}
\renewcommand\arraystretch{1.5}
\begin{tabular}{l}
\noindent{{\zihao{4} 分类号:\underline{\qquad\qquad\qquad\qquad\qquad\qquad}}}  \\
\noindent{{\zihao{4} 密~~~~级:\underline{\qquad\qquad\qquad\qquad\qquad\qquad}}}\\
\end{tabular}
\hskip 1.1cm
\renewcommand\arraystretch{1.5}
\begin{tabular}{l}
\noindent{{\zihao{4} 学校代码:\underline{10269~~~\qquad}}}\\
\noindent{{\zihao{4} 学~~~~~~~~号:\underline{51194501126}}}\\
\end{tabular}
\end{flushleft}


\vskip 1.0cm

\begin{center}
	\hskip 0.5cm
	\scalebox{1.0}{\includegraphics[width=2.7cm]{fig/ecnulogo.png}}
	\scalebox{1.0}{\includegraphics[width=11.5cm,height=2.7cm]{fig/ecnulabel.png}}
	\vskip 0.5cm
	{\textbf{{\xiaoer East China Normal University}}}\\ \vskip 0.2cm
	{\textbf{\erhao 硕~士~专~业~学~位~论~文}}\\ \vskip 0.2cm
	{\textbf{{\xiaoer MASTER'S DISSERTATION (Professional)}}}\\\end{center}





\vskip 1.0cm

\begin{center}
{\zihao{2}\bf 论文题目:基于差分隐私和安全混洗的联邦学习隐私保护研究}
\end{center}

\vskip 1.0cm
\begin{center}

\renewcommand\arraystretch{1.5}
	\begin{tabular}{l}
{\sihao \bf 院\qquad\ \ 系:}\\
{\sihao \bf 专业学位类别:}\\
{\sihao \bf 专业学位领域:}\\
{\sihao \bf 论文指导教师:}\\
{\sihao \bf 论~文~作~者:}
\end{tabular}
\begin{tabular}c
{\sihao \bf  ~~软件工程学院}               \\
\hline {\sihao \bf ~~工程硕士 }              \\
\hline {\sihao \bf ~~软件工程~~}\\
% \hline \bf ~~曹珍富\  教授  \\
\hline ~~\    \\%盲审版本
% \hline \bf ~~何慧娴\   \\
\hline ~~\   \\%盲审版本
\hline
\end{tabular}


\end{center}

\vskip 2.0cm
\begin{center}
{\sihao 2021年09月10日}
\end{center}

 %\clearpage\ \newpage
 \newpage

\pagestyle{empty}

\noindent{Thesis (Professional) for Master’s Degree in 2022}
\hskip 1.4cm {School Code: \underline{10269~~~~~~~~~~~\qquad}}\\
\hspace*{\fill} {Student Number:\underline{51194501126~~~~~~~~}}
\vskip 2cm

\begin{center}
{\Huge \bf EAST\,CHINA\,NORMAL\,UNIVERSITY}
\end{center}

\vskip 3cm

\begin{center}
{\huge \bf \scshape Title: Technologies research for privacy preserving based on federated learning}
\end{center}

\vskip 2cm {\large
\begin{center}
\begin{tabular}{l}
Department:\\
Major:\\
Research Direction:\\
Supervisor:\\
Candidate:
\end{tabular}
\begin{tabular}c
~~~Software Engineering Institute\\
\hline ~~~Software Engineering  \\
\hline ~~~Cryptography and Network Security\\
% \hline ~~~Professor ~ZhenFu Cao~  \\
\hline ~~~  \\%盲审版本
% \hline ~~~HuiXian He  \\
\hline ~~~  \\%盲审版本
\hline
\end{tabular}
\end{center}}

\vskip 30mm

\begin{center}
{\Large Nov 9, 2021}
\end{center}

 %\clearpage\ \newpage
 % \input{parts/A/A3-COPYRIGHT.tex}
 %\clearpage\ \newpage
% \newpage
\pagestyle{empty}
$$\\ \\ \\ $$

\centerline{\bf\Large $\underline{\mbox{\kaishu {XXX}}}\,\,
	$硕士学位论文答辩委员会成员名单}

\vskip 10mm

\begin{center}
	{\large
		\begin{tabular}{| p{25mm}| p{30mm}| p{64mm}| p{23mm}|}\hline
			\vfill\hfill{\heiti 姓名}\hspace*{\fill} &\vfill\hfill{\heiti 职称}\hspace*{\fill} &
			\vfill\hfill{\heiti 单位}\hspace*{\fill} &\vfill\hfill {\heiti 备注~~~~~} \hspace*{\fill} \\[6pt]\hline
			\vfill\hfill{\kaishu }\hspace*{\fill} &\vfill\hfill{\kaishu }\hspace*{\fill} &\vfill\hfill{\kaishu }\hspace*{\fill} & \vfill\hfill {\kaishu ~~~~~~}\hspace*{\fill} \\[6pt]\hline
			\vfill\hfill{\kaishu }\hspace*{\fill} &\vfill\hfill{\kaishu }\hspace*{\fill} &\vfill\hfill{\kaishu \tabincell{c}{} }\hspace*{\fill} & \vfill{\heiti }\\[20pt]\hline
			\vfill\hfill{\kaishu }\hspace*{\fill} &\vfill\hfill{\kaishu }\hspace*{\fill} &\vfill\hfill{\kaishu }\hspace*{\fill} & \vfill{\heiti }\\[20pt]\hline
			%\vfill\hfill{\kaishu ~~~}\hspace*{\fill} &\vfill\hfill{\kaishu ~~~}\hspace*{\fill} &\vfill\hfill{\kaishu ~~~}\hspace*{\fill} & \vfill{\heiti }\\[20pt]\hline
%			\vfill\hfill{\kaishu ~~~}\hspace*{\fill} &\vfill\hfill{\kaishu ~~~}\hspace*{\fill} &\vfill\hfill{\kaishu ~~~}\hspace*{\fill} & \vfill{\heiti }\\[20pt]\hline
%			\vfill\hfill{\kaishu ~~~}\hspace*{\fill} &\vfill\hfill{\kaishu ~~~}\hspace*{\fill} &\vfill\hfill{\kaishu ~~~}\hspace*{\fill} & \vfill{\heiti }\\[20pt]\hline
			%              &             &              &  \vfill{\heiti }\\[20pt]\hline
		\end{tabular}
	}
\end{center}


\clearpage\ \newpage

\newpage
\pagenumbering{roman}
\pagestyle{plain}
\vspace{-2.5cm}
\chapter*{\zihao{2}\heiti{摘~~~~要}}
%\vskip 1cm
%\vspace{-1cm}

近年来,机器学习由于其卓越的性能已被越来越多的应用于各个领域,如自动驾驶,人脸识别,个人信用评估等。但机器学习模型一般都为黑盒,由于其不可见性与不可解释性,使得人们对它的安全性有了很大的担忧。因此,越来越多的研究人员投身到机器学习模型的安全性验证的方法和工具的研究中。

与神经网络模型一样,树模型也容易受到对抗性样本的攻击。树模型的“脆弱性”在使其在某些安全性要求较高的应用中造成了隐患。有些情况下,甚至可能导致灾难性的后果。为此,我们研究了树模型的鲁棒性验证问题。
本文主要的工作和贡献如下:
\begin{enumerate}
	\item 我们提出了一个基于SMT技术的树模型鲁棒性验证框架,它可以有效的验证树模型的两个重要组成部分:随机森林和 GBDT 模型的鲁棒性,并且支持规模较大的模型的验证。该框架的核心思想是将树模型的鲁棒性验证问题转化为SMT公式的约束求解问题。
	\item 在验证的基础上,我们进一步对树模型鲁棒性的可解释性问题进行了研究,提出了鲁棒特征集合和局部鲁棒特征重要度的概念来描述模型鲁棒性与样本特征的内在联系,从而为对抗性样本攻击提供了新的思路。
	\item 我们基于三个基准测试集评估了框架的可行性和有效性,并且在实验中讨论了模型训练超参数与其鲁棒性的关系,从而为训练阶段提高模型的鲁棒性提供了重要参考。
\end{enumerate}
\hspace{-0.5cm}
\sihao{\heiti{关键词:}} \xiaosi{鲁棒性验证,可解释性,随机森林,GBDT,SMT}

 %\clearpage\ \newpage
\newpage
\vspace{-1cm}
\chapter*{\zihao{-2}\heiti{ABSTRACT}}
%\vspace{-0.5cm}

With the rapid development of artificial intelligence and the proliferation of mobile devices, application scenarios that require the collaboration of multiple participants are emerging and the role of distributed data processing and distributed machine learning is becoming increasingly prominent. For example, financial data scattered across multiple banks, medical records in different hospitals, behavioural records of each user under a large platform, as well as data generated by smart meters, sensors or mobile devices all need to be processed and mined in a distributed manner. 

Data silos are one of the key challenges facing distributed data processing and distributed machine learning. As a solution to address data silos, Federated Learning is a promising distributed computing framework that can train models locally on multiple decentralised edge devices without transferring their data to servers. With the increasing awareness of privacy among citizens and the improvement of related laws, privacy security in federation learning is also a growing concern, and recent research work has shown that it has been possible to restore users' private data by attacking the gradient parameters of the model, i.e. it is not enough to protect privacy by keeping the data local, and privacy-preserving techniques can protect privacy at the expense of model accuracy. 

To this end, this paper uses differential privacy techniques to protect user privacy in federation learning, and for distributed scenarios, analyses the adaptive interference mechanism against the gradient descent algorithm during model training to achieve the goal of improving model accuracy, and proposes a secure shuffle framework to prevent attacks by malicious servers. 

The main work of this paper includes the following aspects:
\begin{enumerate}
	\item In a federal learning differential privacy scenario, this paper presents a novel, local differential privacy-based adaptive interference algorithm for weight assignment. In a client-side locally trained neural network model, the contribution ratio of each attribute class to the model output is calculated by analysing the forward propagation algorithm, and then we develop an adaptive noise addition scheme that injects noise with different privacy budgets according to the contribution ratio. Compared with the traditional method of injecting noise, we maximise the accuracy of the model with the same degree of privacy protection, reduce the impact of noise on the model output results and improve the model accuracy.
	\item Considering the attacks on parameter aggregators in federation learning, this paper proposes a new secure aggregation mechanism by adding a new mashup between the local client and the central server, where parameters are mashup before users upload them to the cloud server, and updates to model parameters are sent anonymously to the mashup, achieving client anonymity through splitting and mashup of model parameters, and demonstrating the secure The feasibility of mashup models is demonstrated.
	\item In this paper, we experimentally demonstrate the combination of an adaptive local differential privacy scheme and a secure mashup framework that allows a federally learned model to balance accuracy and privacy budgets.
\end{enumerate}
%\hspace{-0.5cm}
{\sihao{\textbf{Keywords:}}} \textit{Federated learning, Privacy preserving, Local differential privacy , Security aggregation}



\clearpage\ \newpage
\setcounter{tocdepth}{2}

\tableofcontents
\listoffigures
\listofalgorithms

\newpage

\pagenumbering{arabic}
\pagestyle{fancy}

\CTEXsetup[format+={\zihao{3}\heiti}]{chapter}
\CTEXsetup[format+={\raggedright\zihao{4}\heiti}]{section}
\CTEXsetup[format+={\zihao{-4}\heiti}]{subsection}


\setlength{\baselineskip}{25pt}  %%正文设为25磅行间距


\chapter{绪\hskip 0.4cm 论}
\label{ch1}

\section{研究背景及意义}
随着机器学习的不断发展和壮大,我们一方面惊叹于它的成就,比如Alpha GO击败了围棋世界冠军柯洁,或者面部识别技术帮助我们抓住了躲藏多年的逃犯,而大型工业企业也大力推动机器学习技术的应用。另一方面,我们也必须认识到,它的巨大潜力还有待实现,例如:构建基于大量病例的医疗救助诊断系统,运行基于大量商业行为数据的信用风险控制模型,帮助高价值企业融资,并基于整个产业链的数据提供个性化的产品分配和营销策略。我们真正见证了人工智能(AI)的巨大潜力,以及已经开始期待在许多应用中使用更复杂、更尖端的人工智能技术,包括无人驾驶、医疗、金融等今天,人工智能技术几乎在各方面都大显身手每个行业和各行各业。但是传统的机器学习方法依赖于集中管理的训练数据集,建立在大量数据上,从数据中学习特征,从而完成复杂的任务,甚至是人类也难以完成的操作。

然而,这些数据的采集可能涉及到用户的隐私,随着人们的隐私意识的普遍提高,相关的隐私法律法规的不断完善,中国出台的《网络安全与数据合规》白皮书中明确要求加强用户个人信息保护。2018年欧洲联盟出台《通用数据保护条例》中强调保护用户的个人隐私和数据安全用户可以删除或撤回其个人数据。近年来,也有越来越多的涉及数据泄漏和隐私侵权的事情,用户们也越来越关注自己的隐私信息是否在未经个人许可,或者出于商业和政治目的被他人或机构利用。随着个人意识和国家政策的关注,在大数据和人工智能领域数据采集和使用的过程中,保护用户隐私和数据的机密显得越来越重要。

大多数训练数据是由不同组织的个人或部门产生的,一个AI项目可能涉及多个领域,需要融合各个公司、各个部门的数据。(比如研究居民线上消费问题,需要各个消费平台的数据,可能还需要银行数据等等),但在现实中想要将分散在各地、各个机构的数据进行整合几乎是不可能的。传统的机器学习是通过收集数据并将其发送到一个能看到并控制所有数据的中央服务器来完成的。因此,这个中心位置不仅要有强大的计算机集群来训练和创建机器学习模型,还要处理敏感数据并防止数据被用于其他目的。此外,敏感数据的处理方式必须不损害用户的隐私。然而,这用户完全信任服务器的假设已不再适用。在这种情况下,数据拥有者倾向于将数据掌握在自己手中,这就导致了孤立的数据孤岛,数据孤岛使所有利益相关者无法获得更多的数据。例如,每家医院的居民医疗记录的样本量完全不够,导致模型有偏差。在信贷领域,银行只能使用中央银行的信贷报告来建立风险控制模型。

人工智能的力量是基于大数据的,但我们被更多的小数据包围在孤岛中。大数据的基础就没有了,人工智能的基础也没有了。大数据的基础已经消失,人工智能的未来也岌岌可危。要解决大数据的困境,仅仅靠传统的方法已经出现瓶颈。两个公司简单的交换数据在很多法规包括《通用数据保护条例》是不允许的。用户是原始数据的拥有者,在用户没有批准的情况下,公司间不能交换数据。传统的机器学习和深度学习的方法本身已经成为解决大数据困境的绊脚石。简单地在两家公司之间交换数据,无论是《通用数据保护条例》还是GDPR都是不允许的:用户是原始数据的所有者,未经其同意,数据不能在公司之间交换。

那如何创建一个机器学习框架,使人工智能系统能够更有效和准确地集体使用数据,同时满足隐私、安全和监管要求,并解决数据孤岛的问题。如何才能做到这一点呢?

为了解决这个问题,google在2016年率先提出了联邦学习的概念,它提供了一个具有隐私保护功能的分布式机器学习框架,并且能够以分布式方式与成千上万的参与者协作,迭代训练一个特定的机器学习模型。由于训练数据在联合过程中保持在参与者的本地,这种机制允许参与者之间共享训练数据,同时确保每个参与者的隐私[15]。
联合学习的基本工作流程如下:
(1) 初始化:所有用户在他们的设备上都有一个预先分配的神经网络模型,并且可以自愿加入联邦学习协议,指定相同的机器学习和模型训练目标。
(2) 本地训练:在一个给定的通信回合中,联邦参与者首先从中央服务器下载全局模型参数,然后使用他们的私人训练模式训练模型,创建本地模型更新(即模型参数),并将这些更新发送到中央服务器。
(3) 模型平均化。下一轮的全局模型是通过汇总所有通过训练不同的训练模式获得的模型更新并取其平均值来确定的。
(4) 迭代地执行上述步骤以达到优化当前全局模型的目的,整个迭代过程将在全局模型参数满足收敛条件时停止。

联合学习在隐私敏感的场景(包括金融、工业和许多其他与数据相关的场景)中显示出巨大的前景,这是因为它具有独特的优势,能够从多个参与者的本地数据中训练出一个统一的机器学习模型,同时保护数据隐私[16⁃17]。联合学习解决了数据聚合的问题,并允许一些机器学习模型和算法在各机构和部门之间进行设计和训练。在一些移动设备上的机器学习模型应用中,联邦学习显示出良好的性能和稳健性。此外,对于一些没有足够的私人数据来开发准确的本地模型的用户(客户)来说,机器学习模型和算法的性能可以通过联合学习得到显著改善。

\section {问题和挑战}

\subsection{数据异构}
由于联邦学习的重点是通过以分布式方式从所有参与的客户端设备中学习本地数据来获得高质量的全局模型,所以它无法捕捉每个设备的个人信息,导致推理或分类性能下降。此外,传统的联邦学习要求所有参与的设备同意使用一个共同的模型来共同训练,这在复杂的现实世界物联网应用中是不现实的。研究人员对学习在实际应用中面临的问题总结如下[2]。

(1)设备的异质性:由于客户端设备的硬件条件(CPU、内存)、网络连接(3G、4G、5G、WiFi)和电源(电池)的变化,联邦学习网络上每个设备的存储、计算和通信能力都可能不同。由于网络和设备的限制,在任何时候都只有某些设备可以活动。此外,设备可能会受到意外事件的影响,如断电或断网,这可能会导致暂时的断网。这种异质性的系统结构影响了联邦模型的整体学习战略。

(2)统计的异质性:在整个网络中,设备通常以不同的方式产生和收集数据,而且不同设备的数据量、特征等会有很大的不同,所以联合学习网络中的数据不是独立和相同的分布(非IID)。目前,目前的机器学习算法主要是基于对IID数据的假想假设。因此,非IID数据的异质属性给建模、分析和评估带来了重大挑战。[19]提出了FederatedAverageing(FedAvg)方法来解决非均匀同分布数据的问题,但是当数据分布偏态很严重的时候FedAvg的性能退化严重,一方面其性能比中心化的方法差好多,另一方面它只能学习到IoT设备粗粒度的特征而无法学习到细粒度的特征。

(3)模型的异质性:每个客户根据其应用场景要求定制不同模型。

\subsection{高昂的通信代价}
在联邦学习过程中,根据存储在几十甚至几百万个远程客户端设备上的数据来学习一个全局模型。在训练期间,客户设备必须定期与中央服务器进行通信原始数据被储存在本地的远程客户端设备上,这些设备必须不断地与中央服务器互动,以完成全局模型的构建。通常情况下,整个联盟学习网络可能涉及大量的设备,而网络通信可能比本地计算慢几个数量级,因此高通信成本成为联邦学习的关键瓶颈。

\subsection{安全性和隐私威胁}

(1) 由于联合学习系统的云端服务器无法访问参与者的本地数据和他们的训练过程,恶意参与者可以发送无效的模型更新来达到并破坏全局模型。例如,内部攻击者可以通过在修改后的训练数据上引起有毒的模型更新来有效地损害全局模型的准确性。内部攻击可以由联邦学习服务器发起,也可以由联邦学习参与方发起。外部攻击(包括偷听者)通过参与方与服务器之间的通信通道发起。外部攻击的发起者大部分为恶意的参与方,例如敌对的客户、敌对的分析者、破坏学习模型的敌对设备或者其组合。在联邦学习中,恶意设备可以通过白盒或者黑盒的方式访问最终模型,因此在防范来自系统外部的攻击时,需要考虑模型迭代过程中的参数是否存在泄露原始数据的风险,这对严格的隐私保护提出了新的挑战。

(2) 由于局部模型更新和全局模型参数的结合提供了关于训练数据的隐藏知识,用户的个人信息有可能泄露给不受信任的服务器或其他恶意用户。例如,即使是由其他用户的训练数据生成的样本原型也会被恶意用户隐蔽地窃取。在训练过程中,攻击方可以试图学习、 影响或者破坏联邦学习模型。在联邦训练的过程中,攻击方可以通过数据中毒攻击的方式改变训练数据集合收集的完整性,或者通过模型中毒攻击改变学习过程的完 整性。攻击方可以攻击一个参与方的参数更新过程,也可以攻击所有参与方的参数更新过程。
若联邦学习的参与方想利用各方的数据集合训练一个模型,但是又不想让自己的数据集泄露给服务器,就需要约定联邦建模的模型算法(例如神经网络)和参数更新的机制(例如随机梯度下降(stochastic gradient descent,SGD))。那么在训练前,攻击方就可以获取联邦学习参数更新 的机制,从而指定对应的推断攻击策略。

(3) 在不信任的云服务器和恶意参与者的勾结下,任何个人的确切私人信息都会被泄露。



\subsection{针对联邦学习的隐私保护}
在联邦学习中,存在着无数与隐私有关的挑战学习中的隐私问题。除了保证隐私之外,重要的是要保证确保通信成本的低廉和高效。有许多关于联合学习的隐私定义[8][2][19]。我们可以把它们分为两类:局部隐私和全局隐私。在本地隐私中,每个客户端发送一个不同的隐私值,该值是安全的加密的到服务器。在全局模型中,服务器在最终输出中添加不同的隐私噪音。安全多方计算、同态加密和差分隐私是最常见的技术来保证联盟环境中的安全和隐私。

安全多方计算模型涉及多方,并在一个定义明确的模拟框架中提供安全证明,以保证完全的零知识,即每一方除了其输入和输出外一无所知。零知识是非常理想的,但这种理想的属性通常需要复杂的计算协议,而且可能无法有效实现。在某些情况下,如果提供安全保证,部分知识的披露可能被认为是可以接受的。有可能在较低的安全要求下建立一个具有SMC的安全模型,以换取效率[16]。最近,一项研究[46]将SMC框架用于训练具有两个服务器和半诚实假设的机器学习模型。在[33]中,MPC协议被用于模型训练和验证,而用户不会泄露敏感数据。最先进的SMC框架之一是Sharemind[8]。[44]的作者提出了一个具有诚实多数的3PC模型[5,21,45],并考虑了半诚实和恶意假设的安全性。这些作品要求参与者的数据在非共存的服务器之间秘密共享。

同态加密是一种加密形式,它允许人们对密文进行特定形式的代数运算得到仍然是加密的结果,将其解密所得到的结果与对明文进行同样的运算结果一样同态加密[53],明文通过同态加密方法得到密文后,可实现密文间的计算(密文计算后解密的结果等价于明文计算的结果)。如果对密文进行加法(或乘法)运算后解密,与明文进行加法(或乘法)运算,结果相等,则称这种加密算法为加法(乘法)同态。如果同时满足加法和乘法同态,则称为全同态加密。在联邦学习中,因为只需要对中间结果或模型进行聚合,一般使用的同态加密算法为PHE(多见为加法同态加密算法),通过加密机制下的参数交换来保护用户数据隐私[24, 26, 48],例如在FATE中使用的Paillier即为加法同态加密算法。

差分隐私方法涉及向数据添加噪音,或使用概括方法来掩盖某些敏感属性,直到第三方无法区分个人,从而使数据无法被还原以保护用户的隐私。利用差分隐私,可以在本地模型训练及全局模型整个过程中对相关参数进行扰动,从而令敌手无法获取真是模型参数,但是与密码学技术相比,差分隐私无法保证参数传递过程中的机密性,从而 增加了模型遭受隐私攻击的可能性.例如刘俊旭等 人[10]针对联邦学习下差分隐私中存在的攻击方法 进行了详细的调研。在[23]中,作者为联合学习引入了一种差异化的隐私方法,以便通过在训练期间隐藏客户端的贡献来增加对客户端数据的保护。在深度学习中,差分隐私可以作为一种局部隐私保护方案来保护用户梯度的隐私,Abadi等人[43]提出了一种隐私保护的深度学习方法,主要通过使用噪声来扰乱少量步骤后的局部梯度,将差分隐私机制与SGD算法相结合。令人担忧的是,隐私保护预算的成本和联合学习的有效性之间的权衡是困难的,因为较高的隐私保护预算可能对一些大规模的攻击(如基于GAN的攻击)不是很有用[50],而较低的隐私保护预算可能阻碍模型的局部收敛。


总的来说,安全多方计算基于复杂的计算协议,同态加密的运算成本非常高,而差分隐私破坏了数据的可用性,很难在模型性能和隐私成本上达到平衡,当前的研究方向主要集中在对数据集和神经网络中的参数的加密和隐私保护机制上,较少关注到模型整体框架等过程。目前的联邦学习中的隐私保护方法还有许多不足,不能在隐私性和模型可用性上都达到一个相对满意的效果,此外, 大部分方法是基于统一的、固定的参数设置,会导致模型迭代过程中累积大量隐私损失,使模型性能大幅下降。因此,在联邦学习场景下,保护用户隐私的同时保持模型准确性仍需大量的研究,


\section{本文工作与主要贡献}
针对联邦学习中隐私性和模型精度的双重指标,本文提出了参数匿名上传框架和自适应差分隐私算法,主要的工作和贡献包含以下三个方面:
\begin{enumerate}
\item [(1)] 本文提出了一个新的参数聚合框架,该框架支持在参数上传过程中,对于每一个本地模型,通过两个重要实现:拆分和混洗,扰乱模型中各个参数的隐私关联和各个模型之间的隐私关联,实现客户端匿名。
\item [(2)] 本文提出了一个自适应扰动方案,对联邦学习过程中双方所交互的梯度进行分析,在所交互的梯度上添加扰动,并基于梯度自适应加噪,进一步减少隐私预算。
\item [(3)] 本文针对模型训练和上传过程中的隐私安全问题,将改进的参数聚合框架和自适应扰动方案引入联邦学习框架,实现混合隐私保护的联邦学习系统,每个用户在本地训练数据时添加自适应扰动,并在向中心服务器上传时实现客户端匿名,实现了客户端的数据隐私。
\item [(3)] 本文通过实验,展示了自适应假造方案和参数聚合框架的结合,使得联邦学习的模型的精度和隐私预算达到平衡。
\end{enumerate}

\section{本文组织结构}

本文一共六章,主要内容的组织安排如下:

第一章对本文研究内容:联邦学习的研究背景和实际意义进行了阐述,介绍了目前联邦学习中的隐私保护的研究现状和发展方向。

第二章详细介绍本文研究内容所涉及的一些理论基础与背景知识,包含了联邦学习的相关概念,差分隐私的基础知识。

第三章描述了本文所提出的参数聚合框架的设计和实现。我们首先对框架的整体进行了介绍,之后给出了各个模块的设计和实现细节。
  
第四章描述了本文所提出的自适应加噪方案,讨论了隐私预算与的关系,并且详细描述了相关概念和算法。
    
第五章为实验部分,基于本文提出的隐私保护框架,我们在三个基准数据集的进行了实验和讨论。

第六章是对本文的一个内容总结和展望,首先对本文的研究内容进行了概括,并对现有的不足进行总结,对未来的研究和改进方向进行了展望。



\chapter{基础知识}
\label{ch2}
我们在本章节中介绍了本文研究所需要的一些基本知识,有助于更好的理解之后章节的内容。

\section{联邦学习}

\subsection{基本介绍}
深度学习的成功应用需要建立在大量数据的基础之上,才能完成人们指派的学习任务。然而,近年来数据泄漏和隐私侵权事件不断发生,用户开始更加关注他们的隐私信息是否未经自己的许可,或被他人出于商业或者政治目的而被利用。人们逐渐地意识到,在人工智能的构建与使用的过程中保护用户隐私和数据机密的重要性。

大部分拥有的训练数据是由不同组织的个人、部门产生并拥有的,传统机器学习的做法是收集数据并传输到一个中心服务器,服务器可以看见并控制所有的数据,因此这个中心点不仅需要拥有高性能的计算集群来训练和建立机器学习模型,而且还需要处理敏感数据,避免泄漏用户隐私。然而,这种方法需要用户对服务器的完全信任,这已经不再有效或适用了。在这样的情况下,数据拥有者倾向于将自己的数据保留在自己的手中,进而会形成各自孤立的数据孤岛,至此大量数据的基础已经消失,人工智能的未来将面临绝境。作为回应,2016 年谷歌\cite{ref25}率先提出联邦学习概念,旨在建立高质量分布式学习的框架。在联邦学习系统中,数据所有者(参与者)不需要彼此共享原始数据,也不需要依赖单个可信实体(中心服务器)来进行机器学习模型的分布式训练。相反,参与者通过在自己的本地数据上执行本地训练算法,并且只与参数服务器共享模型参数,来共同协作训练联邦模型。在每轮训练中,参数聚合节点会随机选择合适的节点加入到训练池中。那些被选中的本地节点通常是保持充电且无线网络可用。然后参数聚合节点平均所有已提交者的权重并作为下一轮回合的初始化模型。重复此过程直至终止条件。

根据用户维度和模型特征维度的重合去分类,将联合学习分为水平联邦学习、纵向联邦学习和联合迁移学习\cite{ref26}。
\begin{itemize}
\item \textbf{水平联邦学习}:当两个数据集的用户属性重叠较多而用户重叠较少的情况下,我们对数据集进行横向切割(即按用户维度切割),删除两边用户属性相同但用户不完全相同的那部分数据,用于训练。这种方法被称为横向联合学习。例如,两家银行位于不同的地区,有来自各自地区的用户群,而且它们之间的联系非常少。然而,他们的业务活动非常相似,因此他们的用户特征也是一样的。在这个阶段,我们可以使用跨部门的联合学习来建立一个联合模型。2016年,谷歌提出了一个在安卓手机上更新模型的联合数据建模系统:模型参数在本地不断更新,并在各个用户使用安卓手机时上传到安卓云端,使拥有数据的每一方都能建立一个具有相同特征维度的联合模型。

\item \textbf{纵向联邦学习}:在两个数据集与用户重叠较多而与用户属性重叠较少的情况下,我们将数据集纵向切开(即按特征维度),选择数据集中两边用户相同但用户属性不完全相同的部分进行训练。这种方法被称为纵向的联合学习。例如,有两个不同的组织,一个是在一个地方的银行,另一个是在同一个地方的电子商务公司。他们的用户群很可能包括该地的大部分人口,所以有很大的用户交集。然而,由于银行储存的是用户的收入和支出以及信用评分的数据,而电子商务公司储存的是用户的浏览和购买历史的数据,他们的用户档案并没有那么紧密的联系。长期的联邦学习是在一个加密的空间里将这些不同的功能结合起来,以提高模型的性能。渐渐地,人们发现可以在这个联合系统之上建立若干机器学习模型,如逻辑回归、树状结构和神经网络模型。

\item \textbf{联合迁移学习}:联合迁移学习是通过使用迁移学习模型来弥补数据或标签的差距,而不是对数据进行切分,两个数据集中的用户和用户属性几乎没有重叠。这种方法被称为混合式学习迁移。这里举一个例子,考虑两个不同的组织,一个是中国的银行,另一个是美国的电子商务公司。由于地理上的限制,这两个机构的用户群重叠的地方很少。由于它们是不同类型的组织,数据的特点也没有太多的重叠。在这种情况下,为了保证有效的联邦学习,有必要引入反式学习,以克服单变量数据量小和标注样本小的问题,提高模型的效率。

\end{itemize}

\subsection{模型框架}
在很多横向联邦学习应用场景中,参与训练的参与方数据具有类似的数据结构 (特征空间),但是每个参与方拥有的用户是不相同的。有时参与方比较少,例如, 银行系统在不同地区的两个分行需要实现联邦学习的联合模型训练;有时参与方会非常多,例如,做一个基于手机模型的智能系统,每一个手机的拥有者将会是一个独立的参与方。针对这类联合建模需求,可以 通过一种基于服务器客户端的架构来满足很多横向联邦学习的需求。将每一个参与方看作一个客户端,然后引入一个大家都信任的服务器来帮助完成联邦学习的联合建模需求。在联合训练的过程中,被训练的数据将会被保存在每一个客户端本地,同时,所有的客户端可以一起参与训练一个共享的全局模型,最终所有的客户端可以一起享用联合训练完成的全局模型。

\begin{itemize}
\item 步骤1:中心服务器初始化联合训练模型,并且将初始参数传递给每一个客户端。
\item 步骤2:客户端用本地数据和收到的初始化模型参数进行模型训练。具体步骤包括:计算训练梯度,使用加密、差分隐私等加密技术掩饰所选梯度,并将加密后的结果发送到服务器。
\item 步骤3:服务器执行安全聚合。服务器只收到加密的模型参数,不会了解任何客户端的数据信息,实现隐私保护。服务器将安全 聚合后的结果发送给客户端。
\item 步骤4:参与方用解密的梯度信息更新各自的本地模型,具体方法重复步骤2。
\end{itemize}

\subsection{安全和隐私威胁}
尽管联邦学习提供了隐私保护的机制,还是有各种类型的攻击方式可以攻击联邦学习系统,从而破坏联邦学习系统安全和参与方的隐私。本节将讨论关于联邦学习的攻击问题。从参与方的类型来看,可以将联邦学习的威胁模型细分为半诚实模型 (semi-honest model)和恶意模型。从攻击方向角度来看,可以将联邦学习的攻击分为从内部发起和从外部发起两个方面。从攻击者的角色角度来看,可以将攻击分为参与方发起的攻击、中心服务器发起的攻击和第三方发起的攻击。从发动攻击的方式角度来看,可以将攻击分为中毒攻击和拜占庭攻击。从攻击发起的阶段角度,可以将攻击分为模型训练过程的攻击和模型推断过程的攻击。

\begin{itemize}
\begin{figure}[!hbt]
\centering
	\includegraphics[scale=0.6]{fig2/C2/联邦学习模型攻击}%联邦学习的系统架构
	\caption{联邦学习隐私攻击}
	\label{fig:联邦学习隐私攻击}	
\end{figure}

\item \textbf{半诚实但好奇的攻击方}:半诚实但好奇的攻击方假设也被称为 被动攻击方假设。被动攻击方会在遵守联邦学习的密码安全协议的基础上,试图从协议执行过程中产生的中间结果推断或者提取出其他参与方的隐私数据。半诚实但好奇的供给方通常是客户端的角色,它们可以检测从服务器 接收的所有消息,但是不能私自修改训练的 过程。在一些情况下,安全包围或者可信执行环境(trusted execution environment, TEE)等安全计算技术的引入,可以在一定程度上限制此类攻击者的影响或者信息的可见性。半诚实但好奇的参与方将很难从服务器传输回来的参数中推断出其他参与方的隐私信息,从而威胁程度被削弱。

\item \textbf{恶意攻击方}:恶意攻击方也被称为主动攻击方。由于恶意攻击方不会遵守任何协议,为了达到获取隐私数据的目的,可以采取任何攻击手段,例如破坏协议的公平性、阻止协议 的正常执行、拒绝参与协议、不按照协议恶意替换自己的输入、提前终止协议等方式,这些都会严重影响整个联邦学习协议的设计以及训练的完成情况。
恶意的参与方可以是客户端,也可以是服务器,还可以是恶意的分析师或者恶意的模型工程师。恶意客户端可以获取联邦建模过程中所有参与方通信传输的模型 参数,并且进行任意修改攻击。恶意服务器可以检测每次从客户端发送过来的更新模型参数,不按照协议,随意修改训练过程,从而发动攻击。恶意的分析师或者恶意的模型工程师可以访问联邦学习系统的输入和输出,并且进行各种恶意攻击。

\item \textbf{成员推理攻击}:如上文所述,联邦学习机制要求所有参与者通过在本地数据集上训练全局模型来更新梯度。在这种情况下,如果联邦学习系统有一个不被信任的服务器,其知识不能被信任,那么用户的私人信息就不能得到保证。这个不受信任的服务器可以获得关于每个参与者的本地训练模型的大量额外信息(例如,模型结构、用户身份和梯度),并且能够充分损害用户的隐私信息。具体实现如下:攻击者首先在平均化后获得模型的全局参数,并在本地存储这些快照。然后,通过计算以下快照与进一步删除添加的更新,以获得其他用户的模型的汇总更新。通过这种方式,攻击者可以利用数据集的协助,得出所有其他参与者共同合作的数据样本。

\item \textbf{GAN攻击}:Hitaj等人\cite{ref27}发现,一个联邦学习框架非常容易受到系统内参与者发起的主动攻击。他们首先提出了一个由系统内的恶意用户发起的基于GAN的重建攻击。在训练阶段,攻击者可以冒充无害的用户,训练GAN来模拟由其他用户的训练数据产生的原型样本。通过不断添加假的训练样本,攻击可以逐渐影响整个学习过程,使受害者暴露出更多关于攻击者的目标类的敏感信息。除了客户端发起的GAN攻击,服务器也能通过GAN攻击。恶意服务器最初假装是一个为用户提供联邦学习服务的正常服务器,但其主要目标是重建被攻击用户的训练样本。

\end{itemize}


\section{差分隐私}
差异化隐私作为一种隐私保护方法是为一个用户服务的,因为根据隐私的定义,隐私泄露只是与特定用户有关的信息泄露,而一组用户的统计特征不包括在隐私信息中。如果一个对象在数据库中的存在或不存在,或其价值的变化不会对搜索结果产生重大影响,那么该对象的隐私信息就会受到保护,这就是差异性隐私(DP)概念的起源。差异隐私首先被应用于数据查询,为了更好地说明数据集之间的差异,定义了相邻数据集的概念:两个数据集只差一个信息或只差一个数值不同的记录\cite{ref28}。因此,查询数据库相关信息的攻击者将无法以任何概率确定$X_{n}$是否存在于数据集中,而成员$X_{n}$被认为是相对安全的。


\subsection{基本定义}
对于一个有限域 $Z, z \in Z$ 为 $Z$ 中的元素, 从 $Z$ 中抽样所得 $z$ 的集合组成数据集 $D$, 其样本量为 $n$, 属性的个数为维度 $d$。对数据集 $D$ 的各种映射函数被定义为查询 (Query), 用 $F=\left\{f_{1}, f_{2}, \cdots\right\}$ 来表示一组查询,算法 $M$ 对查询 $F$ 的结果进行处理,使之满足隐私保护的 条件,此过程称为隐私保护机制。设数据集 $D$ 和 $D^{\prime}$,具有相同的属性结构,两者 的对称差记作 $D \Delta D^{\prime},\left|D \Delta D^{\prime}\right|$ 表示 $D \Delta D^{\prime}$ 中记录的 数量。若 $\left|D \Delta D^{\prime}\right|=1$, 则称 $D$ 和 $D^{\prime}$ 为邻近数据集 (Adjacent Dataset)。
\begin{figure}[!hbt]
\centering
	\includegraphics[scale=0.7]{fig2/C2/相邻数据集示意图}%联邦学习的系统架构
	\caption{差分隐私的相邻数据集示意图}
	\label{fig:相邻数据集示意图}	
\end{figure}


\begin{define}[成立条件]\label{成立条件}

若随机算法 $M: D \rightarrow R$ 满足 $(\varepsilon, \delta)-D P$, 当且仅当相邻数据集 $d, d^{\prime}$ 对于算法 $M$ 的所有可能输出子集 $S \in R$ 满足不等式 $^{[40]}$ :
$$
\operatorname{Pr}[M(d) \in S] \leq e^{\varepsilon} \operatorname{Pr}\left[M\left(d^{\prime}\right) \in S\right]+\delta
$$

其中,$\varepsilon$ 表示隐私预算参数, $\varepsilon$ 越小意味着隐私预算越低, 信息泄露越少,隐私保护的强度越高。添加项 $\delta$代表允许以概率 $\delta$ 打破 $\varepsilon-\mathrm{DP}$ 的可能性, 其值通常选择为小于 $1 /|D|$. 当 $\delta=0$ 时, 定义转化为 $\varepsilon-\mathrm{DP}$, 这时机制提供了更加严格的隐私保护。隐私预算参数决定着隐私保护强度, 针对传统数据库保护,当 $\varepsilon \in(0,1)$ 时认为隐私保护强度是有效的, 但 应用在深度学习领域, $\varepsilon \in(0,10)$ 都认为是可以被接受的合理范围。
如图 1 所示,算法 $M$ 通过对输出结果的随机化 来提供隐私保护,同时通过参数 $\varepsilon$ 来保证在数据集中删除任一记录时,算法输出同一结果的概率不发生显著变化。
\end{define}

\subsection{相关概念}
差分隐私保护可以通过在查询函数的返回值中加人适量的干扰噪声来实现。加入噪声过多会影响结果的可用性,过少则无法提供足够的安全保障. 敏感度是决定加人噪声量大小的关键参数, 它指删除 数据集中任一记录对查询结果造成的最大改变. 在差分隐私保护方法中定义了两种敏感度, 即全局敏感度(Global Sensitivity)和局部敏感度(Local Sensitivity)。

\begin{define}[全局敏感度]\label{全局敏感度}
设有函数 $f: D \rightarrow R^{d}$, 输人为一数据集,输出为一$d$ 维实数向量. 对于任意的邻近数据集 $D$ 和 $D^{\prime}$,
$$
G S_{f}=\max _{D, D^{\prime}}\left\|f(D)-f\left(D^{\prime}\right)\right\|_{1}
$$
称为函数 $f$ 的全局敏感度。
\end{define}

函数的全局敏感度由函数本身决定,不同的函数会有不同的全局敏感度.一些函数具有较小的全局敏感度(例如计数函数,其全局敏感度为1),因此只需加入少量噪声即可掩盖因一个记录被删除对查询结果所产生的影响,实现差分隐私保护。

\begin{define}[局部敏感度]\label{局部敏感度}
对于一个查询函数 $f_{:} D \rightarrow R^{d}$, 其中 $D$ 为一个数据集, $R^{d}$为d维实数向量,是查询的返回结果。对于给定的数据集D和它的任意邻近数据集 $D^{\prime}$, 有 $f_{\text {在 }} D$ 上的局部敏感度为:
$L S_{f}(D)=\max _{D^{\prime}}\left\|f(D)-f\left(D^{\prime}\right)\right\|_{1}$
\end{define}

局部敏感度由函数及给定数据集犇中的具体数据共同决定.由于利用了数据集的数据分布特征, 局部敏感度通常要比全局敏感度小得多。
敏感度代表了查询函数针对相邻数据集的输出的最大不同,或者说量化评估了最坏情况下单个样本对整体数据带来的不确定性大小。敏感度函数仅与查询函数的类型有关, 为扰动的添加提供了依据。但是,由于局部敏感度在一定程度上体现了数据集的数据分布特征,如果直接应用局部敏感度来计算噪声量则会泄露数据集中的敏感信息。

全局差分隐私技术旨在实现这样一个目标:如果替换数据集中的任意样本的效果足够小,则查询结果不能被用来探索数据集中任何样本的更多信息\cite{ref29}。作为一种优势,这种技术比局部差分隐私技术更准确,因为它不需要向数据集添加大量的噪声。局部差分隐私技术被引入以去除全局差分隐私中所要求的受信任的中央机构\cite{ref30}。与全局差分隐私技术相比,局部差分隐私技术不需要可信的第三方\cite{ref31}。其缺点是,噪声总量比全局差分隐私技术大得多。

可量化性、可组合性和后处理不变性[]是差分隐私最重要的三个性质。可量化性指的是差分隐私算法在计算特定随机化过程时,可以透明化、精准量化所施加的扰动,即上文提及的隐私预算。这样使用者就可以清楚地知道算法的隐私保护力度;组合性可以将相互独立的差分隐私算法进行组合;差分隐私的后处理不变性,确保了即使对算法的结果进行进一步处理,只要不引入额外信息,后处理就并不会削弱算法的隐私
保护力度。 通过组合定理,人们可以利用基础的差分隐私算法设计出复杂的满足差分隐私保证的系统,这也是差分隐私的重要优势之一。 

在差分隐私部署过程中常常不仅仅在一处添加噪声, 也不仅仅针对数据集进隐私预算的分配有序列组合性和并行组合性两种组合特性:

\begin{theorem}[串行组合]\label{串行组合}
给定 $\mathbf{n}$ 个陏机算法 $M_{i}(1 \leq i \leq n)$ 满足 $\varepsilon_{i}-DP$, 那么针对一个数据库 $D$ 而言, 在 $\mathrm{D}$ 上的算法序列组合可以提供 $\varepsilon-\mathrm{DP}$, 其中 $\sum_{i=1}^{n} \varepsilon_{i}=\varepsilon$ 。
\end{theorem}

\begin{theorem}[并行组合]\label{并行组合}
对于数据库 $\mathrm{D}$, 当其被划分成 $\mathrm{n}$ 个不相交的子集 $\left\{\mathrm{D}_{1}, \mathrm{D}_{2}, \ldots, \mathrm{D}_{n}\right\}$, 在每个子集上应用算法 $\mathrm{M}_{i}$, 每个算法提供 $\varepsilon_{i}-\mathrm{DP}$ , 则在序列 $\left\{\mathrm{D}_{1}, \mathrm{D}_{2}, \ldots, \mathrm{D}_{n}\right\}$ 上整体满足 $\left(\max \left\{\varepsilon_{1}, \ldots, \varepsilon_{n}\right\}\right)-\mathrm{DP}$
\end{theorem}


\subsection{实现机制}
在实践中为了使一个算法满足差分隐私保护的要求,对不同的问题有不同的实现方法,这些实现方法称为“机制”.拉普拉斯机制(Lapalace Mechanism)、指数机制(ExponentialMechanism)与高斯机制是三种最基础的差分隐私保护实现机制.其中,Laplace机制和高斯适用于对数值型结果的保护,指数机制则适用于非数值型结果。

在中心化差分隐私中,最为常用的扰动机制是拉普拉斯(Laplace)机制,该机制可以后期处理聚合查询(例如,计数、总和和均值)的结果以使它们差分私有。
Laplace分布是统计学中的概念,是一种连续的概率分布。

\begin{define}[拉普拉斯机制]\label{拉普拉斯机制}
如果随机变量的概率密度函数分布为:

$f(x \mid \mu, b)=\frac{1}{2 b} \exp \left(-\frac{|x-\mu|}{b}\right)=\frac{1}{2 b}\left\{\begin{array}{ll}\exp \left(-\frac{\mu-x}{b}\right) & x<\mu \\ \exp \left(-\frac{x-\mu}{b}\right) & x \geq \mu\end{array}\right.$

其中,D表示数据集,$f(D)$表示的是查询函数,$Y$表示的是Laplace随机噪声,$M(D)$表示的是最后的返回结果。
$M(D)=f(D)+Y$
如果噪声 $Y \sim L\left(0, \frac{\Delta f}{\longrightarrow}\right)$ 满足 $(\epsilon, 0)-$,则表示服从拉普拉斯分布的随机噪声。因此,当隐私预算确定时,敏感度越大,引入的噪声量越大。
\end{define}

对于非数值型的查询结果或数据,通常使用指数机制来随机选择离散的输出结果来满足差分隐私。指数机制整体的思想就是,当接收到一个查询之后,不是确定性的输出一个$R_{i}$结果,而是以一定的概率值返回结果,从而实现差分隐私。而这个概率值则是由打分函数确定,得分高的输出概率高,得分低的输出概率低。

\begin{define}[指数机制]\label{指数机制}
指数机制满足差分隐私, 如果:

$M(D)=\left(\right.$ return $\left.\varphi \propto \exp \left(\frac{\varepsilon q(\mathrm{D}, \varphi)}{2 \Delta q}\right)\right)$

评分函数 $\mathrm{q}(\mathrm{D}, \varphi)$,用于评估输出 $\varphi$ 的质量。$\Delta q$ 代表了输出的敏感度。
\end{define}

$l_{2}$ 敏感度 : 对一个随机函数 $f: \mathbb{N}^{|k|} \rightarrow \mathbb{R}^{k}$, 它的 $\ell_{2}$ 敏感度表示为:
$$
\Delta_{2} f=\max _{x, y e N^{k^{\prime \prime}} \atop \|[x-y \|-1}\|f(x)-f(y)\|_{2}
$$

与拉普拉斯机制类似高斯机制对输入的所有维度施加高斯噪声干扰 $N\left(0, \sigma^{2}\right)$。
\begin{define}[高斯机制]\label{高斯机制}
对于任意 $\varepsilon \in(0,1)$ 与 $c^{2}>2 \ln (1.25 / \delta)$, 参数满足 $\sigma \geq c \Delta_{2} f / \varepsilon$ 的高斯机制为 $(\varepsilon, \delta)$-差分隐私。
\end{define}

\section{联邦学习中的差分隐私}
传统的联邦学习中使用差分隐私的主要流程如下所示:
\begin{itemize}
\item 本地计算:
客户端 $\mathrm{i}$ 根据本地数据库 $\mathcal{D}_{\mathrm{i}}$ 和接受的服务器的全局模型 $\mathrm{w}_{\mathrm{G}}^{\mathrm{t}}$ 作为本地的参数,即 $\mathrm{w}_{\mathrm{i}}^{\mathrm{t}}=\mathrm{w}_{\mathrm{G}}^{\mathrm{t}}$, 进 行梯度下降策略进行本地模型训练得到 $\mathrm{w}_{\mathrm{i}}^{\mathrm{t}+1} \quad(\mathrm{t}$ 表示当前round) 。

\item: 模型扰动:
每个客户端产生一个随机噪音 $\mathrm{n}, \mathrm{n}$ 是符合高斯分布的,使用 $\overline{\mathbf{w}_{\mathrm{i}}}^{\mathrm{t}+1}=\mathrm{w}_{\mathrm{i}}^{\mathrm{t}+1}+\mathrm{n}$ 扰动本地模型 (这里注意w是一个矩阵,那么n就对矩阵的每一个元素产生噪音)。

\item 模型聚合:
服务器使用FedAVG算法聚合从客户端收到的 $\overline{\mathrm{w}}_{\mathrm{i}} \mathrm{t}+1$ 得到新的全局模型参数 $\mathrm{w}_{\mathrm{G}}^{\mathrm{t}+1}$, 也就是扰动过的 模型参数。

\item 模型广播:
服务器将新的模型参数广播给每个客户端。

\item 本地模型更新:
每个客户端接受新的模型参数,重新进行本地计算。
\end{itemize}

上述的差分隐私技术将原始数据集中到一个数据中心, 然后发布满足差分隐私的相关统计信息, 我们称其为中心化差分隐私(centralized differential privacy)技术.因此,中心化差分隐私对于敏感信息的保护始终基于一个前提假设:可信的第三方数据收集者,即保证第三方数据收集者不会窃取或泄露用户的敏感信息.然而,在实际应用中,即使第三方数据收集者宣称不会窃取和泄露用户的敏感信息, 用户的隐私依旧得不到保障。由此可知,在实际应用中想要找到一个真正可信的第三方数据收集平台十分困难,这极大地限制了中心化差分隐私技术的应用.鉴于此, 在不可信第三方数据收集者的场景下, 本地化差分隐私(local differential privacy)\cite{ref32}\cite{ref33}技术应运而生, 其在继承中心化差分隐私技术定量化定义隐私攻击的基础上, 细化了对个人敏感信息的保护.具体来说, 其将数据的隐私化处理过程转移到每个用户上, 使得用户能够单独地处理和保护个人敏感信息,即进行更加彻底的隐私保护.目前,本地化差分技术在工业界已经得到运用:苹果公司将该技术应用在操作系统IOS10上以保护用户的设备数据,谷歌公司同样使用该技术从Chrome浏览器采集用户的行为统计数据\cite{ref34}。


\section{神经网络}
如图\ref{fig:深度神经网络结构图}所示,深度神经网络基于模块化思想,通过在多个层次上部署多个神经元并通过逐层训练的手段调整神经元间的连接权值,从而实现原始特征数据进行多次非线性变换,对于任何有限给定输入/输出数据的拟合,最终获取到稳定的特征用于后续的问题分析。
\begin{figure}[!hbt]
\centering
	\includegraphics[scale=0.7]{fig2/C2/深度神经网络结构图}%
	\caption{深度神经网络结构图}
	\label{fig:深度神经网络结构图}	
\end{figure}

深度神经网络算法中,为评估所提神经网络输出预测值与真实值之间的差异程度,用损失函数 $L$ 表示, 文中采用均方差损失函数,表示为:
$$
L(\theta, x)=\frac{1}{n} \sum_{i=1}^{n}\left(y_{i}-x_{i}\right)^{2}
$$
式中: $\theta$ 为待训练的神经网络权重系数; $x$ 表示目标值; $y$ 表示预测值输出,下标 $i$ 表示样本标签。深度神经网 络算法训练的目的就是使得损失函数 $L$ 最小。而对于 复杂的神经网络而言,最小化损失函数 $L$ 通常采用随机 梯度下降( stochastic gradient descent, SGD)算法来完成。 即每次迭代过程中随机进行批量抽取训练样本 (记为 $B)$, 并计算损失函数 $L$ 的偏导数 $g_{B}=\frac{1}{|B|} \sum_{x \in B} \nabla_{\theta} L(\theta,$, $x$ ),然后沿着负梯度方向 $-g_{B}$ 朝向局部最小值进行更新权重系数 $\theta_{\circ}$

\section{本章小结}
本章对论文需要使用的一些基础理论知识进行了讨论。主要介绍了联邦学习系统的学习协议以及差分隐私的基本概念、定义和定理,分布式联邦学习系统是本论文主要使用的系统架构,所提的攻击模型和隐私对策都是基于该分布式联邦学习系统。本章同时也介绍了差分隐私及其变体的概念、实现机制。最后介绍了联邦学习中各个神经网络的基本结构和随机梯度下降算法。

\chapter{本地自适应差分隐私SGD算法}

\label{ch3}

\section{引言}
与传统的集中式深度学习相比,联邦学习通过分布式训练在一定程度上缓解了隐私泄漏的问题。然而,许多研究表明深度学习技术可以“记忆”模型中的训练数据信息,在训练过程中,本地设备与中央服务器之间的通信信道和传递的模型参数都有可能成为第三方窃取敏感信息的途径,联邦学习的框架仍然存在本地训练数据泄漏等隐私威胁\upcite{ref48}。在这种情况下,敌方一旦通过白盒推理攻击或者黑盒推理攻击访问模型,就可以推演出客户端本地的训练数据。

在第二章的基础知识中我们介绍了联邦学习模型的整体流程,联邦学习模型的优化问题可以概括为ERM(经验风险最小化)问题\upcite{ref40}:
\begin{equation}\label{eq:ERM}
\arg \min _{\theta \in \mathcal{C}}\left(F(\theta):=\frac{1}{m} \sum_{i=1}^{m} F_{i}(\theta)\right)
\end{equation}

通过优化ERM函数间接优化模型参数,使模型的实际输出与预测值更加接近,模型的准确率越高。从隐私保护的角度讲,我们只要截断了从原始输入到输出,在其中加入一道隐私保护屏障,根据在哪一步截断将差分隐私保护联邦学习的方法分为以下几种:
\begin{itemize}
	\item \textbf{输入扰动:} 输入扰动是在获取的训练数据上直接添加噪声,之后的模型训练和优化都是基于加躁后的训练数据\upcite{ref37}\upcite{ref38}\upcite{ref39}。
	\item \textbf{输出扰动:} 输出扰动沿袭了拉普拉斯机制最简单的思路,即考虑函数输出的敏感度来添加噪声,那么在ERM公式中我们只需要考虑argmin函数输出的敏感度,基于这个敏感度来添加拉普拉斯噪声即可得到一个简单的满足差分隐私的ERM方法\upcite{ref36}。
	\item \textbf{梯度扰动:} 梯度扰动是在执行最小化损失函数的过程中,设计满足差分隐私的算法。
	\item \textbf{目标扰动:} 目标扰动是在模型的目标函数中添加一个随机量,以使得最终模型的输出满足随机性。
\end{itemize}

基于输入的扰动和输出的扰动基本可以视为一个黑匣子模型,这种添加噪声的方式虽然简单直接,但无法对训练过程中数据的相互依赖性和输出有效性作出有用的、紧密的描述。在输入数据中加入过多的噪声,可能会影响模型训练的收敛性。在输出参数中加入过于保守的噪声,也就是根据最坏的攻击情况去添加噪声,可能会影响模型的实用性。

当前在深度学习模型中应用差分隐私的主流方案是在模型的梯度上添加噪声,方案的目标是在满足差分隐私的条件下,实现整体模型的最优可用性。Song等人\upcite{ref47}提出了一个$\left(\epsilon_{c}+\epsilon_{d}\right)$-差分隐私版本的随机梯度下降算法,在本地模型的每一次迭代过程中对梯度添加高斯噪声,并通过差分隐私的组合性和隐私放大效果,得到完全隐私损失的上界。Goodfellow\upcite{ref64}提出了$\ell_{2}$范式梯度裁剪的方式以限制函数敏感度,并设计了“Moments Accountant”(MA)来计算更准确的隐私预算估计,在预训练过程中,该方法与PCA相结合,形成了一个满足$\left(\epsilon_{c}+\epsilon_{d}\right)$-差分隐私的PCA。

由Song等人\upcite{ref47}中的实验数据可知,差分隐私随机梯度下降 (DP-SGD) 与SGD 相比严重降低了训练模型的效用。当差分隐私隐私提供的隐私强度增加时,在MNIST数据集上进行逻辑回归的训练和验证的损失率迅速增加。在MNIST上数据集上,采用DP-SGD训练的卷积神经网络(CNN)的测试精度比SGD低得多。

在传统的基于差分隐私的联邦学习模型中,数据管理者倾向于给每个用户的数据相同的隐私预算,同样的隐私预算忽略了用户之间的差异。有些用户希望有更好的隐私保护,而有些用户对某些数据的隐私不敏感。由于联邦学习模型是分布式结构,从一个大数据库到许多小数据库,所以对于每个用户来说,他们只需要关心他们自己的隐私。他们可以设置不同的隐私预算方案,而不是传统的统一分配,然后在最坏的的情况下注入噪音,而且基于梯度扰动的方法的问题在于它们的迭代性质会导致隐私预算的飙升。因此,当前的主要挑战是设计一种新型的满足差分隐私的扰动算法,使其在联邦学习中能按照用户隐私标准,给不同的用户分配不同的隐私预算,既能保证模型的效用性,并且维持较高的计算效率,在模型准确率方面接近非差分隐私的深度神经网络模型。

本文的创新点在于采用一种更加复杂的方法来分析训练过程中训练数据对模型输出的贡献率。在预训练阶段,我们采用逐层关联传播算法在神经网络的前向传播和反向传播流程中分解模型输出,计算网络中每个神经元对模型输出的贡献率。基于每个神经元的贡献率,我们通过高斯机制生成一个噪音图元,然后根据每一层神经网络对模型输出的贡献率,在梯度上根据贡献率添加对应的噪声量,避免了无法量化的噪声,从而提升了隐私保护模型的准确性。在进行梯度下降之前,根据神经网络的层次对梯度进行自适应的裁剪,保证模型的快速瘦脸。我们的方法不仅产生了在模型精度方面最接近非差分隐私的模型,而且还降低了隐私预算的成本。

\section{模型设计}

\subsection{模型概览}
传统的联邦学习中使用差分隐私的主要流程如下所示:
\begin{itemize}
\item \textbf{本地计算:}
客户端 $\mathrm{i}$ 根据本地数据库 $\mathcal{D}_{\mathrm{i}}$ 和接受的服务器的全局模型 $\mathrm{w}_{\mathrm{G}}^{\mathrm{t}}$ 作为本地的参数,即 $\mathrm{w}_{\mathrm{i}}^{\mathrm{t}}=\mathrm{w}_{\mathrm{G}}^{\mathrm{t}}$, 采用梯度下降策略进行本地模型训练得到 $\mathrm{w}_{\mathrm{i}}^{\mathrm{t}+1} \quad(\mathrm{t}$ 表示当前通信回合) 。

\item \textbf{模型扰动:}
每个客户端产生一个随机噪音 $\mathrm{n},\mathrm{n}$ 是符合高斯分布的,使用 $\overline{\mathbf{w}_{\mathrm{i}}}^{\mathrm{t}+1}=\mathrm{w}_{\mathrm{i}}^{\mathrm{t}+1}+\mathrm{n}$ 扰动本地模型 (这里注意w是一个矩阵,n表示对矩阵的每一个元素添加噪音)。

\item \textbf{模型聚合:}
中央服务器使用参数聚合算法聚合从客户端收到的 $\overline{\mathrm{w}}_{\mathrm{i}} \mathrm{t}+1$ ,得到新的全局模型参数 $\mathrm{w}_{\mathrm{G}}^{\mathrm{t}+1}$, 也就是扰动过的模型参数。

\item \textbf{模型广播:}
中央服务器服务器将新的模型参数广播给每个客户端。

\item \textbf{全局收敛:}
重复步骤(1)-(4)直至全局模型收敛。
\end{itemize}

在本文中,我们认为中央参数服务器是一个"诚实但好奇"(Honest but Curious, HbC)的实体。也就是说,服务器将遵循与所有用户的协议。然而,通过利用通信信道访问用户梯度的便利,它也试图在训练过程中反推出关于客户端的额外的信息。出于这个原因,我们设计的本地自适应加噪算法目的是保护发送到服务器的本地梯度不会被中央服务器推断出任何关于用户的本地训练样本信息,并且尽量维持原有模型的精度。

本文的模型主要针对本地设备的模型训练添加自适应的噪声和梯度裁剪算法,保证中央服务器接收到满足$\epsilon$-差分隐私的权重,从而减轻成员推理攻击对联邦学习所构成的隐私威胁。本地客户端进行模型训练的基本流程如图\ref{fig:自适应差分隐私SGD算法流程图}所示,通过从数据集D中随机采样构造各批次的样本,对于每一批样本通过梯度下降算法进行模型训练,在梯度上添加高斯噪声,使得最终的输出满足$\epsilon$-差分隐私,然后在梯度聚合之前通过梯度自适应裁剪算法加快模型的收敛速度。

\begin{figure}[!hbt]
\centering
	\includegraphics[scale=0.2]{fig2/C3/自适应差分隐私流程图}%联邦学习的系统架构
	\caption{自适应差分隐私SGD算法流程图}
	\label{fig:自适应差分隐私SGD算法流程图}	
\end{figure}

为实现相同的效用保证,我们算法的梯度复杂度(即计算的随机梯度总数)为$O\left(n^{3 / 2}\right)$,比之前的最佳结果高出$\Theta\left(n^{1 / 2}\right)$。 之后我们在MNIST和FICAR-10数据集上进行实验,评估我们提出的方法,并与前人的四种差分隐私保护方案进行对比,发现我们的方法不仅产生了在模型精度方面最接近非差分隐私的模型,而且还降低了隐私预算。并且,我们针对添加本地自适应差分隐私方案的模型进行成员推理攻击,评估了该方案的隐私保护效用。

总的来说,本章提出的隐私保护方案是基于本地客户端的本地数据维度,从以下两个方面展开研究:
\begin{enumerate}
\item [(1)] 通过在本地模型训练的梯度下降算法过程中针对不同神经元的贡献率添加对应比例的高斯噪声。
\item [(2)] 根据训练轮数和梯度变化的偏差与方差动态的更新梯度裁剪阈值,对梯度进行自适应裁剪。
\end{enumerate}

\subsection{梯度的自适应加躁算法}
在第二章我们详细介绍了联邦学习的流程和神经网络的结构,每个用户在本地设备的私有数据集上进行训练。本地训练的流程主要概括为前向传播和反向传播。神经网络中前向传播算法的第一步在输入层,我们使用前馈神经网络接收输入的$x$运行前向传播算法,得到预测值,然后通过反向传播算法不断调整参数使与预测值和真实值之间的误差降低。

Bach等人提出了针对神经网络的逐层关联传播算法\upcite{ref65},这是一种用于计算特定图像区域对分类器输出影响的技术。该技术被认为是非常有效的,可以逐个像素地分解图像的预测值\upcite{ref76},可以应用在深度神经网络中对各神经层进行循环重正化\upcite{ref77}。它允许分解深度神经网络的预测值。由于经典的反向传播算法是通过链式法则对模型的输出进行反向求导,根据损失函数计算每个神经元对模型总误差的贡献值,然后来调整模型的权重,以降低总误差。根据这一思路,在我们的工作中我们利用逐层关联传播算法将神经网络的输出值按层进行分解,得到每层的神经元对于模型输出的贡献比,然后根据神经元的贡献率,在梯度下降的过程中添加对应比例隐私预算的高斯噪声。

在逐层关联传播算法中,根据神经网络的结构自前向后计算归因值,也称为相关性分数,它假设一个分类器可以被分解成多个编译层。在卷积神经网络结构中,每个隐藏神经元的转化过程表示为$y=\sigma(\mathbf{x} * \omega+b)$,其中$\mathbf{x}$代表输入向量,$y$是输出,$b$和$\omega$分别代表{}偏置项和权重矩阵。$\sigma()$是一个激活函数,用于结合线性变换和非线性变换。在前向传播过程中,图像输入以像素的形式进入网络,然后与网络权重w相乘,再加上一个偏置b,并通过激活函数使其成为非线性计算,这个过程一直持续到输出层得到模型输出。

在逐层关联传播算法中,网络输出表示为F(x),其中F表示一个分类器,x是以像素形式存在的图像($x_{1}$,$x_{2}$,$x_{3}$...,$x_{n}$)。网络的输出值F(x)以反向传播的方式从输出层反馈到隐藏层,逐层分解每个像素的归因分数,直到到达输入层,反向传播的过程就结束了。对于卷积神经网络等网络结构的前向传播算法,模型输出表示为$F_{\theta}$,输出层的神经元表示为$z$,$Cr_{z}$表示网络总归因分数,逐层关联传播算法利用$F_{\theta}$和$Cr_{z}$在反向传播算法中逐层分解计算各层神经元对应的归因分数。输出层的归因分数通常作为输出层的预激活值,在反向传播过程中,每一层的所有神经元的归因分数总和是恒定。

\begin{figure}[!hbt]
\centering
	\includegraphics[scale=0.22]{fig2/C3/自适应-前向传播}%联邦学习的系统架构
	\caption{逐层关联传播算法:根据前向传播计算总归因分数}
	\label{fig:逐层关联传播算法:根据前向传播计算总归因分数}	
\end{figure}

接着,我们具体阐述如何计算网络中某一个神经元$a_{j}$的归因分数。箭头$(\rightarrow)$和$(\leftarrow)$分别代表第$l_{m}$层和第$l_{m+1}$层的前向连接和反向连接关系, $r_{n \rightarrow z}$表示由神经元n到神经元z的前向传播关系,$R_{n \leftarrow z}^{(l_{m}, l_{m+1})}$表示第m+1层的神经元z到第m层的神经元n的反向传播关系。如图\ref{fig:逐层关联传播算法:根据前向传播计算总归因分数}所示,在前向传播算法中,神经元$a_{i1}$接受上一层输入的像素值 $x_{1}$并计算$r_{a_{i1} \rightarrow a_{j}}^{(l_{m}, l_{m+1})}=\sigma\left(x_{1} * \omega+b\right)$,然后将计算结果传输给第$l_{m+1}$层的神经元$a_{j}$;同理,输出层$l_{m+1}$的神经元$a_{j}$接受来自上一层神经元$a_{i2}$的计算结果$r_{a_{i2} \rightarrow a_{j}}^{l_{m}, l_{m+1}}=\sigma\left(x_{2} * \omega+b\right)$,那么模型输出的结果为$r_{a_{j}}=r_{a_{i1} \rightarrow a_{j}}+r_{a_{i2} \rightarrow a_{j}}+b_{a_{j}}$。具体的来说,神经元$a_{4}$接收来自输入层的神经元$a_{1}$、$a_{2}$、$a_{3}$的计算结果,分为别$\sigma\left(x_{1} * \omega+b\right)$、$\sigma\left(x_{2} * \omega+b\right)$、$\sigma\left(x_{3} * \omega+b\right)$,神经元$a_{4}$接收的总输入值为$r_{a_{1} \rightarrow a_{4}}^{(l_{1}, l_{2})}+r_{a_{2} \rightarrow a_{4}}^{(l_{1}, l_{2})}+r_{a_{3} \rightarrow a_{4}}^{(l_{1}, l_{2})}+b_{a_{4}}$。

$C r_{a_{i}}^{l_{m}}\left(x_{i}\right)$表示第$m$层的神经元 $a_{i}$对于模型输出的归因分数。因为输出层的神经元$a_{4}$的归因分数等于模型的输出,所以有$C r_{a_{4}}^{l_{4}}\left(x_{i}\right)=y_{1}$。

根据神经网络相邻层间的线形关系和反向传播算法,神经元$a_{i}$的归因分数$Cr_{a_{i}}^{l_{m}}\left(x_{i}\right)$即为与之相邻的第$m+1$层的所有神经元${a_{j}\in l_{m+1}}$的归因分数总和:
$$
Cr_{a_{i}}^{l_{m}}\left(x_{i}\right)=\sum_{a_{j} \in l_{m+1}} Cr_{a_{i} \leftarrow a_{j}}^{l_{m} \leftarrow l_{m+1}}\left(x_{i}\right)
$$

根据层间关联传播算法,第m层和第m+1层的反向关联性表示为:
$$
R_{a_{i} \leftarrow a_{j}}^{(l_{m}, l_{m+1})}=\left\{\begin{array}{l}\frac{r_{a_{i}  \rightarrow a_{j} }}{r_{a_{j}}+b_{a_{i}}} \cdot C r_{a_{j}}^{l_{m} }, r_{a_{j}} \geq 0 \\ \frac{r_{a_{i} \rightarrow a_{j} }}{r_{a_{j} }-b_{a_{i}}} \cdot C r_{a_{j}}^{l_{m} }, r_{a_{j}}<0\end{array}\right.
$$

\begin{figure}[!hbt]
\centering
	\includegraphics[scale=0.3]{fig2/C3/自适应-反向传播}%联邦学习的系统架构
	\caption{逐层关联传播算法:根据反向传播计算各神经元的归因分数}
	\label{fig:逐层关联传播算法:根据反向传播计算各神经元的归因分数}	
\end{figure}

因为位于输出层的神经元$a_{j}$的贡献等于模型的输出,第m-1层的神经元$a_{j}$对于第m层的神经元的归因分数$Cr_{a_{i} \leftarrow a_{j}}^{l_{m-1} \leftarrow l_{m}}\left(x_{i}\right)$等于:
\begin{equation}
Cr_{a_{i} \leftarrow a_{j}}^{l_{m-1} \leftarrow l_{m}}\left(x_{i}\right)=\left\{\begin{array}{cc}\frac{a_{i} w_{i, j}}{\sum_{a_{i} \in l_{m-1} a_{i} w_{i, j}}} Cr_{a_{j}}^{l_{m}}\left(x_{i}\right) & \sum_{a_{i} \in l_{m-1}} a_{i} w_{i, j} \neq 0 \\ \mu & \sum_{a_{i} \in l_{m-1}} a_{i} w_{i, j}=0\end{array}\right.
\end{equation}

其中$\mu$是一个无限接近于零,但大于零的数字。从上述公式中,我们可以认为每一层的贡献是相等的,而且贡献是逐层传递的。根据以上公式的推导,我们能得到神经网络模型中每一层以及每个神经元的贡献值。那么模型的输出可以表示为各层神经元归因分数的累加和:
$$
\begin{aligned}
\sum f\left(x_{i}, \omega_{i}^{r}\right) &=Cr_{a_{11}}^{l_{4}}\left(x_{i}\right)+Cr_{a_{12}}^{l_{4}}\left(x_{i}\right)+Cr_{a_{13}}^{l_{4}}\left(x_{i}\right)\\
&+Cr_{a_{7}}^{l_{3}}\left(x_{i}\right)+Cr_{a_{8}}^{l_{3}}\left(x_{i}\right)+Cr_{a_{9}}^{l_{3}}\left(x_{i}\right)+Cr_{a_{10}}^{l_{3}}\left(x_{i}\right)\\
&+Cr_{a_{4}}^{l_{2}}\left(x_{i}\right)+Cr_{a_{5}}^{l_{2}}\left(x_{i}\right)+Cr_{a_{6}}^{l_{2}}\left(x_{i}\right)\\
&+Cr_{a_{1}}^{l_{1}}\left(x_{i}\right)+Cr_{a_{2}}^{l_{1}}\left(x_{i}\right)+Cr_{a_{3}}^{l_{1}}\left(x_{i}\right)
\end{aligned}
$$

通过从上述公式中提取同一属性的贡献,我们可以计算出每个属性类对模型输出的平均贡献:
\begin{equation}\label{eq:属性添加自适应扰动}
Cr_{j}\left(x_{i}\right)=\frac{1}{n} \sum_{i=1}^{n} Cr_{x_{i, j}}\left(x_{i}\right), j \in[1, u]
\end{equation}

% 然后,我们引入了两个调整参数$C$和$p$。其中,$C$代表贡献率阈值,用于决定梯度对模型结果输出的归因分数是高还是低,其值由联邦学习中的本地用户自己定义,即贡献超过阈值$C$的梯度对输出的贡献更大,我们向所有这些梯度注入自适应高斯噪声。当归因分数低于阈值$C$时,对这些梯度进行概率选择。也就是说,我们抛弃概率为$1-p$的原始数据,并对一些概率为$p$的梯度注入自适应高斯噪声。该公式如下:
% \begin{equation}\label{eq:神经网络加噪}
% \tilde{x}_{i, j}=\left\{\begin{array}{ll}
% \ddot{x}_{i, j} & \alpha \geq C \\
% \bar{x}_{i, j} & \alpha<C
% \end{array}\right.
% \end{equation}

% 其中$\alpha$代表该层神经元的贡献率:$\alpha=\frac{\left|\ddot{Cr}_{j}\right|}{\sum_{j=1}^{u}\left|\ddot{Cr}_{j}\right|}$,根据贡献率阈值C将神经层分为两类。当$\alpha>C$时,根据贡献率对所有梯度添加对应的高斯噪声;当$\alpha<C$时,对属性进行概率选择:
% \begin{equation}\label{eq:神经网络加噪2}
% \bar{x}_{i, j}=\left\{\begin{array}{l}
% \ddot{x}_{i, j} \text { with probability } p \\
% x_{i, j} \text { with probability } 1-p
% \end{array}\right.
% \end{equation}

本文提出一种自适应噪声添加算法,针对每个梯度计算其贡献值,根据属性对于模型输出的贡献率添加对应比例的高斯噪声。在此方案中,隐私预算$\sigma_{l}$是根据各自的归因分数按比例分配给每个梯度:$\sigma$=$\sigma_{j}=\frac{u *\left|\ddot{Cr}_{j}\right|}{\sum_{j=1}^{u}\left|\ddot{Cr}_{j}\right|} * \sigma_{l}$,自适应高斯噪声按以下方式注入梯度中:
\begin{equation}\label{eq:神经网络加噪3}
\tilde{\mathbf{g}}_{t} \leftarrow \frac{1}{L} \sum_{i}\left(\overline{\mathbf{g}}_{t}\left(x_{i}\right)+\mathcal{N}\left(0, S_{f}^{2} \cdot \sigma^{2}\right)\right)
\end{equation}

算法\ref{梯度的自适应加躁算法}详细描述了梯度的自适应加躁算法,我们将自适应噪声添加算法与随机梯度下降算法相结合,关键的步骤包括:计算梯度及其归因分数,根据归因分数分配相应的隐私预算,在梯度上添加高斯噪声,最后进行梯度下降,在梯度下降的过程中不断消耗隐私预算,当隐私预算消耗殆尽时,模型停止训练,输出权重。\\

\begin{algorithm}[!htb]
	\caption{梯度的自适应加躁算法}
	\label{梯度的自适应加躁算法}
	\begin{algorithmic}[1]
		\footnotesize
		\STATE \textbf{输入:} 数据集 $\left\{x_{1}, \ldots, x_{N}\right\}$,损失函数$\mathcal{L}(\theta)=$ $\frac{1}{N} \sum_{i} \mathcal{L}\left(\theta, x_{i}\right)$,学习率$\eta_{t}$, 初始隐私预算$\sigma_{l}$, 批大小$L$
		\STATE 初始化:模型权重$\theta_{0}$
		\FOR{ $t \in[T]$}
			\STATE 以概率$L / N$随机采样一批数据集$L_{t}$
			\FOR{$x_{i} \in L_{t}$ }
				\STATE 根据逐层传播算法计算神经元对于模型输出的归因分数:$Cr_{a_{i}}^{l_{m}}\left(x_{i}\right)=\sum_{a_{j} \in l_{m+1}} Cr_{a_{i} \leftarrow a_{j}}^{l_{m} \leftarrow l_{m+1}}\left(x_{i}\right)$
				\STATE 计算神经元对模型输出的平均贡献:$Cr_{j}\left(x_{i}\right)=\frac{1}{n} \sum_{i=1}^{n} Cr_{x_{i, j}}\left(x_{i}\right), j \in[1, u]$
				\STATE 计算神经元对模型输出的贡献率:$\alpha=\frac{\left|\ddot{Cr}_{j}\right|}{\sum_{j=1}^{u}\left|\ddot{Cr}_{j}\right|}$
				\STATE 计算梯度:$\mathbf{g}_{t}\left(x_{i}\right) \leftarrow \nabla_{\theta_{t}} \mathcal{L}\left(\theta_{t}, x_{i}\right)$
				\STATE 根据贡献率分配相应的隐私预算:$\sigma==\frac{u *\left|\ddot{Cr}_{j}\right|}{\sum_{j=1}^{u}\left|\ddot{Cr}_{j}\right|} * \sigma_{l}$
				\STATE 在梯度上添加高斯噪声:$\tilde{\mathbf{g}}_{t} \leftarrow \left(\overline{\mathbf{g}}_{t}\left(x_{i}\right)+\mathcal{N}\left(0, S_{f}^{2} \cdot \sigma^{2}\right)\right)$
			\ENDFOR
			\STATE 计算平均加躁梯度:$\tilde{w}^{t}=\frac{1}{L} \sum_{x_{i} \in L_{t}} \tilde{w}^{t}(x_{i})$
			\STATE 梯度下降:$\theta^{t+1}=\theta^{t}-\eta^{t} \tilde{w}^{t}$
		\ENDFOR
		\STATE \textbf{输出:} $\theta_{t}$
	\end{algorithmic}
\end{algorithm}

\subsection{梯度的自适应裁剪算法}
在传统的差分隐私随机梯度下降算法中,提供隐私保护的常用技术是限制函数的敏感度并添加与敏感度界限成比例的高斯噪声。为此,我们需要在每一轮SGD上限制梯度的敏感性。而且,相关研究表明合适的梯度裁剪能加快模型的收敛速度。Abadi\upcite{ref65}等人提出通过梯度裁剪使得梯度保持在[-C,C]的范围内,以保证函数的敏感度有界。如果损失函数是可微的(如果不可微,则使用子梯度)和Lipschitz有界的,用Lipschitz 界限制梯度范数,并用它来推导出梯度的敏感度。 如果损失函数导数作为输入的函数有界(例如,在逻辑回归的情况下,可以通过最大可能的输入正则来限制梯度),从而推导出梯度敏感度。如果损失函数不像深度学习应用中那样具有已知的Lipschitz界,则很难推导出梯度范数的先验界。

在固定的梯度裁剪算法中,由于梯度的大小没有一个先验的界限,我们采用$\ell_{2}$-范数的固定值对每个梯度进行裁剪。假设用户上传的梯度向量为$\tilde{\mathbf{g}}_{t}$,根据固定梯度范数进行裁剪后,梯度缩放为$\mathbf{g} / \max \left(1, \frac{\|\mathbf{g}\|_{2}}{C}\right)$, 其中 $C$是梯度阈值。对于梯度的裁剪能保证梯度值小于梯度阈值时,也就是当$\|\mathbf{g}\|_{2} \leq C$ ,$\mathbf{g}$保持不变;当$\|\mathbf{g}\|_{2}>C$时, 它会按照裁剪比例缩小为 $C$。在每次训练迭代中,可以使用经验值来获得梯度正则的近似界限,并在损失函数近似界限处裁剪梯度。然而,经验值的可用性是一个强有力的假设,在没有经验值的情况下如何针对自适应添加的噪声裁剪梯度是一个难题。如果梯度阈值$C$ 的值太小,那么裁剪后的梯度会较小,算法添加的噪声较小时可能会破坏梯度估计的无偏性;另一方面,如果不对梯度进行裁剪,大量的噪声添加到每个梯度会导致模型的可用性大大降低。神经网络的架构、损失函数本身、数据的缩放都会影响裁剪范数的选择。本章节所设计的方案根据训练轮数和梯度变化的偏差与方差动态的更新梯度裁剪阈值,对梯度进行自适应裁剪。

% 假设在随机梯度下降算法的第$t$轮迭代过程中,梯度向量表示为$g^{t}=\left(g_{1}^{t}, g_{2}^{t}, . ., g_{d}^{t}\right)$。根据差分隐私的定义,除了第一个维度之外,不需要向其他维度添加太多噪声。这促使我们自适应地将不同的噪声水平添加到不同的维度,以最小化添加噪声后梯度的 $\ell_{2}$-范数。我们提出了两个额外的辅助向量 $a^{t}=\left(a_{1}^{t}, a_{2}^{t}, . ., a_{d}^{t}\right)$和$b^{t}=\left(b_{1}^{t}, b_{2}^{t}, . ., b_{d}^{t}\right)$,初始值为$a^{t}=(0,0, \ldots, 0)$,$b^{t}=(1, 1, \ldots 1)$。

在深度学习的模型训练中,模型的泛化能力取决于预测值的方差、偏差和数据的噪声。偏差度量的是模型预测值与真实值之间的偏离程度;方差度量的是训练数据的变动给模型预测结果带来的影响,也就是噪声的添加会影响梯度的方差,而随机梯度的方差决定了SGD算法的收敛速度,梯度的裁剪会影响偏差。因此我们更关注梯度更新的方差和偏差来决定如何对梯度进行裁剪。之前的梯度裁剪算法给梯度本身添加了额外的噪声,因此我们考虑根据训练时观察到的历史梯度的统计数据来设置梯度阈值,通过计算梯度更新的偏差和方差在每轮随机梯度下降中更新梯度梯度阈值。

首先,我们对梯度进行裁剪,裁剪后的梯度为$\hat{w}^{t}$:
\begin{equation}
\hat{w}^{t}=\operatorname{clip}\left(w^{t}, C^{t}\right) \triangleq w^{t} \cdot \min \left(1, \frac{C^{t}}{\left\|w^{t}\right\|_{2}}\right)
\end{equation}

然后对保留的梯度$\hat{w}^{t}$根据神经元的归因分数添加高斯噪声 $\mathcal{N}\left(0, S_{f}^{2} \cdot \sigma^{2}\right)$:
\begin{equation}
\tilde{w}^{t}=\hat{w}^{t}+N^{t} \quad N^{t} \sim \mathcal{N}\left(0, \sigma^{2} I\right)
\end{equation}

理想情况下,裁剪后的梯度对于模型的收敛的影响应该很小,因此我们希望在每一轮梯度下降算法中找到最佳的裁剪阈值$C^{t}$使$\mathbb{E}\left\|\tilde{w}^{t}-w^{t}\right\|^{2}$最小。 

根据三角不等式和Jensen不等式,更新前后的梯度方差与偏差可以表示为以下公式:
\begin{equation}\label{eq:梯度更新后的方差与偏差}
\operatorname{bias}\left(\tilde{w}^{t}\right) \leq \operatorname{bias}\left(w^{t}\right)+2 \mathbb{E}\left\|\tilde{w}^{t}-w^{t}\right\| \text {和} \operatorname{Var}\left(\tilde{w}^{t}\right) \leq 3 \operatorname{Var}\left(w^{t}\right)+6 \mathbb{E}\left\|\tilde{w}^{t}-w^{t}\right\|^{2}
\end{equation}

我们通过约束$\mathbb{E}\left\|\tilde{w}^{t}-w^{t}\right\|$找到最佳的裁剪阈值$C^{t}$,将上述公式转换为:
\begin{equation}\label{eq:梯度更新后的方差与偏差2}
\mathbb{E}\left\|\tilde{w}^{t}-w^{t}\right\|^{2}=\left\|w^{t}\right\|^{2}\left(1-\frac{1}{\max \left(1,\left\|w^{t}\right\|\right)}\right)^{2}+{C^{t}}^{2} \sigma^{2}
\end{equation}

公式\ref{eq:梯度更新后的方差与偏差2}中的第一项对应于变换后的梯度$w_{t}$可能被裁剪的情况,第二项对应于注入到裁剪梯度中的高斯噪声。理想情况下,我们希望能找到使上述表达式\ref{eq:梯度更新后的方差与偏差2}最小化的裁剪阈值$C^{t}$。

为了使预测的梯度值的偏差最小,根据上一轮迭代得到的加躁梯度,通过指数渐进平均估计可得:
\begin{equation}\label{eq:梯度偏差估计}
m^{t}=\beta_{1} m^{t-1}+\left(1-\beta_{1}\right) \tilde{w}^{t}
\end{equation}
其中$\beta_{1}$是指数移动平均线的衰减参数。

为了使预测的梯度值的方差最小, 假使梯度没有被裁剪时,根据$\tilde{w}^{t}=w^{t}+C^{t} N^{t}$,从$\mathbb{E}\left(\tilde{w}_{i}^{t}-m_{i}^{t}\right)^{2}$推导出$\mathbb{E}\left(w_{i}^{t}-m_{i}^{t}\right)^{2}$:
$$
\begin{aligned}
\mathbb{E}\left(w_{i}^{t}-m_{i}^{t}\right)^{2} &=\mathbb{E}\left(\tilde{w}_{i}^{t}-m_{i}^{t}\right)^{2}+\mathbb{E}\left(C^{t} N_{i}^{t}\right)^{2}+2 \mathbb{E}\left(-C^{t} N_{i}^{t}\right)\left(w_{i}^{t}+C^{t} N_{i}^{t}-m_{i}^{t}\right) \\
&=\mathbb{E}\left(\tilde{w}_{i}^{t}-m_{i}^{t}\right)^{2}-\mathbb{E}\left(C^{t} N_{i}^{t}\right)^{2}-2 \mathbb{E}\left(C^{t} N_{i}^{t}\right)\left(w_{i}^{t}-m_{i}^{t}\right) \\
&=\mathbb{E}\left(\tilde{w}_{i}^{t}-m_{i}^{t}\right)^{2}-{C^{t}}^{2} \sigma^{2}
\end{aligned}
$$
我们需要确保$\left(w_{i}^{t}-m_{i}^{t}\right)^{2}$满足上限和下限:
$$
\left(w_{i}^{t}-m_{i}^{t}\right)^{2} \approx \min \left(\max \left(\left(\tilde{w}_{i}^{t}-m_{i}^{t}\right)^{2}-{C^{t}}^{2} \sigma^{2}, h_{1}\right), h_{2}\right)
$$
其中,$h_{1}$和$h_{2}$为常数,我们使用上式的指数移动平均值来估计方差:
\begin{equation}\label{eq:梯度方差估计}
\begin{aligned}
v_{t} &=\min \left(\max \left(\left(\tilde{g}_{i}^{t}-m_{i}^{t}\right)^{2}-{C^{t}}^{2} \sigma^{2}, h_{1}\right), h_{2}\right) \\
\left(s_{i}^{t}\right)^{2} &=\beta_{2}\left(s_{i}^{t-1}\right)^{2}+\left(1-\beta_{2}\right) v_{t}
\end{aligned}
\end{equation}

梯度自适应裁剪算法在每个训练迭代时刻t设置梯度裁剪阈值$C^{t}$,其中每个迭代对应于一个minibatch的处理,接着跟踪训练过程中看到的每个批次的梯度规范。在每一轮的梯度聚合之后,根据公式\ref{eq:梯度偏差估计}和\ref{eq:梯度方差估计}计算梯度变化的方差和偏差,更新梯度裁剪阈值$C^{t}$使$\mathbb{E}\left\|\tilde{w}^{t}-w^{t}\right\|^{2}$最小。将自适应梯度裁剪应用到随机梯度下降算法中,具体算法如\ref{梯度的自适应裁剪算法}所示。梯度自适应裁剪算法的动态性导致了$C^{t}$的自适应设置,该设置由数据、网络和损失动态决定,而不是由用户在训练初始阶段设置固定的裁剪值,根据神经网络各层的均值和统计特征进行梯度裁剪既能限制敏感度有界,也能保留有效的梯度信息。

\begin{algorithm}[!htb]
	\caption{梯度的自适应裁剪算法}
	\label{梯度的自适应裁剪算法}
	\begin{algorithmic}[1]
		\footnotesize
		\STATE \textbf{输入:} 数据集 $\left\{x_{1}, \ldots, x_{N}\right\}$,损失函数$\mathcal{L}(\theta)=$ $\frac{1}{N} \sum_{i} \mathcal{L}\left(\theta, x_{i}\right)$,学习率$\eta_{t}$,隐私预算$\sigma$,批大小$L$,裁剪阈值$C^{0}$
		\STATE 初始化:模型权重$\theta_{0}$
		\FOR{ $t \in[T]$}
			\STATE 以概率$L / N$随机采样一批数据集$L_{t}$
			\FOR{$x_{i} \in L_{t}$ }
				\STATE 计算梯度:$\mathbf{w}_{t}\left(x_{i}\right) \leftarrow \nabla_{\theta_{t}} \mathcal{L}\left(\theta_{t}, x_{i}\right)$
				\STATE 梯度裁剪:$\hat{w}^{t}=\operatorname{clip}\left(w^{t}, C^{t}\right) \triangleq w^{t} \cdot \min \left(1, \frac{c}{\left\|w^{t}\right\|_{2}}\right)$
				\STATE 在梯度上添加高斯噪声:$\tilde{w}^{t}\leftarrow \left(\hat{w}^{t}\left(x_{i}\right)+\mathcal{N}\left(0, S_{f}^{2} \cdot \sigma^{2}\right)\right)$
			\ENDFOR
			\STATE 计算平均加躁梯度:$\tilde{w}^{t}=\frac{1}{L} \sum_{x_{i} \in L_{t}} \tilde{w}^{t}(x_{i})$
			\STATE 梯度下降:$\theta_{t+1} \leftarrow \theta_{t}-\eta_{t} \tilde{\mathbf{w}}_{t}$
			\STATE 根据公式\ref{eq:梯度方差估计}和\ref{eq:梯度偏差估计}计算梯度变化的方差和偏差,更新$C^{t}$
		\ENDFOR
		\STATE \textbf{输出:} $\theta_{t}$
	\end{algorithmic}
\end{algorithm}

\subsection{基于自适应差分隐私的SGD算法}

结合前两节所提出的自适应梯度加躁和裁剪算法,我们设计了算法\ref{基于自适应差分隐私的随机梯度下降算法}。在本地客户端训练过程中,在随机梯度下降算法中添加自适应噪声使算法整体满足$(\epsilon, \delta)$-差分隐私,最小化目标函数$f(\theta)=\frac{1}{N} \sum_{k=1}^{N} f_{k}(\theta)$,并使用自适应梯度裁剪算法更新梯度裁剪阈值。在随机梯度下降算法中,每一轮用户随机采样小批次的$B$个样本,对于样本集中的每条数据记录,根据逐层关联传播算法计算神经元的归因分数和梯度,根据梯度更新的方差和偏差选择合适的梯度裁剪阈值$C^{t}$裁剪梯度。完成梯度裁剪后,对梯度添加同归因分数等比例的高斯噪声。之后计算批次大小为$L$的样本集的平均加躁梯度,然后进行梯度下降,计算得到梯度更新的均值和标准差。之后不断重复采样迭代的进行梯度下降的训练,使目标函数最小,输出模型权重$\theta_{t}$。
\begin{algorithm}[!htb]
	\caption{基于自适应差分隐私的随机梯度下降算法}
	\label{基于自适应差分隐私的随机梯度下降算法}
	\begin{algorithmic}[1]
		\footnotesize
		\STATE \textbf{输入:} 目标函数$f(\theta)=\frac{1}{N} \sum_{k=1}^{N} f_{k}(\theta)$,学习率$\eta^{t}$,训练批次大小$L$,高斯噪声参数$\sigma_{l}$,裁剪阈值$C^{0}$
		\STATE 初始化:$m^{0}=0 \cdot 1, s^{0}=\sqrt{h_{1} h_{2}} \cdot 1$
		\STATE 随机初始化模型参数:$\theta^{0}$
		\FOR{ $t \in[T]$}
			\STATE 以概率$L / N$随机采样一批数据集$L_{t}$
			\FOR{$x_{i} \in L_{t}$}
				\STATE 根据逐层传播算法计算神经元对于模型输出的归因分数:$Cr_{a_{i}}^{l_{m}}\left(x_{i}\right)=\sum_{a_{j} \in l_{m+1}} Cr_{a_{i} \leftarrow a_{j}}^{l_{m} \leftarrow l_{m+1}}\left(x_{i}\right)$
				\STATE 计算神经元对模型输出的平均贡献:$Cr_{j}\left(x_{i}\right)=\frac{1}{n} \sum_{i=1}^{n} Cr_{x_{i, j}}\left(x_{i}\right), j \in[1, u]$
				\STATE 计算神经元对模型输出的贡献率:$\alpha=\frac{\left|\ddot{Cr}_{j}\right|}{\sum_{j=1}^{u}\left|\ddot{Cr}_{j}\right|}$
				\STATE 计算梯度:$\mathbf{g}_{t}\left(x_{i}\right) \leftarrow \nabla_{\theta_{t}} \mathcal{L}\left(\theta_{t}, x_{i}\right)$
				\STATE 梯度裁剪:$\hat{g}^{t}=\operatorname{clip}\left(g^{t}, C^{t}\right) \triangleq g^{t} \cdot \min \left(1, \frac{c}{\left\|g^{t}\right\|_{2}}\right)$
				\STATE 根据贡献率分配相应的隐私预算:$\sigma==\frac{u *\left|\ddot{Cr}_{j}\right|}{\sum_{j=1}^{u}\left|\ddot{Cr}_{j}\right|} * \sigma_{l}$
				\STATE 在梯度上添加高斯噪声:$\tilde{g}^{t}\leftarrow \left(\hat{g}^{t}\left(x_{i}\right)+\mathcal{N}\left(0, S_{f}^{2} \cdot \sigma^{2}\right)\right)$
			\ENDFOR
			\STATE 计算平均加躁梯度:$\tilde{g}^{t}=\frac{1}{L} \sum_{x_{i} \in L_{t}} \tilde{g}^{t}(x_{i})$
			\STATE 梯度下降:$\theta^{t+1}=\theta^{t}-\eta^{t} \tilde{g}^{t}$
			\STATE 根据公式\ref{eq:梯度方差估计}和\ref{eq:梯度偏差估计}计算梯度变化的方差和偏差,更新$C^{t}$
		\ENDFOR
		\STATE \textbf{输出:} $\theta_{t}$
	\end{algorithmic}
\end{algorithm}

在下一节,我们将给出自适应差分隐私SGD算法的隐私性证明和整体隐私预算$(\epsilon, \delta)$分析。

\section{隐私参数分析}

自适应差分隐私SGD算法在梯度下降过程中根据神经元的归因分数在梯度上添加高斯噪声,算法整体满足$(\epsilon, \delta)$-差分隐私。证明如下:

\begin{proof}
假设现有相邻数据集$D$和$D^{\prime}$,两个数据集上的第n条数据记录$x_{n}$和$x_{n}^{\prime}$不同,在模型上的输出分别为$F_{\theta}$和$F_{\theta}^{\prime}$,$C(D)$和$C(D^{\prime})$分别代表这两个数据集上所有属性值的归因分数之和:

\begin{equation}
Cr(D)=\left\{Cr_{j}\left(x_{i}\right)\right\}, j \in[1, u], \text {其中} Cr_{j}\left(x_{i}\right)=\frac{1}{n} \sum_{i=1}^{n} Cr_{x_{i, j}}\left(x_{i}\right), j \in[1, u], x_{i} \in D
\end{equation}

\begin{equation}
Cr\left(D^{\prime}\right)=\left\{Cr_{j}\left(x_{i}^{\prime}\right)\right\}, j \in[1, u], \text {其中} Cr_{j}\left(x_{i}^{\prime}\right)=\frac{1}{n} \sum_{i=1}^{n} Cr_{x_{i, j}^{\prime}}\left(x_{i}^{\prime}\right), j \in[1, u], x_{i}^{\prime} \in D^{\prime}
\end{equation}

根据神经元的归因分数在其梯度$\mathbf{g}_{t}\left(x_{i}\right) \leftarrow \nabla_{\theta_{t}} \mathcal{L}\left(\theta_{t}, x_{i}\right)$上添加自适应的噪声,其中隐私预算与归因分数成正比:$\sigma$=$\sigma_{j}=\frac{u *\left|\ddot{Cr}_{j}\right|}{\sum_{j=1}^{u}\left|\ddot{Cr}_{j}\right|} * \sigma_{l}$:
\begin{equation}\label{eq:神经网络加噪3}
\tilde{\mathbf{g}}_{t} \leftarrow \frac{1}{L} \sum_{i}\left(\overline{\mathbf{g}}_{t}\left(x_{i}\right)+\mathcal{N}\left(0, S_{f}^{2} \cdot \sigma^{2}\right)\right)
\end{equation}

通过敏感度公式计算梯度归因分数的$L_{2}$敏感度$S_{f}$:
\begin{equation}\label{eq:贡献敏感度}
\begin{aligned}
S_{f} &=\frac{1}{|D|} \sum_{j=1}^{u}\left\|\sum_{x_{i} \in D} Cr_{x_{i, j}}\left(x_{i}\right)-\sum_{x_{i}^{\prime} \in D^{\prime}} Cr_{x_{i, j}^{\prime}}\left(x_{i}^{\prime}\right)\right\|_{2} \\ &=\frac{1}{|D|} \sum_{j=1}^{u}\left\|Cr_{x_{n, j}}\left(x_{n}\right)-Cr_{x_{n, j}^{\prime}}\left(x_{n}^{\prime}\right)\right\|_{2} \\ & \leq \frac{2}{|D|} \max \sum_{j=1}^{u}\left\|C_{x_{i, j}}\left(x_{i}\right)\right\|_{2} \\ & \leq \frac{2 u}{|D|} 
\end{aligned}
\end{equation}

其中,$u$和$|D|$分别表示神经网络属性和元组的数量,然后可以得到:
\begin{equation}\label{贡献数量和元组}
\begin{aligned}
\frac{\operatorname{Pr}(\ddot{Cr}(D))}{\operatorname{Pr}\left(\ddot{Cr}\left(D^{\prime}\right)\right)} &=\frac{\prod_{j=1}^{u} \exp \left(\frac{\epsilon_{c}\left\|\frac{1}{|D|} \sum_{x_{i} \in D} Cr_{j}\left(x_{i}\right)-\ddot{Cr}_{j}\left(x_{i}\right)\right\|_{2}}{S_{f}}\right)}{\prod_{j=1}^{u} \exp \left(\frac{\epsilon_{c}\left\|\frac{1}{\left|D^{\prime}\right|} \sum_{x_{i}^{\prime} \in D^{\prime}} Cr_{j}\left(x_{i}^{\prime}\right)-\ddot{Cr}_{j}\left(x_{i}^{\prime}\right)\right\|_{2}}{S_{f}}\right)} \\
&=\prod_{j=1}^{u} \exp \left(\frac{\epsilon_{c}}{|D| S_{f}}\left\|Cr_{j}\left(x_{n}\right)-Cr_{j}\left(x_{n}^{\prime}\right)\right\|_{2}\right) \\
& \leq \prod_{j=1}^{u} \exp \left(\frac{\epsilon_{c}}{|D| S_{f}} \max \left\|Cr_{j}\left(x_{n}\right)\right\|_{2}\right) \\
&=\exp \left(\epsilon_{c} \frac{\max _{x_{i} \in D} \sum_{j=1}^{u}\left\|Cr_{j}\left(x_{n}\right)\right\|_{2}}{|D| S_{f}}\right) \\
& \leq \exp \left(\epsilon_{c}\right)
\end{aligned}
\end{equation}

% 因此,根据梯度归因分数添加高斯噪声的算法是满足$\epsilon_{c}$-差分隐私的。
根据上述推倒证明可知,在联邦学习的神经网络中添加自适应噪声后,所上传的梯度是满足$\epsilon_{c}$-差分隐私的。在满足差分隐私的基础上,在下一节我们结合差分隐私的组合定理,计算累积的隐私预算。
\end{proof}
% 接着我们根据归因分数对于梯度进行概率选择添加噪声,我们接下来证明这部分的算法是满足$\epsilon_{l}$-差分隐私的。

% 假设两个相邻的数据集$D_{i}^{t}$和$D_{i}^{t^{\prime}}$,其中数据记录$x_{n}$和$x_{n}^{\prime}$不同,$z\left(D_{i}^{t}\right)$和$z\left(D_{i}^{t^{\prime}}\right)$分别为两个数据集上训练采用的线性变换函数。

% 一般来说,我们把偏置项视为第一类数据属性,即:$x_{i,0}=b_{i}$。线性转换可以改写为:$\ddot{\mathbf{z}}_{x \in D_{i}^{t}}(\omega)=\ddot{\mathbf{x}} * \omega$。线性变换函数的敏感度$S_{f}$为:
% \begin{equation}
% \begin{aligned}
% S_{f} &=\sum_{a_{i} \in l_{1}} \sum_{j=1}^{u}\left\|\sum_{x_{i} \in D_{i}^{t}} x_{i, j}-\sum_{x_{i}^{\prime} \in D_{i}^{t^{\prime}}} x_{i, j}^{\prime}\right\|_{1} \\
% &=\sum_{a_{i} \in l_{1}} \sum_{j=1}^{u}\left\|x_{n, j}-x_{n, j}^{\prime}\right\|_{1} \\
% & \leq \sum_{a_{i} \in l_{1}} \sum_{j=1}^{u} \max _{x_{i} \in D_{i}^{t}}\left\|x_{n, j}\right\|_{1} \\
% & \leq \sum_{a_{i} \in l_{1}} u
% \end{aligned}
% \end{equation}

% 其中,$a_{i} \in l_{1}$是指第一个隐藏层$l_{1}$中的神经元$a_{i}$,$u$是数据元组$x_{i} \in D_{i}^{t}$中的属性数。自适应加躁算法包括两个超参数:$C$和$p$,这两个超参数可以适时过滤多余的噪声,再通过超参数调整噪声大小和裁剪阈值后,整体加躁后的梯度的表达式如下:
% \begin{equation}
% \begin{aligned}
% \tilde{x}_{i, j} &=[(1-C)+C * p] * \ddot{x}_{i, j}+C *(1-p) * x_{i, j} \\
% &=[(1-C)+C * p]\left[x_{i, j}+\mathcal{N}\left(0, S_{f}^{2} \cdot \sigma^{2}\right)\right]+[C *(1-p)] x_{i, j} \\
% &=x_{i, j}+[(1-C)+C * p]\left[\mathcal{N}\left(0, S_{f}^{2} \cdot \sigma^{2}\right)\right]
% \end{aligned}
% \end{equation}
% 然后我们可以得到:

% \begin{equation}
% \begin{aligned}
% \frac{\operatorname{Pr}\left(\ddot{\mathbf{z}}_{D_{i}^{t}}(\omega)\right)}{\operatorname{Pr}\left(\ddot{\mathbf{z}}_{D_{i}^{t}}(\omega)\right)} &=\frac{\prod_{a_{i} \in l_{1}} \prod_{j=1}^{u} \exp \left(\frac{\epsilon_{j}\left\|\sum_{x_{i} \in D_{i}^{t}} x_{i, j}-\sum_{x_{i} \in D_{i}^{t}} \tilde{x}_{i, j}\right\|_{1}}{G S_{l}}\right)}{\prod_{a_{i} \in l_{1}} \prod_{j=1}^{u} \exp \left(\frac{\epsilon_{j}\left\|\sum_{x_{i}^{\prime} \in D_{i}^{t^{\prime}}} x_{i, j}^{\prime}-\sum_{x_{i}^{\prime} \in D_{i}^{t^{\prime}}} \tilde{x}_{i, j}^{\prime}\right\|_{1}}{G S_{l}}\right)} \\
% & \leq \prod_{a_{i} \in l_{1}} \prod_{j=0}^{u} \exp \left(\frac{\epsilon_{j}}{G S_{l}}\left\|\sum_{x_{i} \in D_{i}^{t}} x_{i, j}-\sum_{x_{i}^{\prime} \in D_{i}^{t^{\prime}}} x_{i, j}^{\prime}\right\|_{1}\right) \\
% & \leq \prod_{a_{i} \in l_{1}} \prod_{j=0}^{u} \exp \left(\frac{\epsilon_{j}}{G S_{l}} \max _{x_{i} \in D_{i}^{t}}\left\|x_{n, j}\right\|_{1}\right) \\
% & \leq \exp \left(\epsilon_{l} \frac{\sum_{a_{i} \in l_{1}} u\left[\sum_{j=1}^{u} \frac{\left|\ddot{C}_{j}\right|}{\sum_{j=1}^{u}\left|\ddot{C}_{j}\right|}\right]}{G S_{l}}\right) \\
% &=\exp \left(\epsilon_{l}\right)
% \end{aligned}
% \end{equation}

本章所提出的自适应差分隐私保护方案是通过在随机梯度下降算法上添加自适应的高斯噪声,保护数据的隐私性。在上一部分我们已经证明了此算法满足$\epsilon_{c}$差分隐私,那另外一个非常重要的问题就是评估在训练过程中添加噪声所累积的隐私预算成本。

我们向梯度中添加高斯噪声以得到加躁后的数据,根据第二章所给出的高斯机制的定理可知,当隐私参数$\sigma=\sqrt{2 \log \frac{1.25}{\delta}} / \varepsilon$时,每一批次的输出都满足$(\varepsilon, \delta)$-差分隐私。考虑到训练批次是以概率$q=L / N$从数据集中随机采样的,根据隐私放大定理和差分隐私的强组合定理,隐私性由$(\varepsilon, \delta)$-差分隐私扩大到 $(q \varepsilon, q \delta)$-差分隐私。

然而,组合定理并没有考虑到特定的噪声分布,给出的隐私边界较为松散。在本节中,我们采用“Moments Accountant”(MA)机制,去计算算法迭代过程中添加噪声所累积的隐私预算成本。MA机制通过跟踪隐私损失随机变量随时间的变化,并结合组合定理,对于采样的高斯机制给出更严格的隐私损失估计。

隐私损失是一个随机变量,取决于添加到算法中的随机噪声。假使算法$\mathcal{M}$是满足 $(\varepsilon, \delta)$-差分隐私的,那么$\mathcal{M}$中的隐私损失随机变量是存在严格的尾部边界。尾部边界在概率分布中是一个非常重要的信息。我们通过计算隐私损失随机变量的对数矩
,结合标准的马尔科夫不等式,得到尾部边界,也差分隐私算法的隐私损失。

对于邻近数据集 $D, D^{\prime}$,算法$\mathcal{M}$,额外输入aux,算法的输出$o \in \mathcal{R}$的隐私损失表示为:
$$
c\left(o ; \mathcal{M}, \mathrm{aux}, D, D^{\prime}\right) \triangleq \log \frac{\operatorname{Pr}[\mathcal{M}(\mathrm{aux}, D)=o]}{\operatorname{Pr}\left[\mathcal{M}\left(\mathrm{aux}, D^{\prime}\right)=o\right]}
$$

由于在随机梯度下降算法$\mathcal{M}$中,通过迭代的梯度下降更新权重,每一轮输出的梯度满足差分隐私,根据差分隐私的串行组合定理,最终的模型输出也满足差分隐私。我们定义第$\lambda^{\text {th }}$个时刻的时刻生成函数:
$$
\begin{aligned}
\alpha_{\mathcal{M}}(\lambda ; \operatorname{aux},&\left.D, D^{\prime}\right) \triangleq\log \mathbb{E}_{o \sim \mathcal{M}(\operatorname{aux}, D)}\left[\exp \left(\lambda c\left(o ; \mathcal{M}, \text { aux }, D, D^{\prime}\right)\right)\right]
\end{aligned}
$$

为了证明给定算法$\mathcal{M}$的隐私保障,我们对于每一轮的$\alpha_{\mathcal{M}}\left(\lambda ;\right.$ aux $\left., D, D^{\prime}\right)$给出严格的尾部边界,考虑到所有可能的额外输入和邻近数据集$\left., D, D^{\prime}\right)$,整体的时刻函数为:
$$
\alpha_{\mathcal{M}}(\lambda) \triangleq \max _{\text {aux }, D, D^{\prime}} \alpha_{\mathcal{M}}\left(\lambda ; \text { aux }, D, D^{\prime}\right)
$$

$\alpha_{\mathcal{M}}(\lambda)$满足串行组合定理和尾部边界定理:

\begin{theorem}[时刻函数的串行组合]\label{时刻函数的串行组合}
假设算法$\mathcal{M}$是由一系列自适应算法$\mathcal{M}_{1}, \ldots, \mathcal{M}_{k}$组合而成的,满足$\mathcal{M}_{i}: \prod_{j=1}^{i-1} \mathcal{R}_{j} \times \mathcal{D} \rightarrow \mathcal{R}_{i}$,那么对于任意的$\lambda$都满足:
$$
\alpha_{\mathcal{M}}(\lambda) \leq \sum_{i=1}^{k} \alpha_{\mathcal{M}_{i}}(\lambda)
$$
\end{theorem}

\begin{theorem}[时刻函数的尾部边界]\label{时刻函数的尾部边界}
对于任意的 $\varepsilon>0$,算法$\mathcal{M}$是满足$(\varepsilon, \delta)$-差分隐私的,当
$$
\delta=\min _{\lambda} \exp \left(\alpha_{\mathcal{M}}(\lambda)-\lambda \varepsilon\right)
$$
\end{theorem}

根据定理\ref{时刻函数的尾部边界},在算法的每一次迭代中,通过计算$\alpha_{\mathcal{M}_{i}}(\lambda)$然后限定其尾部边界,结合\ref{时刻函数的串行组合},就可以得到整体时刻函数的尾部边界,通过隐私损失变量的尾部边界计算得到整体的隐私损失。


\section{实验评估}
\subsection{实验准备}
在这一节中,我们进行实验来评估本地自适应差分隐私在联邦学习系统中的性能。在模拟的联邦学习系统中,我们假设有四台本地设备,每个本地用户由配备6GB内存、四核2.36GHz Cortex A73处理器和四核Cortex A53 1.8GHz处理器的华为nova3安卓手机模拟。中央服务器由一台联想服务器模拟的,服务器有2个英特尔(R)至强(R)E5-2620 2.10GHZ CPU,32GB内存,512SSD,2TB机械硬盘,运行于Ubuntu 18.04操作系统。

在实验过程中,我们选择了深度学习中常用的两个经典数据集--MNIST手写体数字识别数据集\upcite{ref46}和CIFAR-10数据集\upcite{ref68}进行实验。
\begin{itemize}
\item MNIST数据集包含60000个训练样本和10000个测试样本。MNIST是用于分类任务的经典数据集,来源于美国国家标准与技术研究所。总共包含60000个训练样本和10000个测试样本。每个样本为28x28像素的手写数字图像,每个像素点用灰度值表示,灰度值范围为0到255,图像包含十个类别,如下图\ref{fig:MNIST手写数字数据集}所示。

\item CIFAR-10数据集包含了十个类别的RGB彩色图像,总共包含50000个训练样本和10,000个测试岩本。每张图像的大小为32*32。这些图像有以下十个类别:飞机、汽车、鸟、猫、鹿、狗、青蛙、马、船和卡车。

\end{itemize}


\begin{figure}[!hbt]
\centering
	\includegraphics[scale=0.6]{fig2/C3/MNIST}%联邦学习的系统架构
	\caption{MNIST手写数字数据集}
	\label{fig:MNIST手写数字数据集}	
\end{figure}

此外,我们让所有用户离线训练一个统一的卷积神经网络,以获得本地用户的梯度。在我们的实验中采用的模型网络结构为CNN,包括2个卷积层(分别包含20个特征图和50个特征图),两个池化层和两个全连接层(分别为256和10个神经元)。模型的激活函数为ReLU,并引入了DropOut正则以提高模型的泛化能力。图\ref{fig:CNN模型网络结构}展示了CNN的网络结构。

\begin{figure}[!hbt]
\centering
	\includegraphics[scale=0.6]{fig2/C3/CNN网络结构}%联邦学习的系统架构
	\caption{模型网络结构}
	\label{fig:CNN模型网络结构}	
\end{figure}

所有的实验都是用PYTHON语言编译的,我们使用Tensorflow去实现本地差分隐私算法,这是一个流行的深度学习库。 本文使用了TensorFlow-Federated,这是TensorFlow中的一个联邦学习库。我们在Python的基础上二次开发了该算法,并通过将该算法部署到多个边缘设备上构建了一个真实的联邦学习环境。以本地训练集MNIST为例,图\ref{fig:仿真联邦系统模型概览}展示了联邦学习的仿真模型概览。

\begin{figure}[!hbt]
\centering
	\includegraphics[scale=0.3]{fig2/C3/联邦系统仿真模型概览}%联邦学习的系统架构
	\caption{仿真联邦系统模型概览}
	\label{fig:仿真联邦系统模型概览}	
\end{figure}

\subsection{实验设计}
为了实现隐私保护,我们需要在本地训练的随机梯度下降算法中实现梯度的自适应扰动和裁剪,并采用MA机制跟踪每一次梯度扰动所增加的隐私预算。因此,代码的实现主要分为两个部分:梯度扰动(Sanitizer),隐私预算跟踪(Accountant)。

图\ref{fig:实现本地自适应差分隐私的伪代码片段}展示了实现梯度扰动和裁剪、隐私预算跟踪的代码片段,为了实现整体算法满足$\epsilon$-差分隐私,Sanitizer需要完成以下三个步骤:首先,通过前向传播算法计算每批样本的梯度及其归因分数;其次,根据每个样本的梯度范数进行梯度裁剪以限制函数敏感度;最后,根据归因分数在梯度上添加自适应的噪声然后更新权重。

\begin{figure}[!hbt]
\centering
	\includegraphics[scale=0.5]{fig2/C3/代码片段1}%联邦学习的系统架构
	\caption{实现本地自适应差分隐私的伪代码片段}
	\label{fig:实现本地自适应差分隐私的伪代码片段}	
\end{figure}

在TensorFlow中,由于性能原因,梯度计算是分批进行的,所以在训练过程,随机采样一批训练子样本$B$:
$\mathbf{g}_{B}=1 /|B| \sum_{x \in B} \nabla_{\theta} \mathcal{L}(\theta, x)$
。为了限制梯度更新的敏感度,我们需要计算每个批次的梯度$\nabla_{\theta} \mathcal{L}(\theta, x)$,具体由per\_example\_gradients函数实现。这样即使是大批量的训练,训练速度也不会大幅下降。在每个批次的训练中,我们会单独计算损失函数 
$\mathcal{L}$,也就是每个数据样本$x_{i}$都有单独的损失函数结果$\mathcal{L}$。一旦我们获得了每批数据样本的梯度,
我们可以很容易地使用TensorFlow操作符来对梯度进行裁剪,添加高斯噪声。

我们的实验主要分为三个部分:
\begin{enumerate}
\item [(1)] 针对本地自适应差分隐私SGD方案,分析噪声水平、裁剪阈值、隐藏层数量这些超参数对模型分类准确率影响。
\item [(2)] 将本地自适应差分隐私SGD方案与非隐私的SGD、前人提出的差分隐私SGD方案(如表所示)进行对比,比较各个方案在相同隐私预算的情况下模型分类所能达到的准确率和模型收敛速度。
\item [(3)] 在本地自适应差分隐私的联邦学习模型上应用成员推理攻击进行实验,评估模型的隐私保护效用。

\begin{table}[H]
	\centering
	\begin{tabular}{cc}
		\hline
		基准方案名称& 具体算法\\
		\hline
		SGD& 没有实现差分隐私的随机梯度下降算法\\
		DP-SGD\upcite{ref57}& 在梯度上添加固定噪声大小的差分隐私随机梯度下降算法\\
		DS-SGD\upcite{ref67}& 在梯度下降过程中,选择性的进行参数共享实现隐私保护\\
		LDP-SGD& 本地差分隐私方案\\
		ADP-SGD& 我们的改进方案,使用梯度自适应加躁与裁剪\\
		\hline
	\end{tabular}
	\caption{本地自适应差分隐私与其他四种基准方案}
	\label{tab1}
\end{table}

\end{enumerate}

\subsection{结果分析}

\subsubsection{实验一(分析各个参数对模型准确率的影响)} 
分类模型的精度由多个因素决定,这些因素包括网络的拓扑结构、隐藏单元的数量以及模型训练的参数,如批量大小和学习率,必须仔细调整以获得最佳性能。有些参数是针对隐私的,如梯度范数裁剪阈值和噪声水平。本节实验重点研究噪声大小,裁剪阈值和隐藏层数量这三种参数对于模型分类准确率的影响。为了准确的反映每种参数对于准确率的影响,我们控制变量的进行实验。参考值如下:1,000个隐性单元,600个批量,初始梯度范数裁剪阈值为4,初始学习率为0.1,在10个训练轮次中递减到最终学习率为0.052,噪声参数$\sigma$分别为2和5,用于训练CNN网络模型。对于每一种参数组合进行模型训练,直至隐私预算累积至$\left(2,10^{-5}\right)$-差分隐私。具体的实验结果如图\ref{fig:在MNIST数据集上噪声大小,裁剪阈值和隐藏层数量这三个参数对于训练准确率的影响}所示。

\begin{figure}[!hbt]
\centering
	\includegraphics[scale=0.32]{fig2/C3/第三章实验一}%联邦学习的系统架构
	\caption{在MNIST数据集上噪声大小,裁剪阈值和隐藏层数量这三个参数对于训练准确率的影响}
	\label{fig:在MNIST数据集上噪声大小,裁剪阈值和隐藏层数量这三个参数对于训练准确率的影响}	
\end{figure}

如图\ref{fig:在MNIST数据集上噪声大小,裁剪阈值和隐藏层数量这三个参数对于训练准确率的影响}(a)展示的是噪声参数$\sigma$对模型准确率的影响,X轴是噪声水平,这个值的选择对准确性有很大影响。由于噪声参数$\sigma$与噪声采样的分布的方差是呈反比的,这意味着$\sigma$越大,添加的噪声量越小。通过添加更多的噪音,每轮训练步骤的隐私损失成比例缩小,所以我们可以在给定的累积隐私预算内运行更多的训练轮次。该模型在训练集和验证集上模型的准确率维持在0.88\%至0.98\%之间,我们的框架允许对训练参数进行自适应控制减少了过拟合情况,自适应的噪声添加根据训练轮次合理分配隐私预算,同时提高了模型的准确性和训练性能。

图\ref{fig:在MNIST数据集上噪声大小,裁剪阈值和隐藏层数量这三个参数对于训练准确率的影响}(b)展示了隐藏层数量对模型准确率的影响,X轴表示隐藏层单元数量。对于非差分隐私模型,更多的隐藏单元能有效避免过度拟合,因为更多的隐藏单元会让我们的训练更有针对性。然而,添加了差分隐私的模型训练,隐藏层数量的增加可能会影响梯度的敏感度,使每次梯度更新时添加更多的噪声。针对这个问题,我们的根据梯度的归因分数自适应添加噪声的方案能有效的控制敏感度有界,随着隐藏单位的数量增加,模型的准确率依然维持在93\%以上。

图\ref{fig:在MNIST数据集上噪声大小,裁剪阈值和隐藏层数量这三个参数对于训练准确率的影响}(c)展示了梯度范数裁剪阈值对模型准确率的影响,X轴表示梯度$\ell_{2}$-范数裁剪阈值$C^{0}$。当$C^{0}$为2-3时,模型准确率最高,接着随着$C^{0}$的增加,模型准确率逐渐降低至91\%左右,限制梯度范数会产生两个相反的效果:剪裁破坏了梯度估计的无偏性,如果剪裁参数太小,被剪切的平均梯度可能与真实梯度的方向大不相同。另一方面,增加裁剪阈值迫使我们在梯度中加入更多的噪声,也就是以$\sigma$$C^{0}$的比例添加噪声。而我们的自适应梯度裁剪方案能有效的考虑上一轮训练得到的梯度偏差和方差,取训练过程中未被剪辑的梯度范数的中值,模型准确率最高依然能达到95\%左右。

对于不同隐私预算的训练版本,我们用同样的架构进行了实验:一个包括1000个神经元的ReLU隐藏层,以及600个批量大小。为了限制敏感度,我们设置初始梯度范数裁剪阈值$C^{0}$为3。我们报告了四种隐私参数的训练结果,分别为小($\epsilon$=0.5)、中($\epsilon$=2)、大($\epsilon$=4)和更大($\epsilon$=8),固定$\delta=10^{-5}$,$\sigma$=6,这里$\epsilon$代表训练神经网络的隐私保护水平。学习率最初设置为0.1,在10个训练回合后线性下降到0.052,然后固定为0.052。

图\ref{fig:在MNIST数据集上不同隐私预算下训练的准确率}显示了不同隐私预算下的训练的结果,在每张图中,我们都显示了训练集和测试集上准确率的变化情况。对于隐私预算为$\left(0.5,10^{-5}\right)$,$\left(2,10^{-5}\right)$,$\left(4,10^{-5}\right)$和$\left(8,10^{-5}\right)$的差分隐私,分别达到81\%、94\%、95\%和97\%的测试集准确性。由图\ref{fig:在MNIST数据集上不同隐私预算下训练的准确率}(a)所示,当$\epsilon$=0.5时,由于给定的隐私预算较小,在训练的第30个轮次隐私预算消耗殆尽,模型基本趋近收敛,模型的准确率最高达到81\%。虽然在传统的差分隐私保护中,0<$\epsilon$<1时被认为能提供较强的隐私保护。而在联邦学习的深度神经网络中应用差分隐私时,由于网络的复杂性和训练多次大量迭代,导致隐私预算消耗的很快,0<$\epsilon$<1的取值会导致模型训练的精度大大下降。在深度学习方面应用差分隐私的大量研究表明,当0<$\epsilon$<=10时,能提供较强的隐私保护效果。如图\ref{fig:在MNIST数据集上不同隐私预算下训练的准确率}(b)(c)(d)所示,当$\epsilon$=2,4,8时,随着训练轮数的增加,模型均能达到收敛。我们的自适应差分隐私SGD方案,使模型在训练集和测试集上的准确度差异很小,这与理论上的观点一致,即添加噪声后的梯度训练依然有很好的泛化作用。相比之下,非差分隐私SGD的训练和测试准确率之间的差距随着训练轮次的增加而增加,容易造成过拟合。在噪声参数为$\left(8,10^{-5}\right)$时,模型在600个训练轮次后能达到接近非差分隐私模型的准确率。

\begin{figure}[!hbt]
\centering
	\includegraphics[scale=0.6]{fig2/C3/第三章实验一2}%联邦学习的系统架构
	\caption{在MNIST数据集上不同隐私预算下训练的准确率}
	\label{fig:在MNIST数据集上不同隐私预算下训练的准确率}	
\end{figure}

\subsubsection{实验二(与前人的隐私保护方案进行对比实验)} 
我们将本文提出的本地自适应差分隐私方案(ADP-SGD)与SGD、DP-SGD、DS-SGD、LDP-SGD方案进行对比实验,选取的数据集为MNIST和CIFAR-10,网络模型为CNN5,具体的参数设置如下表所示。我们比较了不同方案在给定相同的隐私预算情况下,在测试集上所计算的目标函数的平均损失误差变化情况。

\begin{table}[H]
	\centering
	\begin{tabular}{ccc}
		\hline
		参数& MNIST& CIFAR-10\\
		\hline
		隐私预算$\epsilon$& 0.5/2/4/8& 0.5/2/4/8\\
		批大小& 600& 600\\
		初始梯度裁剪阈值& 3& 3\\
		学习率& 0.05& 0.05\\
		本地设备数量& 1000& 100\\
		训练轮数& 100& 100\\
		\hline
	\end{tabular}
	\caption{对比实验在数据集MNIST和CIFAR-10上的参数设置}
	\label{tab1}
\end{table}

图\ref{fig:不同隐私保护方案在MNIST数据集上训练的测试误差变化情况}显示了当隐私预算为$\left(0.5,10^{-5}\right)$,$\left(2,10^{-5}\right)$,$\left(4,10^{-5}\right)$和$\left(8,10^{-5}\right)$时,不同方案在MNIST数据集上平均测试误差随训练轮次的变化情况。

\begin{figure}[!hbt]
\centering
	\includegraphics[scale=0.9]{fig2/C3/第三章对比实验}%联邦学习的系统架构
	\caption{不同隐私保护方案在MNIST数据集上训练的测试误差变化情况}
	\label{fig:不同隐私保护方案在MNIST数据集上训练的测试误差变化情况}	
\end{figure}

\begin{table}[H]
	\centering
	\begin{tabular}{ccc}
		\hline
		\diagbox{算法}{隐私预算}& $2,10^{-5}$& $4,10^{-5}$\\ %添加斜线表头
		\hline
		SGD& 98.6\%& 98.6\%\\
		DP-SGD& 88.9\%& 94.0\% \\
		DS-SGD& 88.9\%& 94.0\%\\
		LDP-SGD& 88.9\%& 94.0\%\\
		ADP-SGD& 94.6\%& 95.7\%\\
		\hline
	\end{tabular}
	\caption{本地自适应差分隐私与其他四种基准方案}
	\label{tab1}
\end{table}

首先,对于MNIST数据集,在没有添加差分隐私保护的原始CNN模型上进行梯度下降训练,经过20个训练轮次后模型在训练集上得到的基准准确率为98.6\%,我们的方案(Adaptive Differential Privacy-SGD,ADP-SGD)在训练刚开始的一个轮次,训练集的训练误差下降较慢,这是由于刚开始训练时模型中所有梯度的归因分数较高,导致对于每个梯度分配的隐私预算较高,加躁后与原始值差异较大。然而,在20个轮次过后,模型收敛的速度远远超过其他三个基准差分隐私方案。在50个训练轮次之后,训练集的损失率降低至25\%以下,而DS-SGD和DP-SGD的训练损失值在10个轮次之后仅降低至35\%左右,DP-SGD更次之,在40\%左右。图\ref{fig:不同隐私保护方案在MNIST数据集上训练的测试误差变化情况}中的(c1)和(c2)展示了模型训练中梯度的下降情况,在相同的隐私预算下,ADP-SGD能在2个训练轮次,梯度范数接近0.01且趋于平稳,意味着模型趋近收敛的速度最快。

其次,对于CIFAR-10数据集,由于图像本身复杂度的增加,在没有添加差分隐私保护的原始模型上进行梯度下降训练,经过20个训练轮次后模型在训练集上得到的基准准确率为97\%。LDP在第一个训练轮次的模型收敛速率最高,然而在第二个训练轮次过后,ADP-SGD达到35\%的模型准确率和0.05左右的梯度范数,训练数据损失值和梯度范数均比另外三个基准方案的效果优。

综上,我们的方案在调整梯度自适应加躁和自适应裁剪后使得模型收敛率和准确率大大提升,无论是在模型的准确率还是收敛速度方面都更加接近原始无隐私保护的模型。

\subsubsection{实验三(针对攻击模型,分析该方案的隐私保护效用)} 
我们曾在第一章介绍了针对联邦学习模型的隐私攻击,其中成员推理攻击是最流行的一类攻击,旨在确定一个输入样本x是否存在于模型训练集D中。该攻击只攻击者假设知道模型的输出预测向量(黑盒),并且是针对有监督的机器学习模型进行的。为了推断成员属性,对抗者使用与目标模型的相同算法训练了很多影子模型。对于每个影子模型,对手随机选择一些数据样本来形成一个训练集和一个验证集。影子模型是在训练集上训练的。然后,他将训练特征数据和验证特征数据都送入影子模型,并获得相应的结果,即每个类别的概率。所以对手可以建立一个分类器,以特征数据的预测结果为输入,以数据是来自训练集(表示为1)还是验证集(表示为0)为标签。有了这个分类器,敌手将考虑为敏感样本的特征值发送到目标模型并检索预测结果。然后,他将预测结果发送给分类器,并得到敏感样本是否是目标模型训练集的成员的结论。

我们重现了文献\upcite{ref70}中的成员推理攻击算法,配置和参数与之相同:目标模型是卷积神经网络,数据集为CIFAR-10。我们为目标模型和影子模型选择了两种规模的训练集:2500和10000,验证集与训练集的大小相同。敌手用不同的训练和验证样本子集运行100个影子模型,使用逻辑回归算法根据预测结果对样本是否包含在训练集中进行分类。

该实验的评估指标为模型的隐私保护效用,因此我们对比了在自适应差分隐私和无隐私保护的模型上进行成员推理攻击的攻击准确率,结果如图\ref{fig:在不同模型上进行成员推理攻击的准确率}所示。

\begin{figure}[!hbt]
\centering
	\includegraphics[scale=0.5]{fig2/C3/第三章实验三}%联邦学习的系统架构
	\caption{在不同模型上进行成员推理攻击的准确率}
	\label{fig:在不同模型上进行成员推理攻击的准确率}	
\end{figure}

我们分别在10k和2.5k个数据样本上进行攻击实验,图\ref{fig:在不同模型上进行成员推理攻击的准确率}中的(a)表示在10k数据样本上进行攻击实验,x轴表示对抗攻击的类别,蓝线(original model)表示在原始无隐私保护的模型上进行成员推理攻击的各个类别的攻击准确率,在不同类别上基本都可达到80\%~90\%的准确率。我们可以看到,敌手对识别包含在训练集中的样本有很高的置信值。当样本不在训练集中时,错误率相对较高。图\ref{fig:在不同模型上进行成员推理攻击的准确率}中的红线(ADP-SGD)显示了在添加了自适应差分隐私的模型上进行成员推理攻击的准确性。基准线是50\%(黑色虚线),这是敌手通过随机猜测达到的准确率。在应用差分隐私的情况下,敌手的推断准确率下降到50\%∼65\%,接近于随机猜测。这比原始模型的攻击准确率要低得多。而在2.5k的数据样本上进行攻击,我们的方案针对攻击实验能在两个类别上将准确率降低至50\%左右,证明了本地自适应差分隐私针对成员推理攻击的隐私保护效用。

\section{本章总结}
本章详细介绍了如何在深度学习模型的随机梯度下降算法中添加自适应的高斯噪声以及对梯度进行自适应裁剪。我们设计了一个自适应噪声添加的算法,在神经网络前向传播算法中,根据梯度对于模型输出的贡献率注入不同程度的隐私预算的噪声。与传统的注入噪声的方法相比,本文的方案在联邦学习的架构下避免了由统一本地用户的敏感度而噪声的隐私预算的浪费,在相同的隐私保护程度下最大限度地提高了模型的准确性,并且证明了算法能满足$\epsilon_{c}$-差分隐私。接着,本文设计了一种自适应调整剪裁阈值的方案,通过计算梯度更新的方差和偏差,逐元素地对梯度进行裁剪,与之前的方案相比,通过使用梯度的自适应剪裁实现了相同的隐私保证,而模型的收敛速度大大提升。之后我们利用“Moments Accountant”机制分析加噪累积产生的隐私预算,与传统的注入噪声的方法相比,我们在相同的隐私保护程度下大大减少了噪声对模型输出结果的影响,提高了模型的准确性。
然后,我们在MNIST和CIFAR-10数据集上分别进行实验,评估了方案的隐私保护效用以及对模型性能的影响。

然而,在客户端的本地数据集添加的噪声只能保证本地数据的匿名性,不能够防止外部攻击者针对通信信道的攻击。如果客户端在每次迭代中同时上传了大量的权重更新,中央云服务器仍然可以将它们链接在一起,推导出参数信息。而且,当参与一次迭代的客户端数量达到上千人时,会导致聚合任务升级成一个高维任务,隐私预算暴增。因此,下一章我们对联邦学习模型框架进行了改进,在现有的联邦学习模型上新增混洗器,实现联邦学习框架的隐私安全,提高整体联邦学习模型的精度。




\chapter{联邦学习的安全混洗框架}
\label{ch4}
\section{引言}
上一章节中所提出的本地自适应差分隐私方案是通过在客户端将梯度上传至参数服务器前,对梯度添加自适应噪声,尽管方案采用了本地差分技术减少一定程度的隐私预算,但Truex等人\upcite{ref49}指出的,一个复杂的隐私保护系统将多个本地差分隐私的算法进行组合,会导致这些算法的隐私成本增长。也就是说,隐私预算为ε1和ε2的本地差分算法的组合会消耗的隐私预算总和为ε1+ε2。使用联邦学习训练的联合模型需要客户在多次迭代中向中央服务器上传梯度更新。如果在迭代训练过程中的每一次迭代都应用本地差分隐私,隐私预算就会累积起来,从而导致总隐私预算的爆炸。现有的本地差分隐私协议对于多维聚集的联邦学习框架可能是不可行的,局部噪声带来的误差会随着维度系数的增加而加剧,从而大大降低模型的精度。而且,当参与一次迭代的客户端数量达到上千人时,会导致聚合任务升级成一个高维任务,隐私预算暴增\upcite{ref43}。

在联邦学习的背景下,传统的经验风险最小化问题存在以下条着:(i)需要为客户的数据提供隐私保证,(ii)压缩客户和服务器之间的通信,因为客户提供的链接可能为低带宽,(iii)在服务器和客户之间的每一轮通信中与动态客户群合作,因为每一轮都有一小部分客户被采样。为了应对这些挑战,我们设计了安全混洗算法使ERM的优化方案在每一轮通信会和都能有效地进行梯度聚合。

在最近的研究工作中,人们提出了一个新的隐私框架,使用匿名化的方式上传模型参数到中央服务器,即所谓的shuffle模型。这种
该模型通过隐私放大效应(相比于本地差分隐私算法与客户端数量成比例$\frac{1}{\sqrt{m}}$的放大),使隐私-效用性能显著提高。另一个中隐私放大的机制是通过随机采样,在本地随机梯度下降算法中,通过客户对本地数据的小批量抽样和每个通信回合中混洗器对客户本身的抽样。

在本文,我们将两种隐私放大效应相结合,设计了一个全新的安全混洗器,与本地自适应差分隐私相结合,实现的方案能提高全局模型的精度,也保证在更低的隐私成本下达到相同的隐私预算。本地客户端使用自适应差分隐私在模型训练的梯度下降算法过程中加躁,然后安全混洗器从客户端上传的样本中随机采样,将收集到的梯度以维度进行拆分,打乱次序,达到隐私放大效果,再发送给中央服务器进行聚合。安全混洗器独立于服务器并专门用于本地客户端梯度的子采样、拆分混洗、上传。这个模型通过子采样和拆分混洗两者的结合达到隐私放大效应,降低了隐私预算,从而提高了整体联邦学习模型的精度。当本地差分隐私添加更少的噪音时,对于同样的中央服务器能达到相同水平的隐私预算,但通信成本要低得多。

我们将在本章节详细的描述该框架中各个模块的设计和实现过程。

\section{模型设计}
在安全混洗模型中,我们假设敌手为恶意的第三方服务器和中央服务器,因为它们持有用户本地梯度的所有加密版本。在我们的威胁模型中,我们假设这两种服务器是诚实而好奇的,这意味着每个服务器都诚实地遵守预先商定的程序来完成其任务。然而,它也可能试图通过利用掌握的先验知识来损害用户的数据隐私。此外,我们假设第三方服务器和中央服务器之间不存在串通。

在上述威胁模型下,我们将隐私要求表述如下:
\begin{itemize}
  \item 用户的本地梯度的保密性:敌手如云服务器,可以通过利用共享梯度和全局参数来恢复用户的敏感信息,如数据标签和成员信息。为了保护用户的隐私,每个用户的本地梯度在被发送到服务器之前应该通过差分隐私加躁。
  \item 对用户的可靠性和聚合结果进行隐私保护:为了使学习过程公平和非歧视性,每个用户的可靠性,即用户的“数据质量”信息,应该被保密,在训练过程中不能被服务器和任何用户获取。另一方面,模型聚合的结果可以被视为有价值的知识产权,它是用大量的资源产生的,甚至包含一些用户的专有信息。因此,除了参与训练的用户之外,聚合的结果对敌手来说应该是保密的。
\end{itemize}

\subsection{模型概览}
\begin{figure}[!hbt]
\centering
	\includegraphics[scale=0.5]{fig2/C4/shuffle1}%联邦学习的系统架构
	\caption{联邦学习安全模型框架}
	\label{fig:联邦学习安全模型框架}	
\end{figure}

如图\ref{fig:联邦学习安全模型框架}所示,该框架主要由本地客户端、混洗器和中央服务器3部分组成:
\begin{itemize}
  \item 本地客户端:基于第三章的本地自适应差分隐私方案,在模型训练的梯度下降算法中对梯度进行自适应的扰动,得到满足$\left(\epsilon_{c}+\epsilon_{l}\right)$-差分隐私的梯度。
  \item 混洗器:首先动态采样本地客户端上传的梯度,然后利用本文提出的安全混洗协议在对数据一无所知的情况下,对子采样后的梯度完成安全的拆分混洗操作,通过隐私放大效应使得算法满足$\epsilon_{0}$-差分隐私,达到梯度匿名机制,最后将混洗后的结果发送至中央服务器。
  \item 中央服务器: 一个诚实但好奇的第三方。服务器接受混洗器上传的梯度并进行聚合,然后更新全局模型。
\end{itemize}


假设现在有m个本地客户端,每个客户端表示为$i \in[m]$,有本地数据集\\$\mathcal{D}_{i}=\left\{d_{i 1}, \ldots, d_{i r}\right\} \in \mathbb{S}^{r}$,由$r$ 个数据集合构成。$F_{i}(\theta)$表示在客户端$i$的本地数据集 $\mathcal{D}_{i}$上进行训练,对于模型梯度$\theta \in \mathbb{R}^{d}$进行衡量的损失函数,其中$F_{i}(\theta)=\frac{1}{r} \sum_{j=1}^{r} f\left(\theta ; d_{i j}\right)$,$f(\theta ; \cdot): \mathcal{C} \rightarrow \mathbb{R}$是凸函数。中央服务器的目标是找到一个最佳的模型参数向量$\theta^{*} \in \mathcal{C}$ 使得损失函数$\min _{\theta \in \mathcal{C}}\left(F(\theta)=\frac{1}{m} \sum_{i=1}^{m} F_{i}(\theta)\right)$最小,其中隐私性满足单个客户端的隐私预算,也就是满足$\epsilon_{l}$-差分隐私。

在算法\ref{联邦学习中的安全模型算法}中,首先我们从m个客户端中随机挑选k个客户端,表示为集合$\mathcal{U}_{t}$,其中$k \leq m$。每个客户端$i \in \mathcal{U}_{t}$从本地数据集中抽样$\mathcal{S}_{i t}$个样本训练模型,计算梯度$\nabla_{\theta_{t}} f\left(\theta_{t} ; d_{i j}\right)$。第$i$个客户端采用基于第三章的自适应噪声添加方案,添加噪声、裁剪梯度,然后将加躁后的梯度 $\left\{\mathcal{R}_{p}\left(\mathbf{g}_{t}\left(d_{i j}\right)\right)\right\}_{j \in \mathcal{S}_{i t}}$发送给混洗器。混洗器对收到的梯度 $k s$进行拆分混洗(梯度维度的拆分、输出随机排序的结果),然后发送给中央服务器。最后,中央服务器对混洗后的梯度进行聚合求均值,更新全局模型。


\begin{algorithm}[!htb]
	\caption{联邦学习中的安全模型算法:$\mathcal{A}_{\text {csdp}}$}
	\label{联邦学习中的安全模型算法}
	\begin{algorithmic}[1]
		\footnotesize
		\STATE \textbf{输入:} 数据集$\mathcal{D} \quad=\bigcup_{i \in[m]} \mathcal{D}_{i}$, $\mathcal{D}_{i}=\left\{d_{i 1}, \ldots, d_{i r}\right\}$,损失函数 $F(\theta)=$ $\frac{1}{m r} \sum_{i=1}^{m} \sum_{j=1}^{r} f\left(\theta ; d_{i j}\right)$,本地差分隐私预算$\epsilon_{0}$,梯度范数阈值$C$,模型学习率$\eta_{t}$
		\STATE \textbf{初始化:} $\theta_{0} \in \mathcal{C}$
		\FOR{$t \in[T]$}
			\STATE \textbf{客户端采样:} 混洗器从k个客户端中随机采样$i \in \mathcal{U}_{t}$个客户端
			\FOR{客户端$i \in \mathcal{U}_{t}$}
				\STATE \textbf{梯度选择:} 客户端i从s个样本空间中随机采样$\mathcal{S}_{i t}$个梯度
				\FOR{样本$j \in \mathcal{S}_{i t}$}
					\STATE $\mathbf{g}_{t}\left(d_{i j}\right) \leftarrow \nabla_{\theta_{t}} f\left(\theta_{t} ; d_{i j}\right)$
					\STATE ${\mathbf{g}}_{t}\left(d_{i j}\right) \leftarrow \mathbf{g}_{t}\left(d_{i j}\right) / \max \left\{1, \frac{\left\|\mathbf{g}_{t}\left(d_{i j}\right)\right\|_{p}}{C}\right\}^{3}$
					\STATE $\mathbf{q}_{t}\left(d_{i j}\right) \leftarrow \mathcal{R}_{p}\left(\tilde{\mathbf{g}}_{t}\left(d_{i j}\right)\right)$
				\ENDFOR
				\STATE 客户端i将$\left\{\mathbf{q}_{t}\left(d_{i j}\right)\right\}_{j \in \mathcal{S}_{i t}}$发送给混洗器
			\ENDFOR
			\STATE \textbf{混洗器:} 混洗器对于$\left\{\boldsymbol{q}_{t}\left(d_{i j}\right): i \in \mathcal{U}_{t}, j \in \mathcal{S}_{i t}\right\}$中的权重进行拆分混洗,然后上传给中央服务器
			\STATE \textbf{中央服务器聚合梯度:}$\overline{\mathbf{g}}_{t} \leftarrow \frac{1}{k s} \sum_{i \in \mathcal{U}_{t}, j \in \mathcal{S}_{i t}} \boldsymbol{q}_{t}\left(d_{i j}\right)$
			\STATE \textbf{梯度下降:}$\theta_{t+1} \leftarrow \prod_{\mathcal{C}}\left(\theta_{t}-\eta_{t} \overline{\mathbf{g}}_{t}\right)$
		\ENDFOR
		\STATE \textbf{输出:}最终全局模型参数$\theta_{T}$

	\end{algorithmic}
\end{algorithm}

\newpage

\subsection{梯度采样}
假设在空间$\mathcal{U}$中我们有一个数据集$\mathcal{D}^{\prime}=\left\{U_{1}, \ldots, U_{r_{1}}\right\} \in \mathcal{U}^{r_{1}}$,其中包含$r_{1}$个样本元素。如定义\ref{子采样}所示,本文定义一个子采样程序:首先采样一个客户端数据集$\mathcal{D}^{\prime} \in \mathcal{U}^{r_{1}}$,再从中采样一个子集作为客户端的本地训练数据。
\begin{define}[子采样]\label{子采样}
定义一个抽样程序$\operatorname{samp}_{r_{1}, r_{2}}: \mathcal{U}^{r_{1}} \rightarrow \mathcal{U}^{r_{2}}$,其中$r_{2} \leq r_{1}$:从输入的数据集$\mathcal{D}^{\prime} \in \mathcal{U}^{r_{1}}$ 中以随机概率抽选一个子数据集$\mathcal{D}^{\prime \prime}$,数据集$\mathcal{D}^{\prime}$中的每个元素在数据集$\mathcal{D}^{\prime \prime}$中出现的概率为$q=\frac{r_{2}}{r_{1}}$。
\end{define}

\subsection{混洗器}
McMahan等人先前的研究工作\upcite{ref52}表明,在联邦学习模型中,假如在某个时间段数据是被适当的匿名化,并将数据之间的耦合信息拆分后,模型整体的隐私保障可以得到极{}大的改善。在第三章中的隐私保护方案是基于本地客户端训练数据的,在本地模型上进行差分隐私操作后,整体的模型误差最低也达到了$\Omega(\sqrt{n})$(n代表本地设备数量),然而在中央服务器上对全局参数进行差分隐私的方案可以将误差降低至$O(1)$。

因此在本章中,我们针对客户端上传的梯度,进行参数的拆分混洗,通过混洗器达到客户端的匿名性,打破从中央服务器接收的数据与特定客户端之间的联系,并在每次迭代中从同一客户端发送的梯度更新中将信息解耦。在洗牌模型中,利用一个洗牌器来打破用户身份和上传到数据分析器的信息之间的联系。由于需要引入更少的噪音来实现相同的隐私保证,按照这种模式,保护隐私的数据收集的效用得到了改善。

客户端的匿名性可以通过现有的多种机制来实现,这取决于中央服务器在特定场景下如何跟踪客户端。作为一个典型的保护隐私的最佳做法,如果使每个客户对服务器产生一定程度的匿名性,就能使客户的个人身份识别与他们的权重更新无法关联。例如,如果服务器通过IP地址追踪客户,每个客户可以通过使用网络代理、VPN服务、公共WiFi接入产生一个无法追踪的IP地址。再比如,如果服务器通过软件生成的元数据(如ID)来追踪客户,每个客户可以在向服务器发送元数据之前将其随机化。

但是,我们认为,客户端的匿名性不足以防止通信链道的攻击。例如,如果客户端在每次迭代中同时上传了大量的权重更新,中央服务器仍然可以将它们连接在一起。因此,我们设计了混洗器,以打破来自相同客户的模型权重更新之间的联系,并将其放置于客户端上传梯度更新至中央服务器之间,使中央服务器很难结合多个客户端的同步更新来推断任何本地设备的更多信息,具体算法如\ref{混洗器中的拆分混洗算法}所示。

\begin{algorithm}[!htb]
	\caption{混洗器中的拆分混洗算法}
	\label{混洗器中的拆分混洗算法}
	\begin{algorithmic}[1]
		\footnotesize
		\STATE \textbf{Input:} 本地客户端添加自适应扰动后的权重$W_{l+1}^{s}$
	    \STATE 对权重$W_{l+1}^{s}$进行分割,给每个元素分配id
	    \FOR{$w^{s} \in W$}
	    \STATE 用一个唯一的id标记元素的位置
	    \STATE 在通信时刻(0,$T$)期间随机采样$t_{i d}^{s} \leftarrow U(0, T) \%$
	    \ENDFOR
	    \STATE 在时刻 $t_{i d}^{s}$将梯度$(i d, w_{i d})$发送给中央服务器
	\end{algorithmic}
\end{algorithm}

我们的混洗器通过以下步骤对客户端上传的梯度参数进行混洗,然后上传给中央服务器:
\begin{itemize}
	\item 权重分割:每个客户端都对其本地模型的权重进行分割,但给每个分割后的元素分配一个元数据,以表明其在网络结构中的权重位置。
	\item 权重混洗:对于所有客户端分割后的权重采用随机扰动机制进行混洗。
\end{itemize}

如图\ref{fig:联邦学习安全混洗模型中执行参数拆分混洗的混洗器}所示,假使现有本地模型$X_{1}$,$X_{2}$,$X_{3}$,每个模型都有相同的结构,但权重值不同。原始的联邦学习框架是将模型在本地训练后得到的参数直接发送到中央服务器,如图\ref{fig:联邦学习安全混洗模型中执行参数拆分混洗的混洗器}上半部分所示。

图\ref{fig:联邦学习安全混洗模型中执行参数拆分混洗的混洗器}中的下半部分展示了我们的方案中,首先,对于每个模型,我们分割每个本地模型经过本地训练后所产生的权重。然后,对于每个权重,我们通过随机混洗机制对其进行混洗,并将每个权重及其元数据发送到中央服务器,其中元数据表示该权重值在网络结构中的位置。

\begin{figure}[!hbt]
\centering
	\includegraphics[scale=0.6]{fig2/C4/拆分混洗}%联邦学习的系统架构
	\caption{联邦学习安全混洗模型中执行参数拆分混洗的混洗器}
	\label{fig:联邦学习安全混洗模型中执行参数拆分混洗的混洗器}	
\end{figure}

对于$\epsilon=O(1)$,串行组合的$\epsilon$-LDP算法$\left(A_{1}, \ldots, A_{n}\right)$, let $A_{\text {shuffle }}\left(x_{1}, \ldots, x_{n}\right)=A_{1}\left(x_{\pi(1)}\right), A_{2}\left(x_{\pi(2)}\right), \ldots, A_{n}\left(x_{\pi(n)}\right)$,经过混洗操作$\pi:[n] \rightarrow[n]$后表示为$A_{\text {shuffle }}$,混洗器输出的梯度结果满足$\left(\epsilon^{\prime}, \delta\right)$-DP,其中$\epsilon^{\prime}=O\left(\frac{\epsilon \sqrt{\log (1 / \delta)}}{\sqrt{n}}\right)$。混洗结果并不会改变数据集的统计特性,也不会增加LDP的隐私预算。

\section{隐私性和收敛性证明}
\subsection{隐私性证明}
隐私放大(Privacy Amplification)是本章所提出的安全框架中混洗器对隐私效果增强的理论分析,基于该理论,可将现有的本地化差分隐私方法直接应用在安全框架上。

在算法\ref{联邦学习中的安全模型算法}中,每个本地客户端采用第三章的满足$\left(\epsilon_{c}+\epsilon_{l}\right)$的自适应本地差分隐私算法,将参数上传至混洗器进行拆分混洗后,所获取的数据满足 $\epsilon_{\mathrm{c}}-\mathrm{DP}$。从 $\left(\epsilon_{c}+\epsilon_{l}\right)$到 $\epsilon_{\mathrm{c}}$ 的转变可通过隐私放大理论证明。$\left(\epsilon_{c}+\epsilon_{l}\right)$ 对应于较大的数值, 表示较低的隐私性; $\epsilon_{\mathrm{c}}$ 对应于较小的数值, 表示较高的隐私性。因此经过混洗器后,隐私性得到了增强。由差分隐私的强组合性可保证本地自适应差分隐私算法在每次迭代中对每个样本$d_{i j}$都能保证$\left(\epsilon_{c}+\epsilon_{l}\right)$-本地差异隐私,因此本节只需要分析采样和混洗操作的隐私放大性。

\begin{theorem}\label{隐私性证明}
算法\ref{联邦学习中的安全模型算法}是满足$(\epsilon, \delta)-$差分隐私的,当对于任意$\delta$,$\delta>0$ ,并且有:
$$
\epsilon=\mathcal{O}\left(\epsilon_{0} \sqrt{\frac{q T \log (2 q T / \delta) \log (2 / \delta)}{n}}\right)
$$
\end{theorem}

假设在联邦学习模型中,需要迭代的次数为$t \in[T]$。$\mathcal{M}_{t}\left(\theta_{t}, \mathcal{D}\right)$表示在时刻$t$对于数据集$\mathcal{D}$和模型参数为$\theta_{t}$的差分隐私机制,$\theta_{t+1}$表示模型的输出。因此,在数据集$\mathcal{D}=\bigcup_{i=1}^{m} \mathcal{D}_{i} \in \mathfrak{S}^{n}$上的差分隐私机制定义如下:

\begin{equation}\label{eq:隐私性证明机制}
\mathcal{M}_{t}\left(\theta_{t} ; \mathcal{D}\right)=\mathcal{H}_{k s} \circ \operatorname{samp}_{m, k}\left(\mathcal{G}_{1}, \ldots, \mathcal{G}_{m}\right)
\end{equation}

其中,$\mathcal{G}_{i}=\operatorname{samp}_{r, s}\left(\mathcal{R}\left(\boldsymbol{x}_{i 1}^{t}\right), \ldots, \mathcal{R}\left(\boldsymbol{x}_{i r}^{t}\right)\right)$并且$\boldsymbol{x}_{i j}^{t}=$$\nabla_{\theta_{t}} f\left(\theta_{t} ; d_{i j}\right), \forall i \in[m], j \in[r]$。$\mathcal{H}_{k s}$表示在$k s$个数据样本上进行混洗操作, $\operatorname{samp}_{a, b}$表示从有a个元素的集合中随机抽样b个元素的操作。

接下来我们给出$\mathcal{M}_{t}$的隐私性证明:

假设客户端$i \in[m]$的本地数据集为$\mathcal{D}_{i}=\left\{d_{i 1}, d_{i 2}, \ldots, d_{i r}\right\} \in \mathfrak{S}^{r}$,$\mathcal{D}=\bigcup_{i=1}^{m} \mathcal{D}_{i}$表示总体数据集。根据公式\ref{eq:隐私性证明机制},$\mathcal{Z}\left(\mathcal{D}^{(t)}\right)=\mathcal{H}_{k s}\left(\mathcal{R}\left(\boldsymbol{x}_{1}^{t}\right), \ldots, \mathcal{R}\left(\boldsymbol{x}_{k s}^{t}\right)\right)$表示在本地客户端进行本地差分隐私后输出的$ks$个权重集合上进行混洗后的权重。任取$\tilde{\delta}>0$,当$\epsilon_{0} \leq \frac{\log (k s / \log (1 / \tilde{\delta}))}{2}$时,算法$\mathcal{Z}$ 满足 $(\tilde{\epsilon}, \tilde{\delta})-\mathrm{DP}$差分隐私,可得:

\begin{equation}\label{eq:隐私性证明机制2}
\tilde{\epsilon}=\mathcal{O}\left(\min \left\{\epsilon_{0}, 1\right\} e^{\epsilon_{0}} \sqrt{\frac{\log (1 / \tilde{\delta})}{k s}}\right)
\end{equation}

当$\epsilon_{0}=\mathcal{O}(1)$时,有$\tilde{\epsilon}=\mathcal{O}\left(\epsilon_{0} \sqrt{\frac{\log (1 / \tilde{\delta})}{k s}}\right)$。

令$\mathcal{T} \subseteq\{1, \ldots, m\}$表示在时刻$t$选取的k个客户端。对于$i \in \mathcal{T}$,$\mathcal{T}_{i} \subseteq\{1, \ldots, r\}$表示在时刻$t$客户端$i$所抽样的$s$条数据样本。对于任意的 $\mathcal{T} \in\left(\begin{array}{c}{[m]} \\ k\end{array}\right)$和$\mathcal{T}_{i} \in\left(\begin{array}{c}{[r]} \\ s\end{array}\right), i \in \mathcal{T}$,有$\overline{\mathcal{T}}=\left(\mathcal{T}, \mathcal{T}_{i}, i \in \mathcal{T}\right), \mathcal{D}^{\mathcal{T}_{i}}=\left\{d_{j}: j \in \mathcal{T}_{i}\right\}$ for $i \in \mathcal{T}$, and $\mathcal{D}^{\bar{\top}}=\left\{\mathcal{D}^{\mathcal{T}_{i}}: i \in \mathcal{T}\right\}$。$\mathcal{T}$和$\mathcal{T}_{i}, i \in \mathcal{T}$为抽样产生的任意子集,其中的随机性由客户端抽样和数据集抽样所决定。算法$\mathcal{M}_{t}$可以等价的表示为$\mathcal{M}_{t}=\mathcal{Z}\left(\mathcal{D}^{\overline{\mathcal{T}}}\right)$。

假设现有数据集:$\mathcal{D}^{\prime}=\left(\mathcal{D}_{1}^{\prime}\right) \bigcup\left(\cup_{i=2}^{m} \mathcal{D}_{i}\right) \in \mathfrak{S}^{n}$,其中数据集$\mathcal{D}_{1}^{\prime}=\left\{d_{11}^{\prime}, d_{12}, \ldots, d_{1 r}\right\}$和$\mathcal{D}_{1}$ 为相邻数据集,它们的第$d_{11}$条和第$d_{11}^{\prime}$条数据样本不同。如果$\mathcal{M}_{t}$是满足$(\bar{\epsilon}, \bar{\delta})-\mathrm{DP}$差分隐私的,那么对于算法$\mathcal{M}_{t}$所选的任意子集$\mathcal{S}$ 都应该满足:
\begin{equation}\label{eq:隐私性证明3}
\operatorname{Pr}\left[\mathcal{M}_{t}(\mathcal{D}) \in \mathcal{S}\right] \leq e^{\bar{\epsilon}} \operatorname{Pr}\left[\mathcal{M}_{t}\left(\mathcal{D}^{\prime}\right) \in \mathcal{S}\right]+\bar{\delta}
\end{equation}

\begin{equation}\label{eq:隐私性证明4}
\operatorname{Pr}\left[\mathcal{M}_{t}\left(\mathcal{D}^{\prime}\right) \in \mathcal{S}\right] \leq e^{\bar{\epsilon}} \operatorname{Pr}\left[\mathcal{M}_{t}(\mathcal{D}) \in \mathcal{S}\right]+\bar{\delta}
\end{equation}

由于式\ref{eq:隐私性证明3}和\ref{eq:隐私性证明4}是对称的,因此只需要证明其中一条。下文给出式\ref{eq:隐私性证明3}的证明:

令$q=\frac{k s}{m r}$,我们给出条件概率的定义:
\begin{equation}\label{eq:隐私性证明5}
\begin{array}{l}
A_{11}=\operatorname{Pr}\left[\mathcal{Z}\left(\mathcal{D}^{\overline{\mathcal{T}}}\right) \in \mathcal{S} \mid 1 \in \mathcal{T} \text { and } 1 \in \mathcal{T}_{1}\right] \\
A_{11}^{\prime}=\operatorname{Pr}\left[\mathcal{Z}\left(\mathcal{D}^{\prime} \overline{\mathcal{T}}\right) \in \mathcal{S} \mid 1 \in \mathcal{T} \text { and } 1 \in \mathcal{T}_{1}\right] \\
A_{10}=\operatorname{Pr}\left[\mathcal{Z}\left(\mathcal{D}^{\overline{\mathcal{T}}}\right) \in \mathcal{S} \mid 1 \in \mathcal{T} \text { and } 1 \notin \mathcal{T}_{1}\right]=\operatorname{Pr}\left[\mathcal{Z}\left(\mathcal{D}^{\prime \overline{\mathcal{T}}}\right) \in \mathcal{S} \mid 1 \in \mathcal{T} \text { and } 1 \notin \mathcal{T}_{1}\right] \\
A_{0}=\operatorname{Pr}\left[\mathcal{Z}\left(\mathcal{D}^{\bar{T}}\right) \in \mathcal{S} \mid 1 \notin \mathcal{T}\right]=\operatorname{Pr}\left[\mathcal{Z}\left(\mathcal{D}^{\prime \bar{\tau}}\right) \in \mathcal{S} \mid 1 \notin \mathcal{T}\right]
\end{array}
\end{equation}

令$q_{1}=\frac{k}{m}$,$q_{2}=\frac{s}{r}$,那么$q=q_{1} q_{2}$,然后可以得到:
\begin{equation}\label{eq:隐私性证明6}
\begin{aligned} 
\operatorname{Pr}\left[\mathcal{M}_{t}(\mathcal{D}) \in \mathcal{S}\right] &=q A_{11}+q_{1}\left(1-q_{2}\right) A_{10}+\left(1-q_{1}\right) A_{0}
\end{aligned}
\end{equation}

\begin{equation}\label{eq:隐私性证明7}
\begin{aligned} 
\operatorname{Pr}\left[\mathcal{M}_{t}\left(\mathcal{D}^{\prime}\right) \in \mathcal{S}\right] &=q A_{11}^{\prime}+q_{1}\left(1-q_{2}\right) A_{10}+\left(1-q_{1}\right) A_{0} 
\end{aligned}
\end{equation}

因此,我们可以得到:
\begin{equation}\label{隐私性证明8}
A_{11} \leq e^{\tilde{\epsilon}} A_{11}^{\prime}+\tilde{\delta}
\end{equation}

\begin{equation}\label{隐私性证明9}
A_{11} \leq e^{\tilde{\epsilon}} A_{10}+\tilde{\delta}
\end{equation}
式\ref{eq:隐私性证明7}成立,因此混洗器$\mathcal{M}_{t}$是满足$\varepsilon_{\mathrm{c}}$-差分隐私的。

\subsection{模型收敛性分析}
在本节中,我们分析采用采样和混洗算法后模型的收敛性。

回顾第二章的基础知识,在随机梯度下降算法的每次迭代中,中央服务器将当前的参数向量发送给所有本地客户端,客户端收到后在本地数据集上进行模型训练,计算本地模型的梯度并上传给中央服务器,然后中央服务器计算收到的梯度的平均值/平均数并更新全局模型。

在算法\ref{联邦学习中的安全模型算法}中,在每一轮迭代过程中,中央服务器聚合上传的$ks$个加躁后的梯度,如算法\ref{联邦学习中的安全模型算法}的第15行所示,中央服务器进行聚合后得到结果:$\overline{\mathbf{g}}_{t} \leftarrow \frac{1}{k s} \sum_{i \in \mathcal{U}_{t}, j \in \mathcal{S}_{i t}} \boldsymbol{q}_{t}\left(d_{i j}\right)$,然后通过随机梯度下降算法更新全局模型参数:$\theta_{t+1} \leftarrow \prod_{\mathcal{C}}\left(\theta_{t}-\eta_{t} \overline{\mathbf{g}}_{t}\right)$。其中,$\mathbf{q}_{t}\left(d_{i j}\right)=\mathcal{R}_{p}\left(\nabla_{\theta_{t}} f\left(\theta_{t} ; d_{i j}\right)\right)$。

既然随机机制$\mathcal{R}_{p}$是无偏的,那么平均梯度$\overline{\mathbf{g}}_{t}$也是无偏的,也就是说,我们有 $\mathbb{E}\left[\overline{\mathbf{g}}_{t}\right]=\nabla_{\theta_{t}} F\left(\theta_{t}\right)$,其中期望是相对于客户端和数据点的随机抽样以及机制$\mathcal{R}_{p}$的随机性而言的。

令$F(\theta)$为凸函数,考虑这样一个随机梯度下降算法:$\theta_{t+1} \leftarrow \prod_{\mathcal{C}}\left(\theta_{t}-\eta_{t} \mathbf{g}_{t}\right)$,$\mathbf{g}_{t}$满足$\mathbb{E}\left[\mathbf{g}_{t}\right]=\nabla_{\theta_{t}} F\left(\theta_{t}\right)$并且$\mathbb{E}\left\|\mathbf{g}_{t}\right\|_{2}^{2} \leq G^{2}$。当确定$\eta_{t}=\frac{D}{G \sqrt{t}}$,可以得到:
\begin{equation}\label{eq:模型收敛性证明1}
\mathbb{E}\left[F\left(\theta_{T}\right)\right]-F\left(\theta^{*}\right) \leq 2 D G \frac{2+\log (T)}{\sqrt{T}}=\mathcal{O}\left(D G \frac{\log (T)}{\sqrt{T}}\right)
\end{equation} 

由Nesterov等人在文献\upcite{ref50}中的证明可知,算法\ref{联邦学习中的安全模型算法}的输出$\theta_{T}$满足:
\begin{equation}\label{eq:模型收敛性证明2}
\mathbb{E}\left[F\left(\theta_{T}\right)\right]-F\left(\theta^{*}\right) \leq \mathcal{O}\left(\frac{L D \log (T) \max \left\{d^{\frac{1}{2}-\frac{1}{p}}, 1\right\}}{\sqrt{T}}\left(1+\sqrt{\frac{c d}{q n}}\left(\frac{e^{\epsilon_{0}}+1}{e^{\epsilon_{0}}-1}\right)\right)\right)
\end{equation}

其中,存在$\sqrt{1+\frac{c d}{q n}\left(\frac{e^{\epsilon_{0}}+1}{e^{\epsilon_{0}}-1}\right)^{2}} \leq\left(1+\sqrt{\frac{c d}{q n}}\left(\frac{e^{\epsilon_{0}}+1}{e^{\epsilon_{0}-1}}\right)\right)$。

当$\sqrt{\frac{c d}{q n}}\left(\frac{e^{\epsilon_{0}}+1}{e^{\epsilon_{0}}-1}\right) \geq \Omega(1)$时,可以推导出:
\begin{equation}\label{eq:模型收敛性证明3}
\mathbb{E}\left[F\left(\theta_{T}\right)\right]-F\left(\theta^{*}\right) \leq \mathcal{O}\left(\frac{L D \log (T) \max \left\{d^{\frac{1}{2}-\frac{1}{p}}, 1\right\}}{\sqrt{T}} \sqrt{\frac{c d}{q n}}\left(\frac{e^{\epsilon_{0}}+1}{e^{\epsilon_{0}}-1}\right)\right)
\end{equation}

如果我们在算法\ref{联邦学习中的安全模型算法}中设置学习率为$\eta_{t}=\frac{D}{G \sqrt{t}}$,其中\\$G^{2}=$ $L^{2} \max \left\{d^{1-\frac{2}{p}}, 1\right\}\left(1+\frac{c d}{q n}\left(\frac{e^{\epsilon_{0}}+1}{e^{\epsilon_{0}-1}}\right)^{2}\right)$。那么:

\begin{equation}\label{eq:模型收敛性证明4}
\mathbb{E}\left[F\left(\theta_{T}\right)\right]-F\left(\theta^{*}\right) \leq \\
\mathcal{O}\left(\frac{L D \log (T) \max \left\{d^{\frac{1}{2}-\frac{1}{p}}, 1\right\}}{\sqrt{T}} \sqrt{\frac{c d}{q n}}\left(\frac{e^{\epsilon_{0}}+1}{e^{\epsilon_{0}}-1}\right)\right)
\end{equation}

其中,当$p \in\{1, \infty\}$时,$c=4$否则$c=14$。

\begin{theorem}[随机梯度下降算法的收敛性]\label{随机梯度下降算法的收敛性}
假使有凸函数$F(\theta)$,数据集$D$的维度为$\mathcal{C}$,在模型训练过程中采用随机梯度下降算法$\theta_{t+1} \leftarrow \prod_{\mathcal{C}}\left(\theta_{t}-\eta_{t} \mathbf{g}_{t}\right)$,其中 $\mathbf{g}_{t}$满足$\mathbb{E}\left[\mathbf{g}_{t}\right]=\nabla_{\theta_{t}} F\left(\theta_{t}\right)$并且$\mathbb{E}\left\|\mathbf{g}_{t}\right\|_{2}^{2} \leq G^{2}$。当$\eta_{t}=$ $\frac{D}{G \sqrt{t}}$,$\mathbb{E}\left[F\left(\theta_{T}\right)\right]-F\left(\theta^{*}\right) \leq 2 D G\left(\frac{2+\log (T)}{\sqrt{T}}\right)$成立。
\end{theorem}

根据文献\upcite{ref50}中已有的标准随机梯度下降算法收敛结果中使用的定理\ref{随机梯度下降算法的收敛性}对$G^{2}$的约束条件,证明了混洗算法可在$G^{2}=$ $L^{2} \max \left\{d^{1-\frac{2}{p}}, 1\right\}\left(1+\frac{c d}{q n}\left(\frac{e^{\epsilon_{0}}+1}{e^{\epsilon_{0}-1}}\right)^{2}\right)$时达到全局最优解。

\section{实验评估}
\subsection{实验准备}
在本节中,我们进行实验来评估混洗器的性能。所有的实验都是用PYTHON语言编译的,其中每个用户都由配备6GB内存、四核2.36GHz Cortex A73处理器和四核Cortex A53 1.8GHz处理器的华为nova3安卓手机代替。中央服务器是用两台联想服务器模拟的,这两台服务器有2个英特尔(R)至强(R)E5-2620 2.10GHZ CPU,32GB内存,512SSD,2TB机械硬盘,运行于Ubuntu 18.04操作系统。

在实验过程中,我们选择了深度学习中常用的两个经典数据集--MNIST手写体数字识别数据集、FMNIST和CIFAR-10数据集进行实验,评估所提出的安全聚合框架。此外,我们让所有用户离线训练一个统一的卷积神经网络,以获得本地用户的梯度。在我们的实验中采用的模型网络结构为CNN,包括2个卷积层,两个池化层层和一个全连接层(32个神经元)。模型的激活函数为Softmax,并引入了DropOut正则以提高模型的泛化能力。下表展示了CNN的网络结构。

\begin{table}[H]
	\centering
	\begin{tabular}{cc}
		\hline
		神经层& 参数\\
		\hline
		卷积层& 8 × 8的16个滤波器,步长为2\\
		池化层& 2 × 2\\
		卷积层& 4 × 4的32个滤波器,步长为2\\
		池化层& 2 × 2\\
		全连接层& 32个神经元\\
		Softmax& 10个神经元\\
		\hline
	\end{tabular}
	\caption{安全混洗框架实验的模型网络结构}
	\label{tab1}
\end{table}

\subsection{实验设计}
在我们模拟的联邦学习环境中,我们设置本地客户端的总数为60000个,其中每个客户有一个本地数据集。在SA-FL的每一轮通信回合,我们随机选择10000个客户。每个客户都对梯度$\mathbf{g}_{t}\left(d_{i j}\right) \leftarrow \nabla_{\theta_{t}} f\left(\theta_{t} ; d_{i j}\right)$进行剪裁,剪裁参数C=1/100。之后,在梯度上添加自适应噪声。我们的算法运行了80个历时,在前70个历时中,我们将学习率设置为0.3,在剩余的历时中,将其降低到0.18。我们设置了本地隐私参数$\sigma$=2,而中央隐私参数$\epsilon$的计算则是由我们来完成的。我们首先使用文献中\upcite{ref72}的定理5.3通过洗牌数值计算隐私放大率。然后,我们通过\ref{隐私性证明}中提出的子抽样计算隐私放大;最后,我们使用差分隐私的强组合性质来获得中央隐私参数$\epsilon$。

我们的实验主要分为两个部分:
\begin{enumerate}
\item [(1)] 在MNIST、FMNIST和CIFAR上评估所提出的安全聚合框架,评估参数:客户端数量$n$、客户端采样比$f_{r}$和通信回合$m$对于隐私预算和模型预测准确率的影响。
\item [(2)] 将本文的安全混洗方案与基准方案、前人提出的Shuffle方案(如表所示)进行对比,评估指标为模型分类准确率和隐私预算。
\end{enumerate}

\begin{table}[H]
	\centering
	\begin{tabular}{cc}
		\hline
		基准方案名称& 具体算法\\
		\hline
		FL& 没有添加隐私保护机制的联邦学习模型\\
		PPFL\upcite{ref71}& 通过安全聚合加密方案(SMPC)实现隐私保护的分布式学习框架\\
		DPFL\upcite{ref67}& 通过参数的混洗实现隐私保护的分布式学习框架\\
		\hline
	\end{tabular}
	\caption{安全混洗框架的比较方案}
	\label{tab1}
\end{table}

\subsection{结果分析}

如图\ref{fig:安全混洗模型中参与混洗的本地客户端数量对联合模型精度的影响}所示,通过客户端采样机制和梯度的拆分混洗算法,我们的安全混洗模型(下文简称SA-FL)能够以较低的隐私成本实现较高的准确性。在训练中增加客户数量n的同时,SA-FL能达到的模型精度与不添加噪声的联邦学习几乎接近。与MNIST(n=100,ε=1)、FMNIST(n=200,ε=5)相比,CIFAR-10(n=500,ε=10)需要更多的客户端,这表明对于一个具有较大神经网络模型的更复杂的任务,当在更多的本地数据和更多的客户端上添加扰动之后,需要更多的通信回合才能使联合模型达到更高的精度。

\begin{figure}[!hbt]
\centering
  	\includegraphics[scale=0.37]{fig2/C4/SA-FL1}%联邦学习的系统架构
	\caption{安全混洗模型中参与混洗的本地客户端数量对联合模型精度的影响}
  	\label{fig:安全混洗模型中参与混洗的本地客户端数量对联合模型精度的影响} 
\end{figure}

接着,我们分别在MNIST, FMNIST和CIFAR-10数据集上评估了客户端采样比$f_{r}$和通信回合$m$对于模型训练准确率的影响。由图\ref{fig:安全混洗模型中通信轮数和客户端采样比对联合模型精度的影响}可以发现,当$f_{r}$太小的时候,并不影响在MNIST上的表现,但对FASHION-MNIST和CIFAR-10的表现影响很大。当$f_{r}$接近1时,安全聚合框架可以在MNIST、FASHION-MNIST和CIFAR-10上达到与不添加噪声的联邦学习模型几乎相近的性能。另一个重要的参数是中央参数聚合器和本地客户端之间的通信轮次$m$。不难看出,随着通信次数的增加,我们可以通过所提出的模型在所有数据集上训练出更好的模型。然而,由于数据和任务的复杂性,CIFAR-10需要更多的通信回合以获得更好的模型。

\begin{figure}[!hbt]
\centering
  	\includegraphics[scale=0.4]{fig2/C4/SA-FL2}%联邦学习的系统架构
	\caption{安全混洗模型中通信轮数和客户端采样比对联合模型精度的影响}
  	\label{fig:安全混洗模型中通信轮数和客户端采样比对联合模型精度的影响} 
\end{figure}

最后,我们统一比较应用了自适应差分隐私算法和安全混洗器的联邦学习模型与其他联邦学习隐私保护模型,在相同隐私预算参数下训练模型能达到的精度。如图\ref{自适应差分混洗模型和其他联邦学习隐私保护模型的比较}(a-c)中,SA-FL在ε=4和n=100的情况下可以达到96.24$\%$的准确率,在ε=4,n=200的情况下可以达到86.26$\%$的准确率,在ε=10,n=500的情况下,在MNIST,FMNIST和CIFAR-10上可以达到61.4$\%$的准确率。我们的结果与之前的其他工作相比非常有竞争力。Geyer等人\upcite{ref53}首次将差分隐私应用于联邦学习,虽然他们只使用了100个客户端,但在MNIST上,他们只能在(ε,m)=(8,11),(8,54)和(8,412)的情况下达到78$\%$,92$\%$和96$\%$的准确率,其中(ε,m)代表隐私预算和通信回合。Bhowmick等人\upcite{ref54}首次在联合学习中利用本地差分隐私。由于其机制的高变异性,它需要超过200轮的通信回合和更高的隐私预算才能使模型收敛。最近,Truex等人\upcite{ref55}将压缩后的局部差分隐私(α-CLDP)应用到联邦学习中,在FMNIST数据集上获得了86.93$\%$的准确性。然而,α-CLDP需要相对较大的隐私预算ε = α-2c-10ρ(例如,α = 1,c = 1,ρ = 10)来实现模型的收敛,这导致了方案的隐私保证程度太低。与以往的工作相比,我们的方案大大减少了客户端和中央服务器之间需要的通信回合(例如,MNIST为10,FMNIST和CIFAR-10为15就能达到全局模型收敛),这使得整个解决方案在实际场景中更加实用。总的来说,SA-FL在隐私成本、模型精度和通信成本方面都比之前的作品取得了更好的表现。

\begin{figure}[!hbt]
\centering
  	\includegraphics[scale=0.5]{fig2/C4/SA-FL3}%联邦学习的系统架构
	\caption{自适应差分混洗模型和其他联邦学习隐私保护模型的比较}
  	\label{自适应差分混洗模型和其他联邦学习隐私保护模型的比较} 
\end{figure}

联邦学习系统的额外开销主要来自服务器端的预训练过程,以及用户端在开始训练前对权重贡献率的计算和梯度的扰动。我们使用20个通信回合来训练中央服务器的初始化模型,这平均需要68.22秒。在本地模型的训练开始之前,用户需要使用前向传播算法计算权重。这个过程只需要训练神经网络前向传播算法,而不需要训练反向传播来计算损失函数,进行梯度下降,其平均耗时为4.35毫秒。
为了减轻隐私威胁,我们提出的解决方案是向权重、线性变换函数中的原始数据和损失函数的系数注入拉普拉斯噪声。向权重注入噪声的步骤可以与计算权重的贡献率同步进行,这需要额外的2.67毫秒时间。向线性变换中的原始数据和损失函数的系数注入自适应噪声的操作可以在训练前完成。因此,在模型效率方面的提升是非常突出的。

从隐私成本和模型精度的总体上看,混洗差分隐私方法在各统计问题的结果可用性上都有着相比本地化差分隐私方法明显更优的结果。但从通信代价和计算代价的角度分析,安全混洗算法中混洗器的引入,使得用户数据与用户所使用的编码器之间的关联性消失,使得中央服务器的计算代价增大。如何兼顾数据的隐私性、可用性、算法的计算代价和通信代价是后续基于SA-FL框架构建隐私保护方法需加以研究的部分。


\section{本章总结}
本章节我们针对联邦学习模型的整体框架进行了改进,提出了安全混洗模型,在本地客户端和中央服务器之间加设混洗器,通过对本地客户端进行随机抽样,将上传的梯度进行拆分混洗,增加隐私放大效果。然后发送给中央服务器进行聚合。并对方案进行了隐私性证明,表明此安全混洗算法可以保证$\varepsilon_{\mathrm{c}}$差分隐私,然后对此方案在中央服务器上的随机梯度下降算法进行了收敛性的分析,证明在凸函数上,梯度$\mathbf{g}_{t}$满足$\mathbb{E}\left[\mathbf{g}_{t}\right]=\nabla_{\theta_{t}} F\left(\theta_{t}\right)$时模型能达到全局收敛。最后,通过在三种基准数据集上进行对象,证明本章所提出的方案能在保证模型收敛性的情况下,减少隐私预算。



\chapter{实验与评估}
\label{ch5}
之前的的章节中,我们描述了树模型鲁棒性验证框架的设计和实现过程。在本节的内容中,我们选取了一些基准的数据集在该验证框架上进行实验评估。
\section{基准数据集介绍}
我们选用了以下三个数据集评估了我们的树模型鲁棒性验证框架:
\begin{enumerate}
	\item [(1)] 波士顿房价数据集(Boston House Price Dataset)收集了在20 世纪 70 年代中期位于波士顿郊区的房屋价格的中位数,它是用于回归任务的经典数据集。该数据集有506个样本数据,每个样本数据包含了城镇人均犯罪率,高速公路便利指数,住宅的房间数等13个特征及其房屋价格的中位数。
	\item [(2)] 手写体数字识别数据集(MNIST)是用于分类任务的经典数据集,来源于美国国家标准与技术研究所。总共包含了70000个手写数字图像,每个图像的尺寸为28 x 28像素,每个像素点用灰度值表示,灰度值范围为0到255,图像分为10类别,分别代表0-9。
	\item [(3)] FASHION-MNIST 数据集包含了 70000 个不同商品的正面灰度图像,与 MNIST 数据集一样,每个图像的尺寸为28 x 28像素,灰度值范围同样为0到255。所有的图像分为10种类别,如:T恤,牛仔裤,裙子等。虽然数据集格式与 MNIST 相同,但由于图像内容的差别,使得有些模型或者算法在MNIST 和 FASHION-MNIST 的表现会有很大不同。因此对于分类任务,我们在这两个数据集上都进行了实验作为对比。
\end{enumerate}

\section{实验环境与配置}
本文中的所有的实验均在一台装有64位Ubuntu操作系统的主机上进行,所使用机器的CUP型号为Inter Core i7-5960X,主频为4.00GHz, 运行内存大小为32GB和 1T 存储硬盘大小。我们利用sklearn(scikit-learn)来训练实验中所需要的树模型:随机森林模型和GBDT 模型。sklearn 是一种开源的,基于 Python 编程语言的机器学习框架。需要注意的是,本文提出的树模型鲁棒性验证框架,同样适用于其他机器学习框架下树模型的验证(如:Silas\cite{bride2019silas},H2O,Ranger\cite{wright2015ranger} 等)。在对样本数据预处理的部分,我们使用了Pandas,Numpy 等第三方库。

\section{实验结果与分析}

\subsection{随机森林模型的鲁棒性验证与分析}

\textbf{回归模型的验证}

我们在波士顿房价数据集上展开了对随机森林回归模型的实验。在训练阶段,将数据集随机打乱,按照 4:1 的比例划分训练样本集和测试样本集。 随后利用sklearn训练出拥有不同超参数的随机森林回归模型,如:模型学习率为$\{0.1,0.2,0.3\}$,模型中树的深度为$\{5,8,10\}$,模型中树的棵树为$\{5,8,10\}$。训练出来的模型的准确率都在 $93\%$至 $98\%$之间。我们直接选取测试样本集中的样本来进行鲁棒性的验证。按照回归模型的单样本鲁棒性的定义,我们在此数据集下,对所有数据类型为数值型的特征,我们设置其对应的$\epsilon$的值为 3, $\rho$值为5代表5000美元,即在扰动房屋相关特征的情况下,模型对房屋价格的预测结果误差不能超过5000美元。

\begin{figure}[!hbt]
\centering
	\includegraphics[scale=0.65]{fig2/C5/RF_regression_crop.pdf}
	\caption{随机森林回归模型验证结果}
	\label{fig:rf_regression}	
\end{figure}

折线图\ref{fig:rf_regression}为模型学习率为0.3下随机森林回归模型的鲁棒性验证结果。从图中我们可以看出:随着树的棵树的增加,模型的全局鲁棒性在降低而且树的深度越小,模型的鲁棒性越高。

\textbf{分类模型的验证}

对于随机森林的分类模型来说,我们分别在MNIST和FASHION-MNIST两个数据集上进行了实验验证。 对于每个数据集,首先将数据集随机打乱,将其划分为两个子集:80$\%$训练样本集和20$\%$测试样本集。 然后,我们从测试样本集中随机抽取了10个类别的各100个图像,即每个鲁棒性测试样本集的大小为1000。随后利用 sklearn训练出随机森林的分类模型,用于验证。

\begin{figure}[!hbt]
\centering
	\includegraphics[scale=0.7]{fig2/C5/adversarial_example.pdf}
	\caption{对抗性样本图}
	\label{fig:adversarial_example}	
\end{figure}

图\ref{fig:adversarial_example}展示了两个数据集中不满足单样本鲁棒性的测试样本的例子。根据分类模型单样本鲁棒性的定义,我们设置特征扰动范围值$\epsilon=1$,代表了一个灰度值。图中的第一列图像为原始的样本,第二列为第一列相对应的对抗性样本图像,我们在第三列的图中标记出了受到扰动的特征点。第一个示例来源于MNIST数据集,我们可以看出在受到扰动之后,数字“ 8”被模型错误的预测为数字“3”。第二个示例来源于 FASHION-MNIST 数据集,标记类别为“牛仔裤”的商品图像被错误地分类为“裙子”。在以上示例中,如果我们直接去对比第一列的原始图像和第二列的对抗性样本图像, 凭借我们的肉眼,根本无法去找出这两个图像直接的差别(在此结果中,最多只有一个灰度值的差别)。这也反映出我们树模型鲁棒性验证框架的必要性。

\subsection{GBDT模型鲁棒性的验证与分析}
\textbf{回归模型的验证}

与随机森林的回归模型的验证实验一样,我们同样在波士顿房价数据集上进行了实验。对数据集的划分方式,训练参数的设置都与随机森林的回归模型保持一致。唯一不同的是,在 GBDT 模型中,我们需要设置损失函数,在此实验中,我们选择均方损失函数。同样,设置特征扰动范围值$\epsilon$为 3,$\rho$值为5,代表5000美元,即在房屋特征扰动的情况下,此模型对房屋价格的预测结果误差不能超过5000美元。

\begin{figure}[!hbt]
\centering
	\includegraphics[scale=0.65]{fig2/C5/gb/gb_regression.pdf}
	\caption{GBDT回归模型验证结果}
	\label{fig:GBDT回归验证}	
\end{figure}

图\ref{fig:GBDT回归验证}为模型学习率为0.3下GBDT回归模型的鲁棒性验证结果。从图中我们可以看出,与随机森林回归模型一致,随着树的深度和树的棵树的增加,模型的全局鲁棒性在降低。但在增加同样棵树的决策树情况下,GBDT 的回归模型的鲁棒性要比随机森林模型下降的更快。换句话说,随机森林模型鲁棒性的下降趋势较为“平缓”,而GBDT 模型鲁棒性的下降趋势则比较“陡峭”。

\textbf{分类模型的验证}

在 GBDT 分类模型的鲁棒性验证实验中,我们同样基于MNIST和FASHION-MNIST两个数据集上进行了实验验证。数据集的划分方式为:80$\%$训练样本集和20$\%$测试样本集。之后,从测试样本集中随机抽取了10个类别的各100个图像,总的鲁棒性测试样本集的大小为1000。

\begin{figure}[!hbt]
\centering
	\includegraphics[scale=0.7]{fig2/C5/gb/gb_ad.pdf}
	\caption{GBDT验证反例}
	\label{fig:GBDT验证反例}	
\end{figure}

图\ref{fig:GBDT验证反例}显示了 GBDT 分类模型下不满足单样本鲁棒性的测试样本。其特征扰动范围值$\epsilon=3$,即最多扰动3个灰度值。图中的第一列图像为原始的样本,第二列为第一列相对应的对抗性样本图像,第三列的图像标记出了受到扰动的特征点。我们可以看出在图像受到扰动之后,在第一个示例中标记为类别“运动鞋”的商品图像被错误地预测为“凉鞋”。第二个示例中数字“1”被模型错误的预测为数字“8”。在扰动范围设置为 3 个灰度值的情况下,我们依然无法通过肉眼看出原始图像和对抗性样本图像之间的区别。

\subsection{树模型鲁棒性可解释性的实验与分析}
\textbf{鲁棒特征集合}

根据我们给出的鲁棒特征集合定义,我们在随机森林分类模型和 GBDT 分类模型中,进行了相关的实验与分析。根据定义可知,在测试样本满足单样本鲁棒性的情况下,我们可以获取其鲁棒特征集合。因此,我们可以直接在之前分类模型的实验中,获取满足鲁棒性的样本的鲁棒特征集合。

\begin{figure}[!hbt]
\centering
	\includegraphics[scale=0.5]{fig2/C5/robust_feature_set.png}
	\caption{随机森林分类模型鲁棒特征集合}
	\label{fig:robust_feature_set}	
\end{figure}

图\ref{fig:robust_feature_set}显示了在随机森林分类模型下的两个数据集中满足单样本鲁棒性的测试样本的示例。特征扰动范围$\epsilon$设置为3,代表了 3 个灰度值。在受到扰动的情况下,这些样本仍然被模型正确识别。图中的第一列显示了原始的样本,第二列为对应图像的鲁棒特征集合图。我们用红色矩形标记处了存在于该样本鲁棒特征集合中的特征点。根据鲁棒特征集合的性质,我们知道保持红色矩形标记的特征点的像素灰度值不变,在特征扰动距离最大为3个灰度值的情况下任意改变其他特征点的像素灰度值都不能改变模型对该样本的识别结果。为了验证此结论,我们随机的改变不包含于鲁棒特征集合中的特征点的像素灰度值,而保持红色标记点像素灰度值不变,然后让模型去识别将改变后的测试样本之后,去检查预测结果是否发生变化。经过大量的随机测试,以上结论正确。

\begin{figure}[!hbt]
\centering
	\includegraphics[scale=0.5]{fig2/C5/gb/gb_RFS.pdf}
	\caption{GBDT分类模型鲁棒特征集合}
	\label{fig:gb_robust_feature_set}	
\end{figure}

图\ref{fig:gb_robust_feature_set}显示了在GBDT分类模型下的两个数据集中满足单样本鲁棒性的测试样本的示例。特征扰动范围$\epsilon$设置为1。实验过程与随机森林实验保持一致。第一行显示的为数字“0”样本的鲁棒特征集合,第二行显示的为商品“裤子”样本的结果。

\textbf{局部鲁棒特征重要度}

根据我们对局部鲁棒特征重要度的定义,我们首先收集了的在随机森林分类模型测试样本集中所有满足单样本鲁棒性的测试样本,根据算法\ref{LRFI algorithm},我们们可以求得模型不同类别的鲁棒特征重要度。

\begin{figure}[!hbt]
\centering
	\includegraphics[scale=0.7]{fig2/C5/LBTZ.pdf}
	\caption{随机森林分类的局部鲁棒性特征重要度}
	\label{fig:LRFI}	
\end{figure}

我们在分别在 MNIST 和 FAHSION-MNIST 数据集中,各自选择了一个类别来计算局部鲁棒特征重要度。图\ref{fig:LRFI}展示了基于随机森林分类模型的实验结果。左侧的图片为MNIST中数字“ 0”的结果,右侧显示了FASHION-MNIST中的商品“ 运动鞋”的结果。特征点的鲁棒重要度的值越大,则颜色越黄。如果其鲁棒重要度的值为0,则颜色为紫色。我们可以观察到由于特征点的重要度值的不同,显示出了该类别的基本形状。重要度大的值基本分布在该类别的基本形状周围,而分布在该类别的基本形状之上的特征点的鲁棒重要度值都比较低。这为对抗性样本的攻击提供了新的思路:在进行对抗性样本攻击的时候,应该优先选择这些鲁棒重要度值比较高的点,去产生对抗性样本,这样可以提高攻击的成功率与效率。如果从我们实验得出的特征重要度分布的规律来看,应该优先选择分布在该类别基本形状周围的特征点去进行攻击。需要注意的是,除了我们给出的以上两个类别的结果之外,其他类别的鲁棒特征重要度也有类似的分布规律。

\subsection{不同类别鲁棒性的验证与分析}

在其他关于树模型的鲁棒性的验证研究中,对于分类模型的鲁棒性验证都是针对于该模型的整体而言的。但是同一模型的不同的类别的鲁棒性可能会不同。在此种情况下,整体的模型鲁棒性的验证结果,并不能提供具体类别的鲁棒性信息。所以我们设计了实验去研究同一模型下不同类别的鲁棒性的是否会出现差异的问题。与之前的实验设置保持一致,数据集的划分方式为:80$\%$训练样本集和20$\%$测试样本集,从测试样本集中随机抽取了10个类别的各100个图像。训练出的树模型的识别率都在 95$\%$ 至 98$\%$之间,并且各个类别的识别率也基本相同。之后我们分别对不同类别的 100 个样本进行鲁棒性验证。

\begin{figure}[!hbt]
\centering
        \includegraphics[scale=0.65]{fig2/C5/mnist_com.png}
	\caption{MNIST 中不同类别鲁棒性的验证结果}
	\label{fig:MNIST不同}	
\end{figure}

\begin{figure}[!hbt]
\centering
\includegraphics[scale=0.7]{fig2/C5/fashion_com.png}
	\caption{FASHION-MNIST 中不同类别鲁棒性的验证结果}
	\label{fig:Fashion不同}	
\end{figure}



折线图\ref{fig:MNIST不同}和\ref{fig:Fashion不同}展示了两个数据集中的不同类别的鲁棒性验证结果。图\ref{fig:MNIST不同}显示了MNIST数据集的结果。我们可以观察到,存在几个类别(例如,数字“ 0”,数字“ 2”,数字“ 6”,数字“ 8”)的鲁棒性随着树的棵树的增加而略有提高。而数字“ 4”,数字“ 5”,数字“ 7”,数字“ 9”类中的鲁棒性值有很明显的波动。除此以外,数字“ 1”的鲁棒性始终保持在非常低的值。 尽管模型对数字“1”的识别率与其他数字的识别率基本一样,但它们的鲁棒性值却存在着显着差异,差值大约在40$\%$至80$\%$之间。相比之下,FASHION-MNIST(图\ref{fig:Fashion不同})中各个类别的鲁棒性总体上保持稳定,并没有随着树的棵树的增加而出现明显的波动。但是,商品类别为“衬衫”的鲁棒性值相比于其他类别要低一些。在该数据集中,没有类似于数字“1”这种的鲁棒性非常低的类别去影响该模型的总体的鲁棒性值。根据此次的实验结果,我们可知在验证树模型鲁棒性的时候,将注意力集中在整个模型的鲁棒性上是不准确的,对于不同的数据集来说,模型对于不同类别样本的鲁棒性的表现可能会有很大的差别。这给了我们的一个启示,在验证在验证分类模型鲁棒性的时候,验证结果应该细化到不同的类别。这些信息对于模型的使用者来说是非常有用的,可以让他们更加详细的了解到该模型的优缺点,从而增加了模型的可信度。

\subsection{树鲁棒性超参数与鲁棒性关系的验证与分析}

在本文提出的鲁棒性验证框架下,我们进一步研究了树模型中两个重要的训练超参数:树的棵树和树的深度与树模型鲁棒性的关系。我们基于 MNIST 数据集去进行这部分的实验。通过在不同深度,不同树的棵树参数下去训练模型,通过对比其鲁棒性结果来进行研究和分析。
\begin{table*}
\caption{基于MNIST 数据集模型在不同超参数下的鲁棒性验证结果.}
\begin{center}
\begin{tabular}{|c|c|c|c|c|c|c|c|c|c|c|}
\hline 
\multirow{2}{*}{Trees} & \multirow{2}{*}{Depth} & \multirow{2}{*}{Accuracy} & \multicolumn{3}{c|}{$\epsilon=1$} &  \multicolumn{3}{c|}{$\epsilon=3$} \\ \cline{4-9}
& & & Verified($\rho$) & Timeout & Failed & Verified($\rho$) & Timeout & Failed \\
\hline
25 & 5& 84$\%$&45.71$\%$ &0$\%$ &54.29$\%$& 9.64$\%$& 0.12$\%$& 90.24$\%$ \\
\hline
50& 5& 85$\%$&54.68$\%$ &0$\%$ &45.32$\%$ & 13.58$\%$&2.69$\%$ & 83.72$\%$ \\
\hline
75& 5& 86$\%$& 48.77$\%$&5.61$\%$ &45.61$\%$ &9.24$\%$ & 13.68$\%$&77.08$\%$  \\
\hline
100&5 & 86$\%$& 54.07$\%$& 14.53$\%$& 31.40$\%$&13.02$\%$ &21.74$\%$ &65.23$\%$  \\
\hline
25& 8& 91$\%$& 61.47$\%$& 0$\%$& 38.53$\%$&9.88$\%$ &0.11$\%$ &90.01$\%$  \\
\hline
50& 8& 93$\%$& 61.02$\%$&0$\%$ &38.98$\%$ &14.36$\%$ &3.02$\%$ &82.61$\%$  \\
\hline
75& 8& 92$\%$& 63.63$\%$&5.32$\%$ &31.05$\%$ &12.81$\%$ &17.05$\%$ &70.14$\%$  \\
\hline
100& 8& 93$\%$& 63.48$\%$&15.04$\%$ &21.48$\%$ & 16.86$\%$&24.81$\%$ &58.32$\%$  \\
\hline
25&10 & 93$\%$& 64.34$\%$& 0$\%$& 35.66$\%$& 14.18$\%$& 0$\%$&85.82$\%$  \\
\hline
50& 10& 95$\%$& 63.21$\%$&0$\%$ & 36.79$\%$&17.34$\%$ &0.53$\%$ &82.14$\%$  \\
\hline
75& 10& 94$\%$& 75.32$\%$&5.32$\%$ & 19.36$\%$& 13.83$\%$& 4.57$\%$&81.60$\%$  \\
\hline
100& 10& 95$\%$& 66.84$\%$& 8.74$\%$& 24.42$\%$&15.16$\%$ &14.21$\%$ &70.63$\%$  \\
\hline
\end{tabular}
\label{tab1}
\end{center}
\end{table*}

表\ref{tab1}为随机森林分类模型在不同训练参数下的鲁棒性结果,其中 Trees 和 Depth 分别表示模型中树的棵树和树的深度,Acrruracy 表示的是模型的识别率,Verified($\rho$)表示模型的全局鲁棒性,Timeout 表示验证超时,Failed 表示验证失败即不满足单样本鲁棒性所占的百分比。我们可以观察到,在特征扰动值为$\epsilon=1$的情况下,保持相同树的深度,树的棵树并不会对其鲁棒性造成很大的影响,但在之前的对回归模型的实验中,在保持树的深度相同的情况下,树的棵树的增加,会导致其模型鲁棒性的降低。
此外,在保持树的棵树相同的情况下,增加树的深度参数的值,会使模型的鲁棒性少量的增加。与之相反的是,对于其回归模型来说,树的深度的增加,会导致其模型鲁棒性的降低。根据我们的实验结果,在保证模型准别率的情况下,模型的开发人员可以通过调整训练参数来增加其模型的鲁棒性。与$\epsilon=1$时做对比,$\epsilon=3$的情况下,该模型的鲁棒性有了明显的降低,这是显而易见的,因为在特征扰动范围为 3 个灰度值的情况下,会产生更多的对抗性样本使得原始样本的鲁棒性不满足。

还有一点值得我们注意,随着树的棵树和树的深度的值的增加,验证超时的比例也在增加,这揭露了我们验证框架的不足。因为从本质上来说,验证框架的验证能力一定程度取决于 Z3 求解器的求解能力,当树模型规模变大的时候,我们编码形成的 SMT 公式数目也是剧增的,这就导致状态爆炸的问题,从而使得求解器无法求解,导致验证超时,无法确定该样本的鲁棒性是否满足。

\subsection{验证时间的结果与分析}
框架的验证时间同样也是我们需要关注的部分,我们需要保证在一定时间内返回正确的验证结果。于是,我们基于 MNIST 数据集,通过验证不同规模大小的树模型来统计该框架的验证时间。
\begin{figure}[!hbt]
\centering
	\includegraphics[scale=0.7]{fig2/C5/mnist_time.png}
	\caption{单样本验证时间图}
	\label{fig:mnist_time.png}	
\end{figure}

图\ref{fig:mnist_time.png}为在不同规模树模型下验证单个样本所需要的平均时间统计图。我们在测试样本集中随机选择了 100 个样本来进行验证时间的结果统计,结果值为平均值。我们的验证覆盖了规模很小到大规模的树模型,最小规模为深度为 8, 棵树为 25 的树,单个样本的验证时间4s。 而对于树的棵树为 100,深度为64 的大规模的模型,我们框架的验证时间为287s。随着树模型规模的增加,验证时间也在逐渐增加。总体来说,我们框架可以去验证大规模的树模型的鲁棒性,而且验证时间也是可接受的时间范围。但如表\ref{tab1}所显示的,在进行大规模模型的验证的时候,有些样本可能会出现验证超时的情况,这也是我们在未来工作中,需要解决的问题。

\section{本章小结}
在本章中,我们选取了三个基准数据集对本文提出的鲁棒性验证框架进行了一系列的实验来测试其可行性,并且对树模型鲁棒性的可解释性和树模型训练参数与鲁棒性的关系也进行实验和研究。实验结果表明,我们的验证框架可以有效验证随机森林和 GBDT 这两个树模型的重要组成部分。但也存在不足,就是虽然可以验证大规模的树模型,可是某些样本还是会出现验证超时的情况,这将是我们未来工作中的重点。

\chapter{总结与展望}
\label{ch6}
\section{总结}
随着深度学习的兴起,出现了越来越多新的模型和算法,能够更有效的解决各类问题。基于人工智能的产品也在各个领域迎来了一波新的发展热潮,给人民的生活带来了巨大的便利。然而用户在享受深度学习模型带来便利的同时,必须共享自己的数据,随着隐私泄露事件越来越多,数据的安全和隐私问题也逐步引起了人们的关注。 

与此同时,各类智能设备也在不断发展,用户产生的数据也越来越多,智能设备的算力不断增强。用户不愿意向商业公司或商业机构提供个人隐私数据。分布式联邦学习系统解决分布式终端用户在本地更新模型的问题,联邦学习的目标是保障大数据共享信息时的数据安全、保护本地数据和个人隐私,在多计算节点之间高效的训练机器学习模型。 

分布式联邦学习系统得到了广泛的研究和应用,成为传统集中式机器学习方法的一种改进方法。它不是将数据上传到中心服务器进行集中训练,而是参与者在本地进行模型训练并与参数服务器共享模型更新。参数服务器对来自多个参与者的权重进行聚合,并组合创建一个改进的全局模型,这有助于保障用户的数据隐私和降低通信成本。 

本文主要研究针对分布式联邦学习系统的隐私安全问题。通过研究分布式联邦深度学习的系统漏洞,提出了一套针对分布式数据的攻击模型,同时研究分布式联邦系统中针对攻击的隐私安全方案对策。本文的主要工作和贡献如下:
\begin{enumerate}
\item [(1)] 本文提出了一个满足本地差分隐私的分类变换扰动机制.该机制将数值型数据的扰动与分类型数据的扰动进行结合,提高了均值估计的准确性.同时,将该机制用于梯度下降中的每次迭代的梯度扰动,保护了训练过程中用户隐私的同时得到了 1 个较为准确的模型.而且,本文也从本地差分隐私定义的角度,理论证明了提出的方法满足$\mathcal{E}$-本地差分隐私.最后通过多组真实数据集以及合成数据集验证了分类变换扰动机制的性能,证明了其在相同条件下要优于现有的同类方法。

\item [(2)] 本文提出了SA安全混洗框架,混洗差分隐私摒弃了中心化差分隐私下对可信第三方的依赖,即无需任何可信第三方。对用户的原始数据进行统一的扰动处理,提高了隐私性;弥补了中心化差分隐私与本地化差分隐私在可用性上约O( n) 的间隙[9-30],在差分隐私的保证下实现了数据隐私度与可用性之间的更好平衡。
\end{enumerate}

综上所述,本文的研究充分证明了所提出框架的有效性,可以极大的联邦学习模型的隐私性和可用性,从而进一步推进了联邦学习在安全领域的应用和发展。

\section{展望}
在可预见的未来,大规模、大数据、分布式的深度学习将得到快速发展。5G、边缘计算、物联网等技术也将迅速普及。人类将彻底步入人工智能时代。在此我将对我未来的研究做出几点展望:
\begin{enumerate}
\item [(1)] 本文提出的基于解析高斯机制与函数机制的差分隐私深度学习算法是一种基础算法,它可以令学习模型在训练过程中总体隐私不累加。因此后续可以研究其在大型数据集与复杂模型结构中的表现。 
\item [(2)] 现实中,分布式协作学习可能由极多的参与者组成,如百万部手机等。同时分布式协作学习中的每个设备可能计算、通信和存储能力等都有很大不同。因此有关实际应用中的通信、异构问题等也需要进行大量的研究。 
\item[(3)] 分布式协作学习需要一个公平的平台和激励机制,可以在实际应用中明显体现出效果提升,并能够在永久数据记录机制(如区块链等)中留下记录。这样才能促进分布式协作学习的商业化与大规模应用。
\end{enumerate}


%\appendix

\chapter{模型示例}
\label{ch7}
% \section{SCADE文本模型}

\linespread{1}

% \lstinputlisting[language=Caml, caption=SCADE文本模型示例]{fig/7/scade.scade}

\section{NuSMV目标模型}

\lstinputlisting[language=VHDL, caption=NuSMV目标模型示例-子状态机模块]{fig/7/nusmv1.smv}

\lstinputlisting[language=VHDL, caption=NuSMV目标模型示例-监控变量模块]{fig/7/nusmv2.smv}

\lstinputlisting[language=VHDL, caption=NuSMV目标模型示例-自定义函数节点模块]{fig/7/nusmv3.smv}

\lstinputlisting[language=VHDL, caption=NuSMV目标模型示例-顶层主函数模块]{fig/7/nusmv4.smv}






\pagestyle{plain}
\clearpage
\phantomsection
\addcontentsline{toc}{chapter}{参考文献}
\bibliographystyle{gbt7714-2005}
\bibliography{bib/tex}

\begin{thebibliography}{99}

\bibitem{ref1}Pouyanfar S, Sadiq S, Yan Y, et al. A survey on deep learning: Algorithms, techniques, and applications[J]. ACM Computing Surveys (CSUR), 2018, 51(5): 1-36.
\bibitem{ref2}Voigt P, Von dem Bussche A. The eu general data protection regulation (gdpr)[J]. A Practical Guide, 1st Ed., Cham: Springer International Publishing, 2017, 10: 3152676.
\bibitem{ref3}Kendall A, Gal Y, Cipolla R. Multi-task learning using uncertainty to weigh losses for scene geometry and semantics[C]//Proceedings of the IEEE conference on computer vision and pattern recognition. 2018: 7482-7491.
\bibitem{ref4}Hu R, Dollár P, He K, et al. Learning to segment every thing[C]//Proceedings of the IEEE Conference on Computer Vision and Pattern Recognition. 2018: 4233-4241.
\bibitem{ref5}张仕良. 基于深度神经网络的语音识别模型研究[D]. 合肥: 中国科学技术大学, 2017.
\bibitem{ref6}Sardianos C, Tsirakis N, Varlamis I. A survey on the scalability of recommender systems for social networks[M]//Social Networks Science: Design, Implementation, Security, and Challenges. Springer, Cham, 2018: 89-110.
\bibitem{ref7}Shen D, Wu G, Suk H I. Deep learning in medical image analysis[J]. Annual review of biomedical engineering, 2017, 19: 221-248.
\bibitem{ref8}Papernot N, Abadi M, Erlingsson U, et al. Semi-supervised knowledge transfer for deep learning from private training data[J]. arXiv preprint arXiv:1610.05755, 2016.
\bibitem{ref9}Dwork C, McSherry F, Nissim K, et al. Calibrating noise to sensitivity in private data analysis[C]//Theory of cryptography conference. Springer, Berlin, Heidelberg, 2006: 265-284.
\bibitem{ref10}Rivest R L, Adleman L, Dertouzos M L. On data banks and privacy homomorphisms[J]. Foundations of secure computation, 1978, 4(11): 169-180.
\bibitem{ref11}Wu X, Fredrikson M, Jha S, et al. A methodology for formalizing model-inversion attacks[C]//2016 IEEE 29th Computer Security Foundations Symposium (CSF). IEEE, 2016: 355-370.
\bibitem{ref12}Hitaj B, Ateniese G, Perez-Cruz F. Deep models under the GAN: information leakage from collaborative deep learning[C]//Proceedings of the 2017 ACM SIGSAC Conference on Computer and Communications Security. 2017: 603-618.
\bibitem{ref13}Shokri R, Stronati M, Song C, et al. Membership inference attacks against machine learning models[C]//2017 IEEE Symposium on Security and Privacy (SP). IEEE, 2017: 3-18.
\bibitem{ref14}Dwork C. Differential privacy[C]//International Colloquium on Automata, Languages, and Programming. Springer, Berlin, Heidelberg, 2006: 1-12.
\bibitem{ref15}Alfeld S, Zhu X, Barford P. Data poisoning attacks against autoregressive models[C]//Proceedings of the AAAI Conference on Artificial Intelligence. 2016, 30(1).
\bibitem{ref16}Yao A C. Protocols for secure computations[C]//23rd annual symposium on foundations of computer science (sfcs 1982). IEEE, 1982: 160-164.
\bibitem{ref17}Meng X, Bradley J, Yavuz B, et al. Mllib: Machine learning in apache spark[J]. The Journal of Machine Learning Research, 2016, 17(1): 1235-1241.
\bibitem{ref18}Wang X, Han Y, Wang C, et al. In-edge ai: Intelligentizing mobile edge computing, caching and communication by federated learning[J]. IEEE Network, 2019, 33(5): 156-165.
\bibitem{ref19}Li T, Sahu A K, Talwalkar A, et al. Federated learning: Challenges, methods, and future directions[J]. IEEE Signal Processing Magazine, 2020, 37(3): 50-60.
\bibitem{ref20}Tran N H, Bao W, Zomaya A, et al. Federated learning over wireless networks: Optimization model design and analysis[C]//IEEE INFOCOM 2019-IEEE Conference on Computer Communications. IEEE, 2019: 1387-1395.
\bibitem{ref21}McMahan B, Moore E, Ramage D, et al. Communication-efficient learning of deep networks from decentralized data[C]//Artificial intelligence and statistics. PMLR, 2017: 1273-1282.
\bibitem{ref22}Zhu L, Han S. Deep leakage from gradients[M]//Federated learning. Springer, Cham, 2020: 17-31.
\bibitem{ref23}Aono Y, Hayashi T, Wang L, et al. Privacy-preserving deep learning via additively homomorphic encryption[J]. IEEE Transactions on Information Forensics and Security, 2017, 13(5): 1333-1345.
\bibitem{ref24}Ma C, Li J, Ding M, et al. On safeguarding privacy and security in the framework of federated learning[J]. IEEE network, 2020, 34(4): 242-248.
\bibitem{ref25}曹志义, 牛少彰, 张继威. 基于半监督学习生成对抗网络的人脸还原算法研究[J]. 电子与信息学报, 2018, 40(2): 323-330. Distributed differ- ential privacy via shuffling. In Eurocrypt. Springer, 2019.
\bibitem{ref26}Goodfellow I, Pouget-Abadie J, Mirza M, et al. Generative adversarial nets[J]. Advances in neural information processing systems, 2014, 27.
\bibitem{ref27}Radford A, Metz L, Chintala S. Unsupervised representation learning with deep convolutional generative adversarial networks[J]. arXiv preprint arXiv:1511.06434, 2015.
\bibitem{ref28}Salimans T, Goodfellow I, Zaremba W, et al. Improved techniques for training gans[J]. Advances in neural information processing systems, 2016, 29: 2234-2242.
\bibitem{ref29}Goodfellow I, Bengio Y, Courville A. Deep learning[M]. MIT press, 2016.
\bibitem{ref30}Johnson R, Zhang T. Accelerating stochastic gradient descent using predictive variance reduction[J]. Advances in neural information processing systems, 2013, 26: 315-323.
\bibitem{ref31}Zhang T. Solving large scale linear prediction problems using stochastic gradient descent algorithms[C]//Proceedings of the twenty-first international conference on Machine learning. 2004: 116.
\bibitem{ref32}Dwork C, Kenthapadi K, McSherry F, et al. Our data, ourselves: Privacy via distributed noise generation[C]//Annual International Conference on the Theory and Applications of Cryptographic Techniques. Springer, Berlin, Heidelberg, 2006: 486-503.
\bibitem{ref33}McSherry F, Talwar K. Mechanism design via differential privacy[C]//48th Annual IEEE Symposium on Foundations of Computer Science (FOCS'07). IEEE, 2007: 94-103.
\bibitem{ref34}LBengio Y. Learning deep architectures for AI[M]. Now Publishers Inc, 2009.
\bibitem{ref35}Dwork C, Roth A. The algorithmic foundations of differential privacy[J]. Found. Trends Theor. Comput. Sci., 2014, 9(3-4): 211-407.
\bibitem{ref36}Bassily R, Smith A, Thakurta A. Private empirical risk minimization: Efficient algorithms and tight error bounds[C]//2014 IEEE 55th Annual Symposium on Foundations of Computer Science. IEEE, 2014: 464-473.
\bibitem{ref37}Acs G, Melis L, Castelluccia C, et al. Differentially private mixture of generative neural networks[J]. IEEE Transactions on Knowledge and Data Engineering, 2018, 31(6): 1109-1121.
\bibitem{ref38}Su D, Cao J, Li N, et al. Differentially private k-means clustering and a hybrid approach to private optimization[J]. ACM Transactions on Privacy and Security (TOPS), 2017, 20(4): 1-33.
\bibitem{ref39}Salakhutdinov R, Mnih A, Hinton G. Restricted Boltzmann machines for collaborative filtering[C]//Proceedings of the 24th international conference on Machine learning. 2007: 791-798.
\bibitem{ref40}Bassily R, Smith A, Thakurta A. Private empirical risk minimization: Efficient algorithms and tight error bounds[C]//2014 IEEE 55th Annual Symposium on Foundations of Computer Science. IEEE, 2014: 464-473.
\bibitem{ref41}McSherry F D. Privacy integrated queries: an extensible platform for privacy-preserving data analysis[C]//Proceedings of the 2009 ACM SIGMOD International Conference on Management of data. 2009: 19-30.
\bibitem{ref42}Thakurta A G. Differentially private convex optimization for empirical risk minimization and high-dimensional regression[M]. The Pennsylvania State University, 2013.
\bibitem{ref43}Lee J, Kifer D. Concentrated differentially private gradient descent with adaptive per-iteration privacy budget[C]//Proceedings of the 24th ACM SIGKDD International Conference on Knowledge Discovery Data Mining. 2018: 1656-1665.
\bibitem{ref44}Balle B, Wang Y X. Improving the Gaussian mechanism for differential privacy: Analytical calibration and optimal denoising[C]//International Conference on Machine Learning. PMLR, 2018: 394-403.
\bibitem{ref45}Shokri R, Shmatikov V. Privacy-preserving deep learning[C]//Proceedings of the 22nd ACM SIGSAC conference on computer and communications security. 2015: 1310-1321.
\bibitem{ref46}LeCun Y, Bottou L, Bengio Y, et al. Gradient-based learning applied to document recognition[J]. Proceedings of the IEEE, 1998, 86(11): 2278-2324.
\bibitem{ref47}Song S, Chaudhuri K, Sarwate A D. Stochastic gradient descent with differentially private updates[C]//2013 IEEE Global Conference on Signal and Information Processing. IEEE, 2013: 245-248.
\bibitem{ref48}Geyer R C, Klein T, Nabi M. Differentially private federated learning: A client level perspective[J]. arXiv preprint arXiv:1712.07557, 2017.
\bibitem{ref49}Truex S, Baracaldo N, Anwar A, et al. A hybrid approach to privacy-preserving federated learning[C]//Proceedings of the 12th ACM Workshop on Artificial Intelligence and Security. 2019: 1-11.
\bibitem{ref50}Nesterov Y. Introductory lectures on convex optimization: A basic course[M]. Springer Science Business Media, 2003.
\bibitem{ref51}  M, Lantz E, Jha S, et al. Privacy in pharmacogenetics: An end-to-end case study of personalized warfarin dosing[C]//23rd {USENIX} Security Symposium ({USENIX} Security 14). 2014: 17-32.
\bibitem{ref52}McMahan H B, Ramage D, Talwar K, et al. Learning differentially private recurrent language models[J]. arXiv preprint arXiv:1710.06963, 2017.
\bibitem{ref53}Geyer R C, Klein T, Nabi M. Differentially private federated learning: A client level perspective[J]. arXiv preprint arXiv:1712.07557, 2017.
\bibitem{ref54}Bhowmick A, Duchi J, Freudiger J, et al. Protection against reconstruction and its applications in private federated learning[J]. arXiv preprint arXiv:1812.00984, 2018.
\bibitem{ref55}Truex S, Liu L, Chow K H, et al. LDP-Fed: Federated learning with local differential privacy[C]//Proceedings of the Third ACM International Workshop on Edge Systems, Analytics and Networking. 2020: 61-66.
\bibitem{ref56}Comiter M. Attacking artificial intelligence[J]. Belfer Center Paper, 2019: 2019-08.
\bibitem{ref57}Abadi M, Chu A, Goodfellow I, et al. Deep learning with differential privacy[C]//Proceedings of the 2016 ACM SIGSAC conference on computer and communications security. 2016: 308-318.
\bibitem{ref58}Papernot N, Abadi M, Erlingsson U, et al. Semi-supervised knowledge transfer for deep learning from private training data[J]. arXiv preprint arXiv:1610.05755, 2016.
\bibitem{ref59}Xie L, Lin K, Wang S, et al. Differentially private generative adversarial network[J]. arXiv preprint arXiv:1802.06739, 2018.
\bibitem{ref60}Jordon J, Yoon J, Van Der Schaar M. PATE-GAN: Generating synthetic data with differential privacy guarantees[C]//International conference on learning representations. 2018.
\bibitem{ref61}Zhang J, Zheng K, Mou W, et al. Efficient private ERM for smooth objectives[J]. arXiv preprint arXiv:1703.09947, 2017.
\bibitem{ref62}Wang D, Ye M, Xu J. Differentially private empirical risk minimization revisited: Faster and more general[J]. arXiv preprint arXiv:1802.05251, 2018.
\bibitem{ref63}Wang D, Chen C, Xu J. Differentially private empirical risk minimization with non-convex loss functions[C]//International Conference on Machine Learning. PMLR, 2019: 6526-6535.
\bibitem{ref64}Abadi M, Chu A, Goodfellow I, et al. Deep learning with differential privacy[C]//Proceedings of the 2016 ACM SIGSAC conference on computer and communications security. 2016: 308-318.


\end{thebibliography}


\pagestyle{plain}
\clearpage
\phantomsection
\addcontentsline{toc}{chapter}{致谢}
{\fangsong
	\chapter*{致\qquad 谢}\vskip 2mm
	\vspace{-1cm}
		\large{

研究生的学习过程是我人生中重要的一个阶段,期间个人的价值观发生了变化、学会了为人处世之道、专业知识有了更多积累。在毕业论文及各项实验室指标基本完成之时,感想颇多。借此向给予我帮助、理解和支持的你们致以真挚的感谢。

首先感谢我的母校——华东师范大学。在2019年的时候录取了我,当时的心情是那样的开心、激动,因为这给予了我肯定。学校给我们提供了优美的学习环境、丰富的教学资源和浓厚的学术氛围。因此就算在此期间遇到很多困难,也从不后悔选择华师大。

我的导师曹老师,是一个特别努力上进的人,对密码学与网络安全领域的研究有独到的见解。他一直是我学习的榜样,指引我前进的方向。每次遇到问题时,老师能够深入剖析,帮我们分析问题的解决思路。生活中的他也很亲切、和蔼。还感谢我们软件学院的所有老师,让我学到了丰富的计算机基础知识和前沿技术,辅导员老师等让我感受到华师大的温暖。

还要感谢研究生期间相处时间最多的实验室小伙伴们。我们一起学习、一起吃饭、一起加班、一起聊天、一起为论文奋斗,无比开心。之前我比较喜欢一个人学习,是你们教会了我团队协作。感谢一起进步的每一个日日夜夜!

最后感谢家人对我的理解和支持,你们浓浓的爱,是我前进的动力。

在即将说再见的时刻,心情错综复杂:有面对新环境的恐惧、朋友离别的伤心、顺利毕业的喜悦……感谢让我遇到你们,我想说你们辛苦了,愿你们家庭幸福、快快乐乐、心想事成、永生不忘!	
	}
	
	\vspace{0.2cm}
	
	\vspace{0.2cm} \hspace{9.8cm}
	何慧娴
	
	\hspace{9cm}  二零二壹年九月

} 

\pagestyle{plain}
\clearpage
\phantomsection
\addcontentsline{toc}{chapter}{发表论文和科研情况}
\chapter*{\large 攻读硕士学位期间发表论文、参与科研和获得荣誉情况}
\vskip 2mm
\vspace{-1cm}
\renewcommand{\labelenumi}{[\arabic{enumi}]}
{\heiti $\blacksquare$ 已完成学术论文}\vskip 3mm
\begin{enumerate}
    \item \textbf{第一作者}, 第二作者. Adaptive Privacy-preserving and Shuffling Aggregation in Federated-learning[C]. 2021 The 11th International Workshop on Computer Science and Engineering, Shanghai, China.[第一作者]

\end{enumerate}




\printindex
\end{document}

