\chapter{联邦学习的自适应加噪机制}
\label{ch3}

最近研究表明深度神经网络容易受到对抗样本的攻击。为了解决这个问题,一些工作通过向图像中添加高斯噪声来训练网络,从而提高网络防御对抗样本的能力,但是该方法在添加噪声时并没有考虑到神经网络对图像中不同区域的敏感性是不同的。针对这一问题,提出了梯度指导噪声添加的对抗训练算法。该算法在训练网络时,根据图像中不同区域的敏感性向其添加自适应的噪声,在敏感性较大的区域上添加较大的噪声,抑制网络对图像变化的敏感程度,在敏感性较小的区域上添加较小的噪声,提高其分类精度。
提出一种基于数据差分隐私保护的随机梯度下降算法。引入范数剪切与附加高斯噪声操作,对传统梯度更新策略进行改进。为衡量每次迭代过程中对数据隐私性的破坏,提出隐私损失累积函数在迭代过程中对数据隐私性的侵犯程度进行度量。

\section{模型概况}

\subsection{系统架构}
如图\ref{fig:联邦学习的系统架构}所示,在我们的系统模型中,有两方,即云服务器和用户。
\begin{figure}[!hbt]
\centering
	\includegraphics[scale=0.7]{fig2/C3/联邦学习系统架构}%联邦学习的系统架构
	\label{fig:联邦学习的系统架构}	
\end{figure}


\textbf{云服务器}:云服务器事先与用户协商一个网络框架。然后,服务器通过公共数据训练一个初始模型,然后将初始模型的参数广播给用户。用户在本地训练各自的模型后,云服务器收集用户发送的模型梯度,并更新全球模型。

\textbf{用户}:用户下载由云服务器初始化的模型参数。然后,每个用户在本地数据集上训练私人模型。最后,用户将本地模型的扰动梯度发送到云服务器。

每个参与者在第一次进入联邦学习系统时,都会初始化参数。针对统一的学习目标,在本地训练集上进行模型的训练。联邦学习系统同样包括参数交换协议,在参数交换协议下,参与者将本地所得神经网络梯度的参数上传至参数服务器,同样通过参数服务器下载最新的全局参数值至本地继续训练。参与者可以在本地独立训练时,避免了使用局限训练集的单个本地模型的过拟合。模型经过训练之后,每个参与者都可以使用新的测试集独立且隐私的对其进行评估与测试,无需再进行交互操作。 

\subsection{本地训练}
在分布式联邦学习中, 第 $i$ 个参与者在本地将会对全局神经网络参数的一个局部向量 $w^{i}$进行维护、学习和更新称之为本地训练。参数服务器负责对全局参数向量 $w^{g l o b a l}$进行维护和更新。每个参与者在开始训练时可以随机初始化本地参数, 也可以从参数服务器下载其参数最新值。

每个参与者都会使用统一标准的神经网络算法训练模型,使用的神经网络算法不局限于简单深度神经网络与卷积深度神经网络,但所有参与者需要进行统一, 本文使用的是采用选择性随机梯度下降算法全连接层的卷积神经网络 $\mathrm{CNN}$ 。 本地模型网络多次迭代训练其本地训练集。在本地训练期间,不同参与者之间不需要额外的共享样本和交互,他们通过参数服务器通过参数共享间接影响彼此的训练结果。

\begin{algorithm}[!htb]
	\caption{联邦学习客户端本地训练算法}
	\label{FLLT algorithm}
	\begin{algorithmic}[1]
		\footnotesize
		\STATE \textbf{Input:} 全局模型参数 $\boldsymbol{w}^{\text {global }}$, 初始化参数 $\boldsymbol{w}^{i}$
	    \STATE Enable users for training: initialize model//初始化模型 
	    \STATE for $(e p o c h=1$ to $\mathrm{n})$ do
	    \STATE Download parameters $\boldsymbol{\theta}_{d}$ from PS
	    \STATE Run $\boldsymbol{CNN}$ on local dataset
	    \STATE Update the $\boldsymbol{w}^{i}$ according to $(2-5)$
	    \STATE Compute $\Delta \boldsymbol{w}^{i}$
		\STATE Upload $\Delta \boldsymbol{w}_{s}^{i}$ to $\mathbf{P S}$
		\STATE end
	\end{algorithmic}
\end{algorithm}

算法\ref{FLLT algorithm}描述了参与者在进行本地训练时具体步骤。每个参与者独立进行深度神学习训练,在每个训练阶段由五个步骤组成。在初始化之后, 第 $i$个参与者从参数服务器(Parameter Server, PS)中下载了最新参数的分量 $\theta_{d}$,将下载的值覆盖至其本地参数,之后会在本地训练数据集上训练神经网络。

在算法的第 6 步中,参与者计算全连接层算法训练局部参数变化得到梯度向量 $\Delta w^{i}$ 。参数 $\Delta w^{i}$ 反映了对于第 $i$ 个参与者, 每个神经元中的权重向量需要变化多少能够 得到更精确的模型。 $\Delta w^{i}$的参数信息正是其他参与者需要训练更好模型以及避免的本地数据过拟合的信息。
$\Delta \boldsymbol{w}_{s}^{i}$ 表示经过选择后上传的参数。在上传训练结果前,选择一个大于阈值 $T$ 的子 集替代完整的参数向量, 参与者选择上传更有助于目标函数的梯度值, 可以使得训练迭代过程收敛更快, 模型精度更高, 以及陷入局部最优的可能性更小。

在本地训练时,卷积神经网络的全连接层采用了选择性随机梯度下降算法。Shokri在[16]中证明了其与传统的随机梯度下降算法有着几乎相同的准确性。原 因是选择参数上传更新全局模型与传统随机梯度下降算法求最优值的原因相同, 选择的过程增加了最优化过程的随机性。

参与者单独训练模型时, 由于训练集的多次使用与缺少更新, 很容易陷入局部最优。在训练本地模型时,参与者使用梯度参数的子集对模型进行更新, 会增加模型优化过程中的随机性, 很大程度上避免了本地$\mathrm{SGD}$过多使用相同的小样本集产生的模型过拟合。使用其他参与者用在不同数据集上训练学习的值覆盖本地学习的参数,可以邦助每个参与者跳出局部最优,从而得到更准确的模型。

\subsection{全局参数更新}
联邦学习通过协调深度学习任务, 建立统一的深度学习模型结构后, 参数服务器会初始化全局参数 $w^{gloabal}$。之后处理系统内参与者的上传和下载请求, 存储参与者的局部参数, 并计算更新全局参数 $w^{g l o b a l}$。当参与者上传参数时,参数服务器会将上传的 $\Delta\boldsymbol{w}_{s}^{i}$的值添加至相应的全局参数中,并为每个全局权重参数更新元数据和计数器stat。具体更新规则如下:

对于所有的 $j \in S$ :
$$
w^{\text {global }}:=w^{\text {global }}+\Delta \boldsymbol{w}_{j}^{i}
$$
为了增加更新的参数的权重, 服务器可以周期性地将计数器乘以衰减因子 $\beta$, 即:
$$
\text { stat }:=\beta \cdot \text { stat }
$$
当参与者从服务器获取具有最大统计值参数的最新值时, 将在下载期间使用这些统计信息。每个参与者都可以通过设置 $\theta_{d}$决定下载这些参数的某一部分。


\subsection{威胁模型}
我们认为云服务器是一个 "诚实但好奇 "的实体。也就是说,服务器将遵循与所有用户的协议。然而,通过利用完全访问用户梯度的便利,它也试图在训练过程中获得额外的信息。出于这个原因,我们的提出的自适应加噪机制目的是保护发送到服务器的本地梯度不被推断出任何关于用户的额外信息,并且维持原有模型的精度。


\subsection{联邦学习中的差分隐私}
传统的联邦学习中使用差分隐私的主要流程如下所示:
\begin{itemize}
\item 本地计算:
客户端 $\mathrm{i}$ 根据本地数据库 $\mathcal{D}_{\mathrm{i}}$ 和接受的服务器的全局模型 $\mathrm{w}_{\mathrm{G}}^{\mathrm{t}}$ 作为本地的参数,即 $\mathrm{w}_{\mathrm{i}}^{\mathrm{t}}=\mathrm{w}_{\mathrm{G}}^{\mathrm{t}}$, 进 行梯度下降策略进行本地模型训练得到 $\mathrm{w}_{\mathrm{i}}^{\mathrm{t}+1} \quad(\mathrm{t}$ 表示当前round) 。

\item: 模型扰动:
每个客户端产生一个随机噪音 $\mathrm{n}, \mathrm{n}$ 是符合高斯分布的,使用 $\overline{\mathbf{w}_{\mathrm{i}}}^{\mathrm{t}+1}=\mathrm{w}_{\mathrm{i}}^{\mathrm{t}+1}+\mathrm{n}$ 扰动本地模型 (这里注意w是一个矩阵,那么n就对矩阵的每一个元素产生噪音)。

\item 模型聚合:
服务器使用FedAVG算法聚合从客户端收到的 $\overline{\mathrm{w}}_{\mathrm{i}} \mathrm{t}+1$ 得到新的全局模型参数 $\mathrm{w}_{\mathrm{G}}^{\mathrm{t}+1}$, 也就是扰动过的 模型参数。

\item 模型广播:
服务器将新的模型参数广播给每个客户端。

\item 本地模型更新:
每个客户端接受新的模型参数,重新进行本地计算。
\end{itemize}

\section{方案设计}
\subsection{自适应噪声添加}
在第二章中介绍了关于神经网络的结构,
$$
y=a(\mathbf{x} * \omega+b)
$$是学习模型中每个隐藏神经元的转化过程。
其中$\mathbf{x}$代表输入向量,$y$是输出,$b$和$\omega$分别代表偏置项和权重矩阵。$a()$是一个激活函数,用于结合线性变换和非线性变换。$z(\omega)=mathbf{x}。* \omega+b$ 是线性变换部分。

由于神经网络的结构,上一层的输出是下一层的输入,由此我们可以得出,原始数据只被第一隐层的线性变换所利用。直观地说,为了得到一个具有隐私保护的学习模型,我们可以在第一层隐藏层的数据中注入噪声。正如Phan等人[15]提到的,对于线性变换有一种传统的方法,即向原始数据注入具有相同隐私预算的噪声,但是这容易导致隐私预算增加,并且使原始数据失真过多。因此,本文提出一种自适应噪声添加算法,针对每个梯度计算其贡献值,根据贡献值进行梯度裁剪并添加噪声。

首先,引入了两个调整因素。其中,$f$代表一个阈值,用于决定属性对模型结果输出的贡献是高还是低,其值由用户定义,即贡献超过阈值$f$的属性类对输出的贡献更大。然后,我们向所有这些属性注入自适应拉普拉斯噪声。当贡献率低于阈值$f$时,对这些属性进行概率选择。也就是说,我们选择概率为$1-p$的原始数据,并对一些概率为$p$的属性注入自适应拉普拉斯噪声。该公式如下。
$$
\tilde{x}_{i, j}=\left\{\begin{array}{ll}
\ddot{x}_{i, j} & \beta \geq f \\
\bar{x}_{i, j} & \beta<f
\end{array}\right.
$$

其中$Beta$代表贡献率:$\beta=\frac{\left|\ddot{C}_{j}\right|}{\sum_{j=1}^{u}\left|\ddot{C}_{j}\right|}$,当$\beta<f$时,我们有:

$$\tilde{x}_{i, j}=\left\{\begin{array}{ll}\ddot{x}_{i, j} & \beta \geq f \\ \bar{x}_{i, j} & \beta<f\end{array}\right.$$


$f$和$p$是超参数,用户可以根据自己的情况来调整。

每个属性类的隐私预算比率$epsilon_{j}$由。也就是说,隐私预算$epsilon_{l}$是根据贡献率:$\epsilon_{j}=\frac{u *\left|\ddot{C}_{j}\right|}{\sum_{j=1}^{u}\left|\ddot{C}_{j}\right|} * \epsilon_{l}$按比例分配给每个属性类。自适应噪声按以下方式注入属性中:

$x_{i, j}^{\prime}=x_{i, j}+\frac{1}{\left|D_{i}^{t}\right|} \operatorname{Lap}\left(\frac{G S_{l}}{\epsilon_{j}}\right)$

在不丧失一般性的情况下,调整因子$f$和$p$的值与系统的准确性和隐私水平有关。即$f$越小,$p$越大。越高的秘密水平,准确性越低,反之亦然。

我们用层间相关性传播(LRP)算法将输出分解到每一层。关于LRP算法的更多细节,我们将在以下部分进行介绍。每个用户都在本地对原始数据进行训练前馈操作,这可以获得一个新的数据操作,从而获得本地模型的输出。根据相邻层之间的线性关系,在$k-t h$ 层的神经元的贡献$C_{a_{i}}^{l_{k}}\left(x_{i}\right)$等于连接到神经元$a_{i}$的相邻层的贡献之和:

$$
C_{a_{i}}^{l_{k}}\left(x_{i}\right)=\sum_{a_{j} \in l_{k+1}} C_{a_{i} \leftarrow a_{j}}^{l_{k} \leftarrow l_{k+1}}\left(x_{i}\right)
$$


例如,如图2所示,我们有:
$$
C_{a_{7}}^{l_{2}}\left(x_{i}\right)=\sum_{a_{j} \in l_{3}} C_{a_{7} \leftarrow a_{j}}^{l_{2} \leftarrow l_{3}}\left(x_{i}\right)=C_{a_{7} \leftarrow a_{8}}^{l_{2} \leftarrow l_{3}}\left(x_{i}\right)+C_{a_{7} \leftarrow a_{9}}^{l_{2} \leftarrow l_{3}}\left(x_{i}\right)
$$

其中,"$_leftarrow$"表示两部分之间的连接关系。"$l_{2} \l_{3}$ " 是指深度神经网络(DNNs)中$2-t h$层和第3层之间相邻层的连接关系。
当$k-t h$层为输出层时,我们有:
$$
C_{a_{i}}^{l_{k}}\left(x_{i}\right)=f\left(x_{i}, \omega_{i}^{r}\right)
$$

\subsection{隐私性证明}
随机隐私保护调整技术(RPAT)对第3.2.1节中讨论的线性变换函数进行了扰动,该函数满足$left(\epsilon_{c}+\epsilon_{l}right)$差分隐私。证明如下。
假设两个相邻的批次$D_{i}^{t}$和$D_{i}^{t^{prime}}$,其最后一个元组$x_{n}$和$x_{n}^{prime}$不同,$z\left(D_{i}^{t}\right)$和$z\left(D_{i}^{t^{\prime}}\right)$分别为线性变换函数。RPAT满足$\left(\epsilon_{c}+\epsilon_{l}\right)$的差分隐私。

\begin{proof}
一般来说,我们把偏置项视为第一类数据属性,即:$x_{i, 0}=b_{i}$。线性转换可以改写为:$\ddot{\mathbf{z}}_{x \in D_{i}^{t}}(\omega)=\ddot{\mathbf{x}} * \omega$。线性变换的敏感性$G S_{l}$如下:

$$\begin{aligned} G S_{l} &=\sum_{a_{i} \in l_{1}} \sum_{j=1}^{u}\left\|\sum_{x_{i} \in D_{i}^{t}} x_{i, j}-\sum_{x_{i}^{\prime} \in D_{i}^{t^{\prime}}} x_{i, j}^{\prime}\right\|_{1} \\ &=\sum_{a_{i} \in l_{1}} \sum_{j=1}^{u}\left\|x_{n, j}-x_{n, j}^{\prime}\right\|_{1} \\ & \leq \sum_{a_{i} \in l_{1}} \sum_{j=1}^{u} \max _{x_{i} \in D_{i}^{t}}\left\|x_{n, j}\right\|_{1} \\ & \leq \sum_{a_{i} \in l_{1}} u \end{aligned}$$

其中,$a_{i} \in l_{1}$是指第一隐藏层$l_{1}$中的神经元$a_{i}$,$u$是数据元组$x_{i} \in D_{i}^{t}$中的属性数。
我们设计了RPAT,它包括两个调整因素。$f$和$p$,它们可以过滤多余的噪声。RPAT之后的属性的一般表达式如下:

$$
\begin{aligned}
\tilde{x}_{i, j} &=[(1-f)+f * p] * \ddot{x}_{i, j}+f *(1-p) * x_{i, j} \\
&=[(1-f)+f * p]\left[x_{i, j}+\operatorname{Lap}\left(\frac{G S_{l}}{\epsilon_{j}}\right)\right]+[f *(1-p)] x_{i, j} \\
&=x_{i, j}+[(1-f)+f * p]\left[\operatorname{Lap}\left(\frac{G S_{l}}{\epsilon_{j}}\right)\right]
\end{aligned}
$$
然后我们可以得到:

$$
\begin{aligned}
\frac{\operatorname{Pr}\left(\ddot{\mathbf{z}}_{D_{i}^{t}}(\omega)\right)}{\operatorname{Pr}\left(\ddot{\mathbf{z}}_{D_{i}^{t}}(\omega)\right)} &=\frac{\prod_{a_{i} \in l_{1}} \prod_{j=1}^{u} \exp \left(\frac{\epsilon_{j}\left\|\sum_{x_{i} \in D_{i}^{t}} x_{i, j}-\sum_{x_{i} \in D_{i}^{t}} \tilde{x}_{i, j}\right\|_{1}}{G S_{l}}\right)}{\prod_{a_{i} \in l_{1}} \prod_{j=1}^{u} \exp \left(\frac{\epsilon_{j}\left\|\sum_{x_{i}^{\prime} \in D_{i}^{t^{\prime}}} x_{i, j}^{\prime}-\sum_{x_{i}^{\prime} \in D_{i}^{t^{\prime}}} \tilde{x}_{i, j}^{\prime}\right\|_{1}}{G S_{l}}\right)} \\
& \leq \prod_{a_{i} \in l_{1}} \prod_{j=0}^{u} \exp \left(\frac{\epsilon_{j}}{G S_{l}}\left\|\sum_{x_{i} \in D_{i}^{t}} x_{i, j}-\sum_{x_{i}^{\prime} \in D_{i}^{t^{\prime}}} x_{i, j}^{\prime}\right\|_{1}\right) \\
& \leq \prod_{a_{i} \in l_{1}} \prod_{j=0}^{u} \exp \left(\frac{\epsilon_{j}}{G S_{l}} \max _{x_{i} \in D_{i}^{t}}\left\|x_{n, j}\right\|_{1}\right) \\
& \leq \exp \left(\epsilon_{l} \frac{\sum_{a_{i} \in l_{1}} u\left[\sum_{j=1}^{u} \frac{\left|\ddot{C}_{j}\right|}{\sum_{j=1}^{u}\left|\ddot{C}_{j}\right|}\right]}{G S_{l}}\right) \\
&=\exp \left(\epsilon_{l}\right)
\end{aligned}
$$
\end{proof}

根据上述推倒证明可知,在联邦学习的神经网络中添加自适应噪声后,所上传的梯度是满足$\left(\epsilon_{c}+\epsilon_{l}\right)$差分隐私的。在满足差分隐私的基础上,在下一节我们会给予隐私损失累积函数计算隐私成本。

\subsection{隐私预算分析}
对于所提差分隐私SGD算法,除了确保算法运行的准确率以外,另一个重要的问题就是评估算法训练时的数据隐私损失成本。为此,提出隐私损失累积函 数的概念来进行每次迭代过程访问训练数据的隐私损 失以及随着训练进展时的累积隐私损失。
为不失一般性,令 $\sigma=\frac{\sqrt{2 \log (1.25 / \delta)}}{\varepsilon}$, 文献[14]严格证明,对于抽样概率 $q=\frac{\mathcal{L}}{N}$ 且 $\varepsilon<1$, 则对于完整样本而言,每次迭代过程都是 $(O(q \varepsilon), q \varepsilon)$-差分隐私的。 但文献并末对迭代过程以及噪声强度对差分隐私损失的影响展开研究,故无法对噪声强度以及剪切阈值$C$进行有依据的选取。故首先需要研究迭代过程对差分隐私的影响机制。

事实上,若令 $\sigma \geqslant c_{2} \frac{q \sqrt{T \log (1 / \delta)}}{\varepsilon}$,则同样应用文献[14]方法,可以严格证明算法对于任意的 $\varepsilon<c_{1} q^{2} T$ 都是 $(O(q \varepsilon \sqrt{T}), \delta)-$ 差分隐私的,其中 $c_{1}$ 和 $c_{2}$ 为常数。与文献 14]相比,本文算法能够在相同迭代步骤下,大幅度降低 $\varepsilon$ 的数值,对数据的隐私性保护更高。进一步地, 对于两个相邻的数据集 $d$,$d^{\prime} \in D$ 和映射机制 $M$,引入一个辅助输入变量 aux和输出$o \in R$, 定义映射机制$M$在输出$o$处的隐私损失为:

$$
c\left(o ; M, a u x, d, d^{\prime}\right) \triangleq \log \frac{\operatorname{Pr}[M(a u x, d)=o]}{\operatorname{Pr}\left[M\left(a u x, d^{\prime}\right)=o\right]}
$$

对于所提差分隐私 SGD 算法而言,神经网络各层 权重系数的参数值与每次迭代过程中的差分隐私机制 有着紧密的关联,从而对于给定的映射机制 $M$,在第 $\lambda$ 次迭代过程的隐私损失定义为:

$$
\begin{array}{r}
\alpha_{\mathcal{M}}\left(\lambda ; a u x, d, d^{\prime}\right) \triangleq \log \mathbb{E}_{o \sim M(a u x, d)}[\exp (\lambda c(o) ; M \\
\left.\left.\left.d, d^{\prime}\right)\right)\right]
\end{array}
$$

进一步地,映射机制 $M$ 的损失边界值定义为:

$$\alpha_{\mathcal{M}}(\lambda) \triangleq \max _{a u x, d, d^{\prime}} \alpha_{M}\left(\lambda ; a u x, d, d^{\prime}\right)$$

其满足以下特性:

\begin{itemize}
\item 组合特性:给定一个机制 $M$, 由一组子机制顺序 $\left\{M_{1}, M_{2}, \cdots, M_{k}\right\}$ 组成,并满足$M_{i}: \prod_{j=1}^{i-1} R_{j} \times D \rightarrow R_{i}$,从而总隐私损失边界满足:
$$
\alpha_{M}(\lambda) \leqslant \sum_{i=1}^{k} \alpha_{M_{i}}(\lambda)
$$

\item 差分隐私边界:$\forall \varepsilon>0$, 映射机制 $M$ 是 $(\varepsilon, \delta)$ 差分隐私的,当且仅当:
$$
\delta=\min _{\lambda} \exp \left(\alpha_{M}(\lambda)-\lambda \varepsilon\right)
$$
\end{itemize}

上述2条性质确定了深度神经网络算法每次迭代的隐私损失以及所能够达到侵犯数据隐私容忍度的最大迭代次数。特别地,在附加高斯噪声的情况下,不妨令 $\mu_{0}$,$\mu_{1}$ 分别为 $N\left(0, \sigma^{2}\right)$ 和 $N\left(0, \sigma^{2}\right)$ 的概率密度函数,而 $\mu$ 为两个高斯密度函数的混合概率密度函数,即 $\mu=(1-q) \mu_{0}+q \mu_{1} \circ$ 依据式 $(5)-$ 式 $(7)$ 可推导得 $\alpha$ $(\lambda)=\log \max \left(E_{1}, E_{2}\right)$, 其中:

$$
E_{1}=\mathbb{E}_{z \sim \mu_{0}}\left[\left(\frac{\mu_{0}(z)}{\mu(z)}\right)^{\lambda}\right]
$$

$$
E_{2}=\mathbb{E}_{z \sim \mu}\left[\left(\frac{\mu(z)}{\mu_{0}(z)}\right)^{\lambda}\right]
$$

