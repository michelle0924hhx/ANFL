\chapter{总结与展望}
\label{ch6}
\section{总结}
随着机器学习在各个领域的大规模应用,尤其是在安全领域也占有了重要的地位。机器学习模型的安全性和可解释性也引起各国政府和研究学者的极大关注。鲁棒性是模型安全性的重要体现之一,所以验证模型鲁棒性的方法和工具也变得尤为迫切。

树模型以其高效,方便,泛化能力强的特点,在各个领域都有广泛的应用。但与神经网络模型一样,树模型也易收到对抗性样本的影响。本文基于 SMT 技术对机器学习树模型的鲁棒性进行了研究和分析。本文的主要工作和贡献如下:
\begin{enumerate}
\item [(1)] 本文提出了一个基于SMT技术树模型的鲁棒性验证框架,该框架支持树模型中两个重要实现:随机森林与 GBDT 模型的鲁棒性的验证。该框架能够有效验证大规模的树模型的鲁棒性。
\item [(2)] 本文提出了鲁棒特征集和局部鲁棒特征重要度的概念,从模型可解释性的角度,进一步研究了树模型的鲁棒性和样本特征之间的关系。为对抗性样本反例的生成和对抗性攻击提供新的思路,也可以帮助模型开发人员进一步优化模型提高其鲁棒性。
\item [(3)] 本文通过实验讨论了树模型超参数与模型鲁棒性的关系,从而为训练阶段提高模型鲁棒性的研究提供了重要参考。
\end{enumerate}

综上所述,本文的研究充分证明了所提出框架的有效性,可以极大的增加树模型的可靠性,同时也对鲁棒性的可解释性做了研究,从而进一步推进了树模型在安全领域的应用和发展。

\section{展望}

我们的研究还留存一些待解决的问题,可以考虑从下面的几个方面展开研究:
\begin{enumerate}
\item [(1)] 本文所提出的鲁棒性验证框架的验证能力很大程度上受限于 SMT求解器本身的求解能力,在未来应该考虑针对树模型编码成SMT公式的特点,对SMT的底层求解算法进行优化,从而提高验证的效率。
\item [(2)] 虽然在现有的验证框架下,可以支持一些大规模树模型的验证,但是对于高维度,更大规模的模型还是会发生状态爆炸的问题,导致验证结果无效。在未来可以先对树模型本身先进行模型缩减的操作,之后再进行验证,以便可以验证超大规模的模型。
\item[(3)] 我们的验证框架在对树模型的鲁棒性验证问题上已经取得了一定的成效,接下来可以将其扩展到复杂度更高的机器学习模型上去。目前,已有研究者考虑将神经网络转换为决策树,使其模型的复杂度降低。我们可以沿着这个思路,将我们的方法扩展到神经网络模型上去,进一步加强框架的验证能力。
\end{enumerate}

