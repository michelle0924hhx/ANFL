\chapter{联邦学习中的本地自适应差分隐私机制}

\label{ch3}

\section{引言}
与传统的集中式深度学习相比,联邦学习通过分布式训练在一定程度上缓解了隐私泄漏的问题。然而,许多研究表明,在训练过程中,本地设备与中央服务器之间的通信信道和传递的模型参数都有可能成为第三方窃取敏感信息的途径,联邦学习的框架仍然存在本地训练数据泄漏等隐私威胁\upcite{ref48}。深度学习技术可以“记忆”模型中的训练数据信息,在这种情况下,敌方一旦通过白盒推理攻击或者黑盒推理攻击访问模型,就可以推演出客户端本地的训练数据。

联邦学习模型的优化问题可以概括为ERM(经验风险最小化)问题\upcite{ref40}:
\begin{equation}\label{eq:ERM}
\arg \min _{\theta \in \mathcal{C}}\left(F(\theta):=\frac{1}{m} \sum_{i=1}^{m} F_{i}(\theta)\right)
\end{equation}

从隐私保护的角度讲,我们只要截断了从原始输入到输出,在其中加入一道隐私保护屏障,具体在哪一步截断则对应于不同的方法。差分隐私保护机器学习的方法具体有以下几种:
\begin{itemize}
	\item \textbf{输入扰动:} 输入扰动是在获取的训练数据上直接添加噪声,之后的模型训练和优化都是基于加躁后的训练数据\upcite{ref37}\upcite{ref38}\upcite{ref39}。
	\item \textbf{输出扰动:} 输出扰动沿袭了拉普拉斯机制最简单的思路,即考虑函数输出的敏感度来添加噪声,那么在ERM公式中我们只需要考虑argmin函数输出的敏感度,基于这个敏感度来添加拉普拉斯噪声即可得到一个简单的满足差分隐私的ERM方法\upcite{ref36}。
	\item \textbf{梯度扰动:} 梯度扰动是在执行最小化损失函数的过程中,设计满足差分隐私的算法。
	\item \textbf{目标扰动:} 目标扰动是在模型的目标函数中添加一个随机量,以使得最终模型的输出满足随机性。
\end{itemize}

基于输入的扰动和输出的扰动基本可以视为一个黑匣子模型,简单直接。但是这种添加噪声的方式无法对训练过程中数据的相互依赖性和输出有效性作出有用的、紧密的描述。在输入数据中加入过多的噪声,可能会影响模型训练的收敛性。在输出参数中加入过于保守的噪声,也就是根据最坏的攻击情况去添加噪声,可能会影响模型的实用性。

当前在深度学习模型中应用差分隐私的主流方案是在模型的梯度上添加噪声,方案的目标是在满足差分隐私的条件下,实现整体模型的最优可用性。Song等人\upcite{ref47}提出了一个$\left(\epsilon_{c}+\epsilon_{d}\right)$-差分隐私版本的随机梯度下降算法。在模型的每一次迭代过程中,对梯度添加高斯噪声,并通过差分隐私的组合性和隐私放大效果,得到完全隐私损失的上界。传统的联邦学习中使用差分隐私的主要流程如下所示:
\begin{itemize}
\item 本地计算:
客户端 $\mathrm{i}$ 根据本地数据库 $\mathcal{D}_{\mathrm{i}}$ 和接受的服务器的全局模型 $\mathrm{w}_{\mathrm{G}}^{\mathrm{t}}$ 作为本地的参数,即 $\mathrm{w}_{\mathrm{i}}^{\mathrm{t}}=\mathrm{w}_{\mathrm{G}}^{\mathrm{t}}$, 采用梯度下降策略进行本地模型训练得到 $\mathrm{w}_{\mathrm{i}}^{\mathrm{t}+1} \quad(\mathrm{t}$ 表示当前通信回合) 。

\item 模型扰动:
每个客户端产生一个随机噪音 $\mathrm{n},\mathrm{n}$ 是符合高斯分布的,使用 $\overline{\mathbf{w}_{\mathrm{i}}}^{\mathrm{t}+1}=\mathrm{w}_{\mathrm{i}}^{\mathrm{t}+1}+\mathrm{n}$ 扰动本地模型 (这里注意w是一个矩阵,n表示对矩阵的每一个元素添加噪音)。

\item 模型聚合:
服务器使用参数聚合算法聚合从客户端收到的 $\overline{\mathrm{w}}_{\mathrm{i}} \mathrm{t}+1$ ,得到新的全局模型参数 $\mathrm{w}_{\mathrm{G}}^{\mathrm{t}+1}$, 也就是扰动过的模型参数。

\item 模型广播:
服务器将新的模型参数广播给每个客户端。

\item 本地模型更新:
每个客户端接受新的模型参数,重新进行本地计算。
\end{itemize}

然而,在传统的基于差分隐私的联邦学习框架中,数据管理者倾向于给每个用户的数据以相同的隐私预算,同样的隐私预算忽略了用户之间的差异。有些用户希望有更好的隐私保护。而有些用户对某些数据的隐私不敏感。在这种情况下,由于联邦学习模型是分布式结构,从一个大数据库到许多小数据库,所以对于每个用户来说。他们只需要关心他们自己的隐私。他们可以设置不同的隐私预算方案,而不是传统的统一分配,然后在最坏的的情况下注入噪音。因此本文采用一种更加复杂的方法来分析训练过程中训练数据对模型输出的贡献比率,然后根据每一层神经网络对模型输出的贡献率,在梯度上自适应添加噪声。

在本文中,我们认为中央参数服务器是半可信的(Honest but Curious, HbC),一个"诚实但好奇"的实体。也就是说,服务器将遵循与所有用户的协议。然而,通过利用通信信道访问用户梯度的便利,它也试图在训练过程中反推出关于客户端的额外的信息。出于这个原因,我们的提出的自适应加噪机制目的是保护发送到服务器的本地梯度不被推断出任何关于用户的本地训练样本信息,并且尽量维持原有模型的精度。

总的来说,本章提出的隐私保护方案是基于本地客户端的本地数据维度的,从以下三个方面展开研究:第一,通过在本地模型训练的梯度下降算法过程中针对不同层的贡献比自适应添加噪声;第二,根据训练轮数和层间贡献率对梯度进行自适应裁剪;第三,结合差分隐私的组合定理和后处理定理设计动量组合机制,计算累积的隐私预算,分析模型整体的隐私性。

在过去的十年中,已经提出了许多用于解决ERM问题的差分私有机器学习算法。几乎所有这些都是针对具有凸损失函数的ERM,但许多重要的机器学习方法,包括深度学习,都被表述为具有非凸损失函数的ERM问题。此外,这些学习问题通常需要大量的训练集,因此需要使用随机优化算法,例如随机梯度下降算法。由于非凸ERM函数的灵敏度难以计算,很难在非凸ERM算法上使用目标扰动、输出扰动等隐私保护方法,大多数非凸ERM的差分隐私算法都基于梯度扰动。而基于梯度扰动的方法的问题在于它们的迭代性质会导致隐私预算的飙升。因此,当前的主要挑战是为非凸ERM设计一种新型的满足差分隐私的扰动算法,既能保证模型的效用性,并且维持较高的计算效率。

最近几项关于在非凸优化问题上应用差分隐私算法的研究难点在于难以权衡模型精度和隐私成本。例如,Wang等人\upcite{ref62}提出了一种具有隐私和效用保证的差分隐私梯度下降 (DP-GD)算法。然而,DP-GD在每次模型每次迭代过程中都需要计算完整的梯度,在大型训练集会造成使用成本太高。Zhang\upcite{ref61}等人提出了一种随机轮差分隐私随机梯度下降 (RRPSGD),它可以实现与DP-GD相同的隐私保证,但运行时复杂度降高,而且模型效用稍差。Goodfellow等人\upcite{ref57}提出了一种称为动量会计的方法,用于在训练过程中跟踪随机梯度下降算法的隐私成本,这提供了强大的隐私保证。 Papernot等人\upcite{ref58}建立了一个私人教师集合(PATE)框架,以提高用于分类任务的深度学习的隐私保障。Xie\upcite{ref59}和Jordon\upcite{ref60}研究了具有不同距离度量的差分私有生成对抗网络 (GAN)。然而,这些作品都没有为其算法提供有效的隐私效用保证。

现有研究为非凸ERM提供效用保证的差分隐私非凸优化算法包括随机回合私有随机梯度下降算法(Random Round Private Stochastic Gradient
Descent,RRPSGD)和差分隐私的梯度下降算法(Differentially Private Gradient Descent,DP-GD)。RRPSGD是由Zhang\upcite{ref61}等人提出的首个针对非凸ERM优化提出的满足差分隐私,并提供隐私效用保证的算法。该算法在模型迭代过程中,随机选取梯度下降的回合,在梯度上添加高斯噪声,该方案针对隐私效用保证的分析是基于由子采样和强组合性所构成的标准隐私放大效应。尽管此方案在非凸的随机优化算法可以使模型达到收敛点,但与时刻会计和高斯差分隐私的宽松定义相比,此方案会导致添加噪声后的训练结果产生较大的方差。并且,算法的时间复杂度达到了$O\left(n^{2}\right)$,使得训练该模型达到收敛的时间大大增加。

Wang等人\upcite{ref62}等人提出的DP-GD算法也是针对非凸ERM优化函数满足差分隐私的算法。与RRPSGD相比,DP-GD的算法时间复杂度为$O\left(n^{2} \epsilon / d^{1 / 2}\right)$,相对提高了$O\left((d \log (1 / \delta))^{1 / 4} /(n \epsilon)^{1 / 2}\right)$,这是因为DP-GD在全部的梯度上都添加了噪声,而RRPSGD仅在随机梯度上添加了噪声。然而这也使得DP-GD在计算上非常昂贵,甚至难以处理大规模机器学习问题(当训练样本量特别大时)。最近,Wang等人\upcite{ref63}还提出了一种用于非凸ERM优化函数的差分隐私随机算法。他们的目标是找到局部最小值,而我们的目标是找到函数的静止点。此外,它们的效用保证是渐近的——只有在可以运行无限次迭代时才提供所需的效用保证。相比之下,我们的效用保证适用于有限次数的迭代。

在本文中,我们提出了一种用于非凸ERM的差分隐私随机递归动量算法。我们算法的核心是随机递归动量技术,它能显著减少梯度下降过程中累积的方差。我们的方法比随机方差减少算法更具可扩展性,因为它消除了检查点梯度的周期性计算,降低了运行时算法的时间复杂度,而且提高了模型的效用性。我们针对非凸ERM优化函数设计了一种新的差分隐私随机优化算法,并使用Renyi差分隐私(RDP)对算法的隐私保证进行了详细分析。 我们的算法与非凸优化的最著名的效用保证相匹配,具有较低的计算复杂度。为实现相同的效用保证,我们算法的梯度复杂度(即计算的随机梯度总数)为$O\left(n^{3 / 2}\right)$,比之前的最佳结果高出$\Theta\left(n^{1 / 2}\right)$。 之后我们在两种非凸ERM技术(非凸逻辑回归和卷积神经网络)上进行实验,评估我们提出的方法,发现我们的方法不仅产生了在模型精度方面最接近非私有模型的模型,而且还降低了计算成本。

\section{基于自适应差分隐私的随机梯度下降算法}

算法\ref{基于自适应差分隐私的随机梯度下降算法}详细描述了在本地客户端训练过程中,在SGD算法中添加自适应差分隐私,并使用差分隐私组合定理衡量所添加的噪声大小。首先,我们采用先验组合机制计算$eps_{iter}$和$\delta_{iter}$(算法第7行)。每个客户端对训练数据进行采样,并计算他们的隐私预算$\delta_{u}$。如果$\delta_{u}>\delta$,用户将终止采样和训练,并且不上传其梯度信息(算法第9-12行)。否则,用户将计算梯度的贡献率(算法第16行),接着使用拉普拉斯机制注入自适应的噪声量(算法第18行),然后对梯度进行范数裁剪(算法第19行)。最后,服务器对用户上传的梯度进行聚合,并更新模型参数$w$。该算法有两个主要部分:自适应差分隐私,梯度范数裁剪。

\begin{algorithm}[!htb]
	\caption{基于自适应差分隐私的随机梯度下降算法}
	\label{基于自适应差分隐私的随机梯度下降算法}
	\begin{algorithmic}[1]
		\footnotesize
		\STATE \textbf{输入:} 预估迭代次数$T$,学习率$\alpha$,梯度裁剪阈值$C$,目标损失函数$l$,拉普拉斯机制噪声$(\Delta, \varepsilon, \delta)$,$f$贡献率阈值,$p$注入噪声的概率
		\STATE \textbf{输出:} 模型梯度
		\STATE 初始化模型权重$w$
		\WHILE{$\exists \delta_{u}<\delta$}
			\STATE $n$=0
			\STATE $grad$=0
			\STATE 计算$eps_{iter}$,$\delta_{iter}$
			\FOR {each $u \in$ Users}
				\STATE 计算$\delta_{u}$
				\IF{$\delta_{u}>\delta$}
					\STATE continue
				\ENDIF
				\STATE 从客户端数据集中随机采样
				\STATE $g t_{u}=\nabla l(w, x)$
				\STATE 计算梯度的平均贡献率
				\STATE $C_{j}\left(x_{i}\right)=\frac{1}{n} \sum_{i=1}^{n} C_{x_{i, j}}\left(x_{i}\right), j \in[1, u]$
				\STATE 添加自适应噪声
				\STATE $g t_{u}^{\prime}=g t_{u}+\frac{1}{\left|D_{i}^{t}\right|} \operatorname{Lap}\left(\frac{G S_{l}}{\epsilon_{j}}\right)$
				\STATE $g t_{u}=g t_{u} / \max \left(1, \frac{\left\|g t_{u}\right\|}{C}\right)$
				\STATE $n$++
			\ENDFOR
			\STATE $w=w-\alpha * g r a d / n$
		\ENDWHILE
	\end{algorithmic}
\end{algorithm}

在本节接下来的两个部分,我们将详细描述如何在神经网络的随机梯度下降算法中进行自适应的噪声添加和自适应的梯度裁剪。

\subsection{自适应噪声添加}
在第二章我们详细介绍了神经网络的结构,每个用户在本地用原始数据进行训练,在神经网络中进行前向传播操作,得到本地模型的输出。神经网络中前向传播算法的第一步在输入层。

我们采用神经网络前向传播算法计算每一层对模型输出的贡献比率。每个本地用户对原始数据在神经网络中进行前向传播的训练,这可以获得一个新的数据操作,从而获得本地模型的输出。

用"$\leftarrow$"表示两部分之间的连接关系。"$l_{2} \leftarrow l_{3}$" 是指深度神经网络中第2层和第3层之间相邻层的连接关系。
当第k层为输出层时,我们有:
\begin{equation}
C_{a_{i}}^{l_{k}}\left(x_{i}\right)=f\left(x_{i}, \omega_{i}^{r}\right)
\end{equation}

根据矩阵层之间的线性相关性,神经元$a_{i}$在第k层的贡献$C_{a_{i}}^{l_{k}}\left(x_{i}\right)$等于连接到神经元$a_{i}$的相邻层的贡献之和:
\begin{equation}\label{eq:层间传播1}
C_{a_{i}}^{l_{k}}\left(x_{i}\right)=\sum_{a_{j} \in l_{k+1}} C_{a_{i} \leftarrow a_{j}}^{l_{k} \leftarrow l_{k+1}}\left(x_{i}\right)
\end{equation}

因此,神经元$a_{j}$对于输出层的贡献等于模型的输出。第k-1层的神经元$a_{j}$对于第k层的神经元$C_{a_{i} \leftarrow a_{j}}^{l_{k-1} \leftarrow l_{k}}\left(x_{i}\right)$的贡献等于:
\begin{equation}
C_{a_{i} \leftarrow a_{j}}^{l_{k-1} \leftarrow l_{k}}\left(x_{i}\right)=\left\{\begin{array}{cc}\frac{a_{i} w_{i, j}}{\sum_{a_{i} \in l_{k-1} a_{i} w_{i, j}}} C_{a_{j}}^{l_{k}}\left(x_{i}\right) & \sum_{a_{i} \in l_{k-1}} a_{i} w_{i, j} \neq 0 \\ \mu & \sum_{a_{i} \in l_{k-1}} a_{i} w_{i, j}=0\end{array}\right.
\end{equation}

其中$\mu$是一个无限接近于零,但大于零的数字。从上述公式中,我们可以认为每一层的贡献是相等的,而且贡献是逐层传递的。根据以上公式的推导,我们能得到神经网络模型中每一层以及每个神经元的贡献值。

通过从数据元组中提取同一属性的贡献,我们可以计算出每个属性类对模型输出的平均贡献:
\begin{equation}\label{eq:属性添加自适应扰动}
C_{j}\left(x_{i}\right)=\frac{1}{n} \sum_{i=1}^{n} C_{x_{i, j}}\left(x_{i}\right), j \in[1, u]
\end{equation}

在原始的参数上计算神经网络中每个属性类对于模型输出的贡献后,按照公式\ref{eq:属性添加自适应扰动-拉普拉斯机制}采用拉普拉斯机制在属性类的贡献率中注入噪音以保护原始的参数。
\begin{equation}\label{eq:属性添加自适应扰动-拉普拉斯机制}
\ddot{C}_{j}\left(x_{i}\right)=C_{j}\left(x_{i}\right)+\operatorname{Lap}\left(\frac{G S_{c}}{\epsilon_{c}}\right), j \in[1, u]
\end{equation}
其中,函数的局部敏感度为$G S_{c}=\frac{2 u}{|D|}$,$u,|D|$分别代表了属性和数据元组的最大数量。

在第二章中我们介绍了关于神经网络的结构,
\begin{equation}\label{eq:神经网络参数传递}
y=a(\mathbf{x} * \omega+b)
\end{equation}

公式\ref{eq:神经网络参数传递}表示神经网络中每个隐藏神经元的转化过程。
其中$\mathbf{x}$代表输入向量,$y$是输出,$b$和$\omega$分别代表{}偏置项和权重矩阵。$a()$是一个激活函数,用于结合线性变换和非线性变换。$y=a(\mathbf{x} * \omega+b)$ 是线性变换部分。

由于神经网络的结构,上一层的输出是下一层的输入,由此我们可以得出,原始的训练数据只被第一个隐藏层的线性变换所利用。直白地说,为了得到一个具有隐私保护的学习模型,我们可以在第一个隐藏层的数据中注入噪声。正如Phan等人\upcite{ref36}提到的,对于线性变换有一种传统的方法,即向原始数据注入具有相同隐私预算的噪声,但是这容易导致隐私预算增加,并且使原始数据失真过多。因此,本文提出一种自适应噪声添加算法,针对每个梯度计算其贡献值,根据贡献值进行梯度裁剪并添加噪声。

首先,我们引入了两个调整因素$f$和$p$。其中,$f$代表一个阈值,用于决定属性对模型结果输出的贡献是高还是低,其值由用户定义,即贡献超过阈值$f$的属性类对输出的贡献更大。然后,我们向所有这些属性注入自适应拉普拉斯噪声。当贡献率低于阈值$f$时,对这些属性进行概率选择。也就是说,我们抛弃概率为$1-p$的原始数据,并对一些概率为$p$的属性注入自适应拉普拉斯噪声。该公式如下:
\begin{equation}\label{eq:神经网络加噪}
\tilde{x}_{i, j}=\left\{\begin{array}{ll}
\ddot{x}_{i, j} & \beta \geq f \\
\bar{x}_{i, j} & \beta<f
\end{array}\right.
\end{equation}

其中$\beta$代表贡献率:$\beta=\frac{\left|\ddot{C}_{j}\right|}{\sum_{j=1}^{u}\left|\ddot{C}_{j}\right|}$,当$\beta<f$时,我们有:
\begin{equation}\label{eq:神经网络加噪2}
\bar{x}_{i, j}=\left\{\begin{array}{l}
\ddot{x}_{i, j} \text { with probability } p \\
x_{i, j} \text { with probability } 1-p
\end{array}\right.
\end{equation}


$f$和$p$是超参数,用户可以根据自己的情况来调整。

在此方案中,隐私预算$\epsilon_{l}$是根据贡献率:$\epsilon_{j}=\frac{u *\left|\ddot{C}_{j}\right|}{\sum_{j=1}^{u}\left|\ddot{C}_{j}\right|} * \epsilon_{l}$按比例分配给每个属性类,自适应噪声按以下方式注入属性中:
\begin{equation}\label{eq:神经网络加噪3}
x_{i, j}^{\prime}=x_{i, j}+\frac{1}{\left|D_{i}^{t}\right|} \operatorname{Lap}\left(\frac{G S_{l}}{\epsilon_{j}}\right)
\end{equation}

在不失一般性的情况下,调整因子$f$和$p$的值与系统的准确性和隐私水平有关。即$f$越小,$p$越大,代表越高的隐私保护水平,模型准确性越低,反之亦然。在第五章我们将通过实验证明,当f值为0.15,p设置为0.85时,使用自适应的噪声分布与拉普拉斯机制基本吻合。在相同的噪声水平下,自适应噪声添加的隐私预算接近于原始的拉普拉斯机制,因此我们的方案在缩小调整系数范围的情况下能达到相近的隐私保护效果。

\subsection{梯度范数裁剪}
在模型的每一轮迭代过程中,算法将计算添加了拉普拉斯噪声的梯度$g t_{u}^{\prime}=g t_{u}+\frac{1}{\left|D_{i}^{t}\right|} \operatorname{Lap}\left(\frac{G S_{l}}{\epsilon_{j}}\right)$, 方差是 $2 b^{2}$ 。对梯度注入的噪声量为$\frac{1}{\left|D_{i}^{t}\right|} \operatorname{Lap}\left(\frac{G S_{l}}{\epsilon_{j}}\right)$,决定于用户个体对于梯度 $g$ 在二范数下的最大全局敏感度, 即$\delta$ 。由于梯度的大小没有一个先验的界限, 我们采用二范数的固定值对每个梯度进行裁剪。

用户上传的梯度向量可以改写为$g t_{u}=g t_{u} / \max \left(1, \frac{\left\|g t_{u}\right\|}{C}\right)$, 其中 $C$是裁剪阈值。对于梯度的裁剪能保证梯度值小于设定的阈值$\mathrm{i}$。 也就是当 $\|g\| \leq C$ ,$g$ 保持不变;当$\|g\|>C$ 时, 它会按照裁剪比例缩小为 $C_{\circ}$。

但是如果裁剪阈值$C$ 的值如果太小,那么裁剪后的噪声会较小,算法添加的噪声较小时可能会破坏梯度估计的无偏性;可是如果不对梯度进行裁剪,大量的噪声添加到每个梯度会导致模型的可用性大大降低。在模型训练前期,梯度所包含的数据信息更多,因此可以对应添加更多的拉普拉斯噪声,使用较大的$C$ 的值,使得梯度裁剪后的模型偏差更小;而在模型训练后期,梯度所包含的数据信息相对较小了,如果还使用相同的$C$ ,会引入很多不必要的噪声。

因此我们根据训练轮数和层间贡献率动态调整梯度裁剪阈值$C$:在每次迭代中,该算法使用方差为$2 b^{2}$ 的拉普拉斯机制来计算噪声梯度$g t_{u}^{\prime}=g t_{u}+\frac{1}{\left|D_{i}^{t}\right|} \operatorname{Lap}\left(\frac{G S_{l}}{\epsilon_{j}}\right)$。噪声$2 b^{2}$ 的大小取决于一个个体在$l_{2}$规范下对$g$的最大影响,即$\delta$。由于对梯度的大小没有先验的约束,我们以$l_{2}$规范对每个梯度进行裁剪。因此,梯度向量$g$被$g t_{u}=g t_{u} / \max \left(1, \frac{\left\|g t_{u}\right\|}{C}\right)$取代,以达到裁剪阈值$C$。这种裁剪保证了如果$|g t_{u}\| \leq C$,那么$g t_{u}$将被保留,而如果$|g\|>C$,它将被裁减为梯度准则$C$。

\section{隐私性证明}
自适应差分SGD算法对线性变换函数进行了扰动,该函数满足$\left(\epsilon_{c}+\epsilon_{l}\right)$差分隐私。证明如下:

在贡献中添加的扰动为:
\begin{equation}\label{eq:贡献中添加的噪声}
\ddot{C}_{j}\left(x_{i}\right)=C_{j}\left(x_{i}\right)+\operatorname{Lap}\left(\frac{G S_{c}}{\epsilon_{c}}\right), j \in[1, u]
\end{equation}
它是满足$\epsilon_{c}$-差分隐私的。

贡献$G S_{c}$的敏感度为:
\begin{equation}\label{eq:贡献敏感度}
\begin{aligned}
G S_{c} &=\frac{1}{|D|} \sum_{j=1}^{u}\left\|\sum_{x_{i} \in D} C_{x_{i, j}}\left(x_{i}\right)-\sum_{x_{i}^{\prime} \in D^{\prime}} C_{x_{i, j}^{\prime}}\left(x_{i}^{\prime}\right)\right\|_{1} \\
&=\frac{1}{|D|} \sum_{j=1}^{u}\left\|C_{x_{n, j}}\left(x_{n}\right)-C_{x_{n, j}^{\prime}}\left(x_{n}^{\prime}\right)\right\|_{1} \\
& \leq \frac{2}{|D|} \max \sum_{j=1}^{u}\left\|C_{x_{i, j}}\left(x_{i}\right)\right\|_{1} \\
& \leq \frac{2 u}{|D|}
\end{aligned}
\end{equation}

其中,$u$和$|D|$分别表示属性和元组的数量,然后可以得到:
\begin{equation}\label{贡献数量和元组}
\begin{aligned}
\frac{\operatorname{Pr}(\ddot{C}(D))}{\operatorname{Pr}\left(\ddot{C}\left(D^{\prime}\right)\right)} &=\frac{\prod_{j=1}^{u} \exp \left(\frac{\epsilon_{c}\left\|\frac{1}{|D|} \sum_{x_{i} \in D} C_{j}\left(x_{i}\right)-\ddot{C}_{j}\left(x_{i}\right)\right\|_{1}}{G S_{c}}\right)}{\prod_{j=1}^{u} \exp \left(\frac{\epsilon_{c}\left\|\frac{1}{\left|D^{\prime}\right|} \sum_{x_{i}^{\prime} \in D^{\prime}} C_{j}\left(x_{i}^{\prime}\right)-\ddot{C}_{j}\left(x_{i}^{\prime}\right)\right\|_{1}}{G S_{c}}\right)} \\
&=\prod_{j=1}^{u} \exp \left(\frac{\epsilon_{c}}{|D| G S_{c}}\left\|C_{j}\left(x_{n}\right)-C_{j}\left(x_{n}^{\prime}\right)\right\|_{1}\right) \\
& \leq \prod_{j=1}^{u} \exp \left(\frac{\epsilon_{c}}{|D| G S_{c}} \max \left\|C_{j}\left(x_{n}\right)\right\|_{1}\right) \\
&=\exp \left(\epsilon_{c} \frac{\max _{x_{i} \in D} \sum_{j=1}^{u}\left\|C_{j}\left(x_{n}\right)\right\|_{1}}{|D| G S_{c}}\right) \\
& \leq \exp \left(\epsilon_{c}\right)
\end{aligned}
\end{equation}

因此,添加噪声后的属性贡献值是满足$\epsilon_{c}$-差分隐私的。

假设两个相邻的数据集$D_{i}^{t}$和$D_{i}^{t^{\prime}}$,其最后一个元组$x_{n}$和$x_{n}^{\prime}$不同,$z\left(D_{i}^{t}\right)$和$z\left(D_{i}^{t^{\prime}}\right)$分别为线性变换函数。

一般来说,我们把偏置项视为第一类数据属性,即:$x_{i,0}=b_{i}$。线性转换可以改写为:$\ddot{\mathbf{z}}_{x \in D_{i}^{t}}(\omega)=\ddot{\mathbf{x}} * \omega$。线性变换的敏感性$G S_{l}$如下:
\begin{equation}
\begin{aligned}
G S_{l} &=\sum_{a_{i} \in l_{1}} \sum_{j=1}^{u}\left\|\sum_{x_{i} \in D_{i}^{t}} x_{i, j}-\sum_{x_{i}^{\prime} \in D_{i}^{t^{\prime}}} x_{i, j}^{\prime}\right\|_{1} \\
&=\sum_{a_{i} \in l_{1}} \sum_{j=1}^{u}\left\|x_{n, j}-x_{n, j}^{\prime}\right\|_{1} \\
& \leq \sum_{a_{i} \in l_{1}} \sum_{j=1}^{u} \max _{x_{i} \in D_{i}^{t}}\left\|x_{n, j}\right\|_{1} \\
& \leq \sum_{a_{i} \in l_{1}} u
\end{aligned}
\end{equation}

其中,$a_{i} \in l_{1}$是指第一个隐藏层$l_{1}$中的神经元$a_{i}$,$u$是数据元组$x_{i} \in D_{i}^{t}$中的属性数。它包括两个调整因素:$f$和$p$,这两个超参数可以适时过滤多余的噪声,再通过超参数调整噪声大小和裁剪阈值后,属性的表达式如下:
\begin{equation}
\begin{aligned}
\tilde{x}_{i, j} &=[(1-f)+f * p] * \ddot{x}_{i, j}+f *(1-p) * x_{i, j} \\
&=[(1-f)+f * p]\left[x_{i, j}+\operatorname{Lap}\left(\frac{G S_{l}}{\epsilon_{j}}\right)\right]+[f *(1-p)] x_{i, j} \\
&=x_{i, j}+[(1-f)+f * p]\left[\operatorname{Lap}\left(\frac{G S_{l}}{\epsilon_{j}}\right)\right]
\end{aligned}
\end{equation}
然后我们可以得到:

\begin{equation}
\begin{aligned}
\frac{\operatorname{Pr}\left(\ddot{\mathbf{z}}_{D_{i}^{t}}(\omega)\right)}{\operatorname{Pr}\left(\ddot{\mathbf{z}}_{D_{i}^{t}}(\omega)\right)} &=\frac{\prod_{a_{i} \in l_{1}} \prod_{j=1}^{u} \exp \left(\frac{\epsilon_{j}\left\|\sum_{x_{i} \in D_{i}^{t}} x_{i, j}-\sum_{x_{i} \in D_{i}^{t}} \tilde{x}_{i, j}\right\|_{1}}{G S_{l}}\right)}{\prod_{a_{i} \in l_{1}} \prod_{j=1}^{u} \exp \left(\frac{\epsilon_{j}\left\|\sum_{x_{i}^{\prime} \in D_{i}^{t^{\prime}}} x_{i, j}^{\prime}-\sum_{x_{i}^{\prime} \in D_{i}^{t^{\prime}}} \tilde{x}_{i, j}^{\prime}\right\|_{1}}{G S_{l}}\right)} \\
& \leq \prod_{a_{i} \in l_{1}} \prod_{j=0}^{u} \exp \left(\frac{\epsilon_{j}}{G S_{l}}\left\|\sum_{x_{i} \in D_{i}^{t}} x_{i, j}-\sum_{x_{i}^{\prime} \in D_{i}^{t^{\prime}}} x_{i, j}^{\prime}\right\|_{1}\right) \\
& \leq \prod_{a_{i} \in l_{1}} \prod_{j=0}^{u} \exp \left(\frac{\epsilon_{j}}{G S_{l}} \max _{x_{i} \in D_{i}^{t}}\left\|x_{n, j}\right\|_{1}\right) \\
& \leq \exp \left(\epsilon_{l} \frac{\sum_{a_{i} \in l_{1}} u\left[\sum_{j=1}^{u} \frac{\left|\ddot{C}_{j}\right|}{\sum_{j=1}^{u}\left|\ddot{C}_{j}\right|}\right]}{G S_{l}}\right) \\
&=\exp \left(\epsilon_{l}\right)
\end{aligned}
\end{equation}

根据上述推倒证明可知,在联邦学习的神经网络中添加自适应噪声后,所上传的梯度是满足$\left(\epsilon_{c}+\epsilon_{l}\right)$差分隐私的。在满足差分隐私的基础上,在下一节我们结合差分隐私的组合定理和后处理定理设计动量组合机制,计算累积的隐私预算。

\section{隐私预算分析}
本章所提出的自适应差分隐私保护方案是通过在随机梯度下降算法上添加自适应的拉普拉斯扰动,保护数据的隐私性。在上一节我们已经证明了此算法满足$\left(\epsilon_{c}+\epsilon_{l}\right)$差分隐私,那另外一个非常重要的问题就是评估在训练过程中添加噪声所累积的隐私预算成本。在本节中,我们提出动量组合的概念,去计算算法迭代过程中添加噪声所累积的隐私预算成本。

根据差分隐私的并串行组合定理,被查询n次的数据的隐私预算将增加n倍。因此,我们希望查询次数越少越好,至少要有一个界限。在实验环境中,我们可以通过几次尝试确定一个相对理想的迭代次数,然后在每次迭代中平均分配隐私预算。然而,在实践中很难选择迭代的数量,因为任何尝试都会增加额外的隐私风险。当数量太小时,会发生预拟合,导致性能不佳;如果数量太大,注入的噪声会过大,这将影响模型的准确性。另一种尝试是使用等比例递增的注入噪声序列,这样无论我们有多少次迭代,我们都能找到有限的隐私预算\upcite{ref32}。

根据上文提出的差分隐私的并串行组合定理,我们设计了一个动量组合定理:
\begin{theorem}[动量组合定理]\label{动量组合定理}
假使存在算法$M_{i}$满足$\left(\varepsilon_{i}, \delta_{i}\right)$-差分隐私,那么对于$M_{[k]}=\left(M_{1}, M_{2}, \ldots, M_{k}\right)$,有$M_{[k]}$也是满足$(\varepsilon, \delta)$-差分隐私的,其中
$$
\delta=\sum_{i=1}^{k} B(i, k, p)\left[\Phi\left(\frac{H \sqrt{i}}{2 \sigma}-\frac{\varepsilon \sigma}{H \sqrt{i}}\right)-e^{\varepsilon} \Phi\left(-\frac{H \sqrt{i}}{2 \sigma}-\frac{\varepsilon \sigma}{H \sqrt{i}}\right)\right]
$$
\end{theorem}

我们采用朴素贝叶斯机制计算$\delta$可以得到:
\begin{equation}\label{eq:朴素贝叶斯}
\delta=\sum_{i=1}^{k} B(i, k, p)\left[\operatorname{Pr}\left[L_{d, d^{\prime}} * i>\varepsilon\right]-e^{\varepsilon} \operatorname{Pr}\left[L_{d^{\prime}, d} * i<-\varepsilon\right]\right]
\end{equation}

在此情况下,隐私损失变量$L_{M, d, d^{\prime}}$和$L_{M, d^{\prime}, d}$同时满足$N(\eta, 2 \eta)$分布,并且$\eta=$ $H^{2} / 2 \sigma^{2}$。因此可以采用动量组合定理这样表达隐私预算损失:
\begin{equation}\label{eq:隐私预算计算1}
\begin{aligned}
& \operatorname{Pr}\left[L_{d, d^{\prime}} * i>\varepsilon\right]=\operatorname{Pr}[N(\eta i, 2 \eta i)>\varepsilon] \\
=& \operatorname{Pr}\left[N(0,1)>\frac{-\eta i+\varepsilon}{\sqrt{2 \eta i}}\right]=\operatorname{Pr}\left[N(0,1)<\frac{\eta i-\varepsilon}{\sqrt{2 \eta i}}\right] \\
=& \operatorname{Pr}\left[N(0,1)<\sqrt{\frac{\eta i}{2}}-\frac{\varepsilon \sigma}{\sqrt{2 \eta i}}\right]=\Phi\left(\frac{H \sqrt{i}}{2 \sigma}-\frac{\varepsilon \sigma}{H \sqrt{i}}\right)
\end{aligned}
\end{equation}

然后,可以计算得到隐私预算:
\begin{equation}\label{eq:隐私预算计算2}
\delta=\sum_{i=1}^{k} B(i, k, p)\left[\Phi\left(\frac{H \sqrt{i}}{2 \sigma}-\frac{\varepsilon \sigma}{H \sqrt{i}}\right)-e^{\varepsilon} \Phi\left(-\frac{H \sqrt{i}}{2 \sigma}-\frac{\varepsilon \sigma}{H \sqrt{i}}\right)\right]
\end{equation}

因为随着算法迭代次数$T$的增加,$\Phi\left(\frac{H \sqrt{T}}{2 \sigma}-\frac{\varepsilon \sigma}{H \sqrt{T}}\right)-e^{\varepsilon} \Phi\left(-\frac{H \sqrt{T}}{2 \sigma}-\frac{\varepsilon \sigma}{H \sqrt{T}}\right)$也在增加,因此可以计算得到:

\begin{equation}\label{eq:隐私预算计算3}
\begin{aligned}
& \sum_{i=1}^{T} B(i, T, p)\left[\operatorname{Pr}\left[L_{d, d^{\prime}} * i>\varepsilon\right]-e^{\varepsilon} \operatorname{Pr}\left[L_{d^{\prime}, d} * i<-\varepsilon\right]\right] \\
\leq & \sum_{i=1}^{T} B(i, T, p)\left[\operatorname{Pr}\left[L_{d, d^{\prime}} * T>\varepsilon\right]-e^{\varepsilon} \operatorname{Pr}\left[L_{d^{\prime}, d} * T<-\varepsilon\right]\right] \\
\leq & \operatorname{Pr}\left[L_{d, d^{\prime}} * T>\varepsilon\right]-e^{\varepsilon} \operatorname{Pr}\left[L_{d^{\prime}, d} * T<-\varepsilon\right]
\end{aligned}
\end{equation}

当隐私预算$\delta$相同时,$\sigma_{1} \leq \sigma_{2} \leq \sigma_{3}$。替换$\sigma=\alpha H \sqrt{T} / \sqrt{2 \epsilon}$,则有
\begin{equation}\label{eq:隐私预算计算4}
\Phi(\sqrt{\epsilon / 2}(1 / \alpha-\alpha))-e^{\varepsilon} \Phi(-\sqrt{\epsilon / 2}(1 / \alpha+\alpha)) \leq \delta
\end{equation}

因此,$\sigma_{1} \leq \sigma_{3}=O(\sqrt{T})$。与之前的工作相比,我们的隐私预算能够在相同的迭代次数$T$,更低的上紧界,达到满足$\left(\epsilon_{c}+\epsilon_{l}\right)$的差分隐私。

\section{本章总结}
联邦学习通过本地客户端以分布式学习的形式,共同训练一个全局模型。虽然基本的分布式协作深度学习没有直接暴露参与者的隐私数据集,但是恶意攻击者仍然可以通过共享的参数等信息获得一定的隐私信息。 

本章详细介绍了如何在深度学习模型的随机梯度下降算法中添加自适应的拉普拉斯噪声。我们设计了一个自适应噪声添加的方案,在神经网络前向传播算法中,根据属性对于模型输出的贡献率注入不同隐私预算的噪声。与传统的注入噪声的方法相比,我们在相同的隐私保护程度下最大限度地提高了模型的准确性,并且证明了算法能满足$\left(\epsilon_{c}+\epsilon_{l}\right)$差分隐私。然后我们采用动量组合机制计算对梯度施加的噪声量大小,分析了模型整体的隐私预算。

然而,在客户端的本地数据集添加的噪声只能保证本地数据的匿名性,不能够防止外部攻击者针对通信信道的攻击。如果客户端在每次迭代中同时上传了大量的权重更新,中央云服务器仍然可以将它们链接在一起,推导出参数信息。而且,当参与一次迭代的客户端数量达到上千人时,会导致聚合任务升级成一个高维任务,隐私预算暴增。因此,下一章我们对联邦学习模型框架进行了改进,在现有的联邦学习模型上新增混洗器,实现联邦学习框架的隐私安全,提高整体联邦学习模型的精度。

