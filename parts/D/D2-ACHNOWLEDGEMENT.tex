{\fangsong
	\chapter*{致\qquad 谢}\vskip 2mm
	\vspace{-1cm}
		\large{

时光荏苒,岁月如梭,研究生的日子过得飞快,转眼间我的硕士研究生学习生涯即将接近尾声。在华东师范大学读研的这两年时光,我不但学习到了很多知识,也结识了许多良师益友,此时此刻,我的内心充满了无限的感慨。所谓饮水思源,在此我要向每位陪伴我,鼓励我,教导我的人表示由衷的感谢。

从2019年收到华东师范大学的研究生录取通知书,我满怀憧憬和抱负的来到华师大,来到上海可信计算实验室,有幸成为曹珍富老师的学生。
感谢实验室的各位老师们,他们不但为我们提供了优质的教学环境和资源,还创造了良好的学习氛围,通过一流的科研实力和丰富的科研热情带领我们学习最前沿的科研成果。为了充实我们的研究生生活,学院定期举办各种学术会议和活动,邀请到国内外知名学者给我们做讲座,让我们有机会接触到最新的科研成果。而且,无论是在科研还是生活上遇到问题,老师们都会耐心的给我们提建议,鼓励帮助我们一起克服这些困难。

研究生的时光是轻快而稍纵即逝的,和实验室同学、室友的朝夕相处是我最难忘的回忆。因为有室友高圆圆、陈少敏、冯世玲,宿舍的氛围一直是欢快的,我们早晨共同早起去图书馆自习,下课了去实验室读论文,空闲时间一起在操场打篮球,欢声笑语,常伴我们。三年时光里,我们彻夜未睡,通宵准备数模竞赛;早出晚归,一起在理科楼度过日日夜夜,都将成为我的学生时代美好的回忆。

同门情谊似手足之情,感谢实验室的各位同窗好友,吴楠、汤琦、陆鹏皓、李翔宇、任城东、李明冲等,是有你们的互励互助,我才得以开心努力而充实的度过了这段美好的研究生生活,希望以后仍然有机会共同努力、共同奋斗。

最后,非常感谢我的父母和家人一直以来对我的鼓励与陪伴。在研究生生涯的这两年,我更加深刻体会到未来自己身上所担负的责任,希望我在未来的工作中能兢兢业业,踏实负责,实现我的社会价值;在未来的生活中,希望我能多多陪伴我的父母以回报养育之恩。 

在这篇论文完成之际意味着三年的硕士生涯即将告一段落,而自己也将踏上人生的下一段旅程。回顾硕士三年的时光,非常有幸能成为华东师范大学的学子。非常庆幸能成为曹珍富老师的学生,非常庆幸能和实验室的大家成为朋友,这是人生中可与而不可求的经历。最后,也感谢各位评审和答辩的专家在百忙之中对我论文的指导,谢谢你们。
	}
	
	\vspace{0.2cm}
	
	\vspace{0.2cm} \hspace{9.8cm}
	何慧娴
	
	\hspace{9cm}  二零二壹年九月

} 