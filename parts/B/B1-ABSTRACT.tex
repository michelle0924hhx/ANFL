\vspace{-2.5cm}
\chapter*{\zihao{2}\heiti{摘~~~~要}}
%\vskip 1cm
%\vspace{-1cm}

随着人工智能的快速发展与移动设备的普及,需要多个参与方协作的应用场景不断涌现,分布式数据处理和分布式机器学习的作用日益凸显。比如分散在多个银行的金融数据、不同医院里的医疗记录、大平台下的每个用户的行为记录,以及智能电表、传感器或移动设备等产生的数据都需要分布式处理与挖掘。数据孤岛是分布式数据处理和分布式机器学习面临的重要挑战之一,作为解决数据孤岛的解决方案,联邦学习是一种很有前景的分布式计算框架,可以在多个分散的边缘设备上本地训练模型,而无需将其数据传输到服务器。随着公民隐私意识的提高和相法律的完善,联邦学习中的隐私安全问题也日益受到人们的关注,且最新的研究工作表明已经能通过对模型的梯度参数进行攻击,还原用户的隐私数据,即仅通过保持数据的局部性来保护隐私是不够的,并且隐私保护技术在保护隐私的同时,还会牺牲模型精度。为此,本文使用差分隐私技术来保护联邦学习中用户的隐私,并针对分布式场景,分析模型训练过程中针对梯度下降算法的自适应干扰机制,实现提高模型精度的目的,并提出安全混洗模型,防止恶意服务器的攻击。本文主要工作包括如下几个方面:

本文主要的工作和贡献如下:
\begin{enumerate}
\item 在联邦学习差分隐私的场景下,本文提出了一种新型的、基于本地差分隐私的权重分配自适应干扰算法。在客户端本地训练的神经网络模型中,通过分析前向传播算法,计算每个属性类对于模型输出的贡献比,然后,我们设计了一个自适应噪声添加的方案,根据贡献率注入不同隐私预算的噪声。与传统的注入噪声的方法相比,我们在相同的隐私保护程度下最大限度地提高了模型的准确性,减少噪声对模型输出结果的影响,提高模型精度。
\item 考虑到联邦学习中参数聚合器的攻击,本文提出了一种新的安全聚合机制,在本地客户端和中心服务器之间新增混洗器,在用户将参数上传到云服务器之前,先对参数进行混洗,模型参数的更新被匿名的发送到混洗器,通过对模型参数的拆分和混洗实现客户端匿名,并且证明了安全混洗模型的可行性。
\item 本文通过实验,展示了自适应本地差分隐私方案和安全混洗框架的结合,使得联邦学习的模型的精度和隐私预算达到平衡。
\end{enumerate}
\hspace{-0.5cm}
\sihao{\heiti{关键词:}} \xiaosi{联邦学习,隐私保护,本地差分隐私,安全混洗}
