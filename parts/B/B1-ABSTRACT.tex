\vspace{-2.5cm}
\chapter*{\zihao{2}\heiti{摘~~~~要}}
%\vskip 1cm
%\vspace{-1cm}

近年来,机器学习由于其卓越的性能已被越来越多的应用于各个领域,如自动驾驶,人脸识别,个人信用评估等。但机器学习模型一般都为黑盒,由于其不可见性与不可解释性,使得人们对它的安全性有了很大的担忧。因此,越来越多的研究人员投身到机器学习模型的安全性验证的方法和工具的研究中。

与神经网络模型一样,树模型也容易受到对抗性样本的攻击。树模型的“脆弱性”在使其在某些安全性要求较高的应用中造成了隐患。有些情况下,甚至可能导致灾难性的后果。为此,我们研究了树模型的鲁棒性验证问题。
本文主要的工作和贡献如下:
\begin{enumerate}
	\item 我们提出了一个基于SMT技术的树模型鲁棒性验证框架,它可以有效的验证树模型的两个重要组成部分:随机森林和 GBDT 模型的鲁棒性,并且支持规模较大的模型的验证。该框架的核心思想是将树模型的鲁棒性验证问题转化为SMT公式的约束求解问题。
	\item 在验证的基础上,我们进一步对树模型鲁棒性的可解释性问题进行了研究,提出了鲁棒特征集合和局部鲁棒特征重要度的概念来描述模型鲁棒性与样本特征的内在联系,从而为对抗性样本攻击提供了新的思路。
	\item 我们基于三个基准测试集评估了框架的可行性和有效性,并且在实验中讨论了模型训练超参数与其鲁棒性的关系,从而为训练阶段提高模型的鲁棒性提供了重要参考。
\end{enumerate}
\hspace{-0.5cm}
\sihao{\heiti{关键词:}} \xiaosi{鲁棒性验证,可解释性,随机森林,GBDT,SMT}
